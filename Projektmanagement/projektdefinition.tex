% !TEX root = projectmanagement.tex
Im ersten Abschnitt dieses Dokuments soll die Projektdefinition sowie die Entstehung des Projekts erläutert werden. Zudem soll kurz auf die die generelle Idee zur Umsetzung eingegangen werden.

\subsection{Grundlagen des Projekts}
Die Studienarbeit mit dem Titel \textbf{HIER TITEL} wird von Marc Mahler und Marvin Zerulla verfasst. Die Betreuung übernimmt dabei Prof. PhD. Kay Berkling. Neben der Implementierung des Spiels sollen auch die verschiedenen Faktoren der Gamification analysiert und evaluiert werden. Zudem gilt es den Lernerfolg für die Spieler zu sichern.

\subsection{Zielsetzung}
Wie in den Grundlagen bereits beschrieben, soll das Spiel einen Lernerfolg für die Spieler erbringen und gleichzeitig Spaß bereiten.
Das spielerische Lernen soll für den Spieler in den Hintergrund rücken und dazu führen, dass es unbewusst geschieht.

\subsection{Grundsätzliche Projekt-/Spielidee}


\subsection{Aufbau}
Neben den bereits erwähnten wissenschaftlichen Evaluationen über Gamification und spielerisches Lernen wird das Spiel in einer gängigen Spiel-Engine entwickelt.
Die Entscheidung, welche Game-Engine verwendet werden wird, wird im Laufe der Bearbeitung der Studienarbeit evaluiert und kann somit noch nicht niedergeschrieben werden. Der \enquote{Gewinner} der Evaluation wird zur Impemenierung des Spiels verwendet.
Am Ende der Implementierung soll das entstandene Spiel in die bereits vorhandene Lernplattform von Prof. PhD. Kay Berkling eingearbeitet und über diese vermarktet werden.
Ein zusammenfassendes Ergebnis der beschriebenen Evaluationen befindet sich im Kapitel Qualitätssicherung.