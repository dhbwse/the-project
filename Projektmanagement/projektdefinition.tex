% !TEX root = projectmanagement.tex
Im ersten Abschnitt dieses Dokuments soll die Projektdefinition sowie die Entstehung des Projekts erläutert werden. Zudem soll kurz auf die die generelle Idee zur Umsetzung eingegangen werden.

\subsection{Grundlagen des Projekts}
Die Studienarbeit mit dem Titel \textbf{Konzeption und Entwicklung eines educational games - Spielerisch Schulwissen vermitteln auf Android} wird von Marc Mahler und Marvin Zerulla verfasst. Die Betreuung übernimmt dabei Prof. PhD. Kay Berkling. Neben der Implementierung des Spiels sollen auch die verschiedenen Faktoren der Gamification analysiert und evaluiert werden. Zudem gilt es den Lernerfolg für die Spieler zu sichern.

\subsection{Zielsetzung}
Wie in den Grundlagen bereits beschrieben, soll das Spiel einen Lernerfolg für die Spieler erbringen und gleichzeitig Spaß bereiten.
Das spielerische Lernen soll für den Spieler in den Hintergrund rücken und dazu führen, dass es unbewusst geschieht.

\subsection{Grundsätzliche Projekt-/Spielidee}
Um zur endgültigen Spielidee zu gelangen wurden mehrere Ideen ausgearbeitet und gegen unsere Kriterien abgewogen.
	Die Finale Spielidee wurde so ein Rennspiel in welchem verschiedene Dinge (Beispielsweise: ein Würfel, ein Auto oder eine Rakete) am Rennen teilnehmen sollen. Um Lerninhalte in ein Rennspiel einzubauen wurden mehrere Möglichkeiten ausgearbeitet:
	\begin{itemize}
		\item{ Aufgabe zum Start des Rennens. Hier kann der schnellste Spieler als erstes starten. Durch eine maximale Zeit und mehrere Versuche sollte Rücksicht auf Spieler genommen werden, welche es nicht schaffen die Aufgabe zu lösen. }
		\item{ Aufgaben in Form von Weggabelungen (wie beispielsweise Tore oder Schanzen). Eine Aufgabe erscheint rechtzeitig\footnote{Was in diesem Kontext rechtzeitig bedeutet wird noch definiert.} vor der Gabelung auf dem Bildschirm. Bei jeder Abzweigungsmöglichkeit steht eine Lösung zur Aufgabe. Fährt ein Spieler falsch  bekommt er einen Nachteil oder alle Spieler die korrekt fahren einen Vorteil. }
		\item{ Learn-To-Win: Spieler können durch lösen von Aufgaben virtuelles Geld verdienen und gewisse Vorteile im Spiel durch Einsatz des Geldes genießen. Diese Idee ist von dem Pay-To-Win Monetarisierungsmodell abgeleitet. }
	\end{itemize}
	Ein wichtiges Kriterium für die Spielidee war, dass Lerninhalte ein Teil vom Spiel sind und so auch das Lernen selbst durch Spaß angeregt wird.
	Das Spiel selbst soll in mindestens zwei Spielmodi spielbar sein:
	\begin{itemize}
		\item{ Online Multiplayer: Rennen fahren mit anderen echten Personen über das Internet. }
		\item{ Offline Zeitrennen: auf Strecken gegen die Zeit fahren. }
		\item{ Optional: Offline gegen Computer: wie das Onlinespiel nur mit Computergegnern. }
	\end{itemize}


\subsection{Aufbau}
Neben den bereits erwähnten wissenschaftlichen Evaluationen über Gamification und spielerisches Lernen wird das Spiel in einer gängigen Spiel-Engine entwickelt.
Die Entscheidung, welche Game-Engine verwendet werden wird, wird im Laufe der Bearbeitung der Studienarbeit evaluiert und kann somit noch nicht niedergeschrieben werden. Der \enquote{Gewinner} der Evaluation wird zur Impemenierung des Spiels verwendet.
Am Ende der Implementierung soll das entstandene Spiel in die bereits vorhandene Lernplattform von Prof. PhD. Kay Berkling eingearbeitet und über diese vermarktet werden.
Ein zusammenfassendes Ergebnis der beschriebenen Evaluationen befindet sich im Kapitel Qualitätssicherung.

% !TEX root = Projektmanagement/projectmanagement.tex
\pagebreak
\subsection{Terminplan}
\begin{tabular}{|p{6cm}|p{3.5cm}|c|c|}
\hline
\multicolumn{1}{|c|}{\textbf{Art des Termins}}               & \centering{\textbf{Teilnehmer}}                                  &              \textbf{Startdatum} & \textbf{Enddatum} \\ \hline
Themenbekanntgabe durch Prof. PhD. Kay Berkling              & \centering{Kay Berkling, Marc Mahler}                            & 2016-05-11          & 2016-05-11        \\ \hline
Suchen von Literatur                                         & \centering{Marvin Zerulla}                                       & 2016-05-15          & 2016-11-03        \\ \hline
Offizielle Themenbekanntgabe                                 & \centering{Marc Mahler, Marvin Zerulla}                          & 2016-10-19          & 2016-10-19        \\ \hline
Mehrere Brainstorming  Sessions                              & \centering{Marc Mahler, Marvin Zerulla}                          & 2016-05-11          & 2016-11-03        \\ \hline
Kickoff-Meeting                                              & \centering{Marc Mahler, Marvin Zerulla, Prof. PhD. Kay Berkling} & 2016-11-03          & 2016-11-03        \\ \hline
Erstes Teammeeting                                           & \centering{Marc Mahler, Marvin Zerulla}                          & 2016-11-06          & 2016-11-06        \\ \hline
Layout und Outline bei Prof. PhD. Kay Berkling vorlegen      & \centering{Marc Mahler, Marvin Zerulla, Prof. PhD. Kay Berkling} & 2016-11-10          & 2016-11-10        \\ \hline
Fertigstellung der  Einleitung und  Definition               & \centering{Marc Mahler, Marvin Zerulla}                          & 2016-11-17          & 2016-11-17        \\ \hline
Finale Wahl des Titels \& zweites Teammeeting				 & \centering{Marc Mahler, Marvin Zerulla}							&
2016-11-17 			& 2016-11-17		\\ \hline
Fertigstellung der Grundlagen                                & \centering{Marc Mahler, Marvin Zerulla}                          & 2016-12-01          & 2016-12-01        \\ \hline
\end{tabular}

\begin{tabular}{|p{6cm}|p{3.5cm}|c|c|}
\hline
\multicolumn{1}{|c|}{\textbf{Art des Termins}}               & \centering{\textbf{Teilnehmer}}                                  &              \textbf{Startdatum} & \textbf{Enddatum} \\ \hline
40\% Text fertig                                             & \centering{Marc Mahler, Marvin Zerulla}                          & 2016-12-22          & 2016-12-22        \\ \hline
First draft: 80\% Text fertig                                & \centering{Marc Mahler, Marvin Zerulla}                          & 2017-01-15          & 2017-01-15        \\ \hline
Feedback round: final for Berkling                           & \centering{Marc Mahler, Marvin Zerulla, Prof. PhD. Kay Berkling} & 2017-05-1           & 2017-05-1         \\ \hline
Finale Abgabe der Studienarbeit                              & \centering{Marc Mahler, Marvin Zerulla}                          &  2017-05-15          & 2017-05-15        \\ \hline
\end{tabular}






\subsection{Risikomanagement}
\begin{tabular}{|p{4cm}|p{3cm}|p{3.5cm}|p{3.5cm}|}
\hline
{\textbf{Wahrscheinlichkeit / Schwere}} & \multicolumn{1}{c|}{\textit{\textbf{gering}}}  & \multicolumn{1}{c|}{\textit{\textbf{kritisch}}}  	& \multicolumn{1}{c|}{\textit{\textbf{katastrophal}}} \\ \hline
\textit{\textbf{(Sehr) wahrscheinlich}} &
\cellcolor[HTML]{FCFF2F}Teammitglied keine Zeit für Meetings 	& \cellcolor[HTML]{FE0000}Missverständnisse mit dem Betreuer & \cellcolor[HTML]{FE0000}schwerer Fehler in Zeitplanung / Datenverlust     \\ \hline
\textit{\textbf{Möglich}}               &
\cellcolor[HTML]{34FF34}Missverständnisse im Team            	& \cellcolor[HTML]{F8FF00}Ausfall eines Teilnehmers durch Krankheit / Kunde ist nicht zufrieden / Spiel wird nicht gekauft / Persönliche Konflikte im Team & \cellcolor[HTML]{F8FF00}Gewählte Engine funktioniert nicht wie gedacht / Kompletter Hardwareausfall 				  \\ \hline
\textit{\textbf{(Sehr) unwahrscheinlich}}&
\cellcolor[HTML]{34FF34} & \cellcolor[HTML]{34FF34}Systemfehler in verwendeten Geräten & \cellcolor[HTML]{F8FF00}Exmatrikulation eines Teilnehmers                                               \\ \hline
\end{tabular}

Die Risiken, welche in grün markiert sind, haben nicht das Potential, das Projekt zu gefährden. Systemfehler in den verwendeten Geräten können in kürzester Zeit gefixt werden, alternativ stehen genügend Austauschgeräte für die Programmierung und den Test der Anwendung zur Verfügung. Da beide Teammitglieder gut miteinander auskommen können Misverständnisse im Team schnell und effektiv beseitigt werden.
Die Risiken, welche in gelb markiert sind haben das Potential, den Fortschritt und den Erfolg des Projekt zu gefährden, bzw. zu verzögern. Falls die Teammitglieder sich nicht auf Meetings einigen können, werden wichtige Aspekte der Kommunikation ignoriert. Dabei müsste dann auf Chats zurückgegriffen werden, was wiederum Potential für Missverständnisse liefert. Daher sollten beide Teilnehmer versuchen, sich auf Termine zu einigen.
Ein Ausfal eines Teilnehmers durch Krankheit ist nicht auszuschließen und muss dementsprechend über Mehrarbeit des gesunden Teammitglieds aufgeholt werden. Zur Vermeidung eines solchen Falls ist der Zeitrahmen des Projekts recht großzügig gewählt. Im Fall von persönlichen Konflikten zwischen den Teammitgliedern steht eine Schlichtung an erster Stelle. Da wie bereits erwähnt aber ein gutes Verhältnis zwischen beiden Teilnehmern herscht, sollte dies im Fall des Auftretens in annehmbarere Zeit aus der Welt geschafft sein. Für den Fall, dass das Spiel nicht gekauft wird bzw. die Kunden nicht zufrieden sind, ist das Projekt nicht gefährdet. Da keine großen Investitionen in das Projekt geflossen sind, ist ein negativer Ausgang des Projekts faktisch nicht möglich. Da durch eine negative Meinung der Kunden über das Spiel allerdings das Ziel der Arbeit verfehlt wurde (Spieler sollen etwas lernen), ist dies als kritisch zu betrachten.
Für den Fall, dass die gewählte Spielengine nicht wie gedacht funktioniert oder ein Teammitglied einen vollständigen Hardwareausfall erleidet, kann das Projekt stark zurückgeworfen werden. Aus diesem Grund ist dieses Risiko bei katastrophalen Folgen eingestuft. Da dieses Risiko zwar möglich ist, jedoch eher zu unwahrscheinlich tendiert, ist es ebenfalls als gelb eingestuft.
Eine Exmatrikulation eines Teammitglieds würde ebenfalls zu einem harten Rückfall des Projekts führen, da dies allerdings sehr unwahrscheinlich ist, wurde das Risiko ebenfalls als gelb eingestuft.
Die Risiken, welche das Projekt am ehesten gefährden können, sind als rot markiert. Da Missverständnisse mit dem Betreuer aufgrund verschiedener Ansichten sehr wahrscheinlich sind und diese kritische Auswirkungen haben können ist diese Einstufung sehr einfach gefallen. Zur Risikovermeidung wurde in diesem Fall sehr detailliert mit dem Betreuer über das Projekt gesprochen und alle Fakten klar ausgelegt.
Ein schwerer Fehler in der Zeitplanung ist ein nicht zu verachtendes Risiko, welches sehr wahrscheinlich auftritt und zu katastrophalen Folgen führen kann. Aus diesem Grund wurde zur Risikovermeidung ein genaues Netz aus Deadlines gelegt, an denen überprüft werden kann, ob der Zeitplan realisierbar ist. Im Fall eines Datenverlusts würde das Projekt zum Start zurückgeworfen. Um dies zu vermeiden wurde GitHub als Backup verwendet.

\subsection{Arbeitsmittel}

\subsection{Durchführungsplanung}