% !TEX root = projectmanagement.tex
Im ersten Abschnitt dieses Dokuments soll die Projektdefinition sowie die Entstehung des Projekts erläutert werden. Zudem soll kurz auf die die generelle Idee zur Umsetzung eingegangen werden.

\subsection{Grundlagen des Projekts}
Die Studienarbeit mit dem Titel \textbf{HIER TITEL} wird von Marc Mahler und Marvin Zerulla verfasst. Die Betreuung übernimmt dabei Prof. PhD. Kay Berkling. Neben der Implementierung des Spiels sollen auch die verschiedenen Faktoren der Gamification analysiert und evaluiert werden. Zudem gilt es den Lernerfolg für die Spieler zu sichern.

\subsection{Zielsetzung}
Wie in den Grundlagen bereits beschrieben, soll das Spiel einen Lernerfolg für die Spieler erbringen und gleichzeitig Spaß bereiten.
Das spielerische Lernen soll für den Spieler in den Hintergrund rücken und dazu führen, dass es unbewusst geschieht.

\subsection{Grundsätzliche Projekt-/Spielidee}
Um zur endgültigen Spielidee zu gelangen wurden mehrere Ideen ausgearbeitet und gegen unsere Kriterien abgewogen.
	Die Finale Spielidee wurde so ein Rennspiel in welchem verschiedene Dinge (Beispielsweise: ein Würfel, ein Auto oder eine Rakete) am Rennen teilnehmen sollen. Um Lerninhalte in ein Rennspiel einzubauen wurden mehrere Möglichkeiten ausgearbeitet:
	\begin{itemize}
		\item{ Aufgabe zum Start des Rennens. Hier kann der schnellste Spieler als erstes starten. Durch eine maximale Zeit und mehrere Versuche sollte Rücksicht auf Spieler genommen werden, welche es nicht schaffen die Aufgabe zu lösen. }
		\item{ Aufgaben in Form von Weggabelungen (wie beispielsweise Tore oder Schanzen). Eine Aufgabe erscheint rechtzeitig\footnote{Was in diesem Kontext rechtzeitig bedeutet wird noch definiert.} vor der Gabelung auf dem Bildschirm. Bei jeder Abzweigungsmöglichkeit steht eine Lösung zur Aufgabe. Fährt ein Spieler falsch  bekommt er einen Nachteil oder alle Spieler die korrekt fahren einen Vorteil. }
		\item{ Learn-To-Win: Spieler können durch lösen von Aufgaben virtuelles Geld verdienen und gewisse Vorteile im Spiel durch Einsatz des Geldes genießen. Diese Idee ist von dem Pay-To-Win Monetarisierungsmodell abgeleitet. }
	\end{itemize}
	Ein wichtiges Kriterium für die Spielidee war, dass Lerninhalte ein Teil vom Spiel sind und so auch das Lernen selbst durch Spaß angeregt wird.
	Das Spiel selbst soll in mindestens zwei Spielmodi spielbar sein:
	\begin{itemize}
		\item{ Online Multiplayer: Rennen fahren mit anderen echten Personen über das Internet. }
		\item{ Offline Zeitrennen: auf Strecken gegen die Zeit fahren. }
		\item{ Optional: Offline gegen Computer: wie das Onlinespiel nur mit Computergegnern. }
	\end{itemize}


\subsection{Aufbau}
Neben den bereits erwähnten wissenschaftlichen Evaluationen über Gamification und spielerisches Lernen wird das Spiel in einer gängigen Spiel-Engine entwickelt.
Die Entscheidung, welche Game-Engine verwendet werden wird, wird im Laufe der Bearbeitung der Studienarbeit evaluiert und kann somit noch nicht niedergeschrieben werden. Der \enquote{Gewinner} der Evaluation wird zur Impemenierung des Spiels verwendet.
Am Ende der Implementierung soll das entstandene Spiel in die bereits vorhandene Lernplattform von Prof. PhD. Kay Berkling eingearbeitet und über diese vermarktet werden.
Ein zusammenfassendes Ergebnis der beschriebenen Evaluationen befindet sich im Kapitel Qualitätssicherung.

% !TEX root = Projektmanagement/projectmanagement.tex
2016-10-19	Themenbekanntgabe
2016-11-03	erstes gespräch mit berkling
2016-11-06	Team Meeting \#1
2016-11-10	layout/outline absprechen, einleitung fertig
2016-11-17	?
2016-11-24	Grundlagen fertig
2016-12-01	?
2016-12-08	?
2016-12-15	?
2016-12-22	40\% text (28pages)
2016-12-29	?
2017-01-15	first draft: 80\% text  (56pages)
??	presentation of your final working tool, most likely to a software engineering course of the year below
2017-05-1	feedback round - final for Berkling
2017-05-15	Final


\subsection{Risikomanagement}

\subsection{Arbeitsmittel}

\subsection{Durchführungsplanung}