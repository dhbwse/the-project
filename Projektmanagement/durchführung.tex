% !TEX root = projectmanagement.tex
	Die Durchführung der Studienarbeit geschieht im Wesentlichen in zwei Abschnitten. Im ersten Abschnitt geht es darum, die schriftliche Leistung zu erbringen, sprich die Studienarbeit zu schreiben. Der zweite Abschnitt ist die Implementierung des Spiels. Da die Implementierung des Spiels jedoch auch fester Bestandteil des Schriftlichen ist, wird dieser nicht extra betrachtet.

	\subsection{Verfassen der Studienarbeit}
	Für das Verfassen der Studienarbeit wurde zunächst eine grobe Gliederung erstellt. Diese wurde dann gerecht unter beiden Projektteilnehmern aufgeteilt, um eine Aufteilung des Workloads in zwei in etwa gleichgroße Teile zu schaffen. Da eine Anforderung von Prof. PhD. Kay Berkling besagt, bis zum 15.01.2017 ca. 80\% des Textes verfasst zu haben, hatte dies für die Durchführung Priorität.
	Unter Beachtung der zu Beginn ausgewählten wissenschaftlichen Grundlagen konnten die für das Projekt nötigen Grundlagen schnell erläutert und belegt werden. Diese Grundlagen beschäftigen sich unter Anderem mit:
	\begin{itemize}
		\item{Android als Zielplattform für die Anwendung}
		\item{Einer Evaluation verschiedener Game-Engines}
		\item{Einer Auswahl verschiedener Lernziele}
		\item{Einer Evaluation unterschiedlicher Gamification-Modelle}
	\end{itemize}

	Nachdem alle Grundlagen geklärt und wissenschaftlich belegt waren, stand die Konzipierung des Spiels im Fokus. Dabei spielen viele Aspekte aus dem Bereich Game-Design, sowie einige technische Details eine Rolle.
	Zuerst wurden alle nötigen Kriterien für die Spielidee zusammengetragen. Auf Basis dieser Kriterien konnte dann in mehreren langen Brainstorming-Sessions eine finale Spielidee entwickelt werden, welche bereits in dieser Ausarbeitung erwähnt wurde.
	Zur Spezifizierung dieser Spielidee war es im Anschluss nötig, die wichtigsten Aspekte der Spielmechanik zusammenzutragen. Einige Beispiele, welche zu definieren waren sind:
	\begin{itemize}
		\item{Regeln zum Spielfortschritt}
		\item{Regeln innerhalb der Levelpakete}
		\item{Gewinnbedingungen eines Rennens}
		\item{Zusatzaufgaben innerhalb eines Rennens}
		\item{Oberflächendesign und Kameraperspektive}
	\end{itemize}
	Zusätzlich ging es um die Implementierung einer Ingame-Währung, verschiedene Interaktionen der Strecke mit dem Nutzer etc.
	Mit der erfolgreichen Definition jeder Kleinigkeit, welche Relevanz für die Implementierung hat, ging es um die Entwicklung der bereits erwähnten Play-Personas, also einer Zusammenfassung der Eigenschaften verschiedener Nutzer des Spiels. Auf Basis dieser Personas konnten dann die Lernziele definiert werden, um möglichst alle Personas zu befriedigen. Dies ist mit gutem Erfolg gelungen und kann in der Studienarbeit nachgelesen werden. Zum Abschluss der Konzipierung musste eine Evaluation aller möglichen Game-Engines durchgeführt werden. Der Gewinner dieser Evaluation ist die \enquote{Unity-Engine}.

	Der folgende Bereich beschäftigt sich mit der Implementierung des Spiels selbst. Da dies aufgrund bereits erwähnter Fristen bisher noch nicht stattgefunden hat, kann an dieser Stelle lediglich das geplante Vorgehen erläutert werden, welches im Rahmen des Schreibens bestens ausgearbeitet wurde.

	Der erste Schritt der Entwicklung ist das Prototyping. Dabei geht es darum, eine erste Version des Spiels zu erstellen, ohne großen Programmieraufwand zu betreiben. Beim Prototyping spielen sowohl Grafik als auch Steuerung nur eine sekundäre Rolle. Für das Prototyping können Grafikmodelle verwendet werden, welche sich bereits in der Unity-Engine befinden. Unity liefert diverse Standardmodelle, welche sich für das Protoyping perfekt eignen.
	Im nächsten Schritt wird der Protoyp dann so lange überarbeitet, bis das Spiel den Erwartungen und Wünschen entspricht. Dabei ist es besonders wichtig, jeden Aspekt des Spiels zu überarbeiten. Da jedoch ein Protoyp vorhanden ist, sollte dies sehr einfach sein.
	Ein konkretes Vorgehen kann dabei aktuell nicht definiert werden, da es von der Art und dem Aufbau des Prototypen abhängig ist.

	Für die Studienarbeit wird es im Anschluss nötig, das Vorgehen zu erläutern. Dabei muss zunächst ein Klassendiagramm erstellt und der Arbeit beigefügt werden. Anschließend können wichtige Aspekte der Implementierung, sowie Besonderheiten bei der Umsetzunng etc. näher beleuchtet werden.
	Als letzten Schritt in der Implementierung muss das Spiel der bereits fertigen Lernplattform von Prof. PhD. Kay Berkling zugeführt werden. Dafür ist es nötig, die gegebenen Schnittstellen zu analysieren und ebenfalls eine Schnittstelle zu implementieren, welche die Verbindung ermöglicht.

	Der letzte große Abschnitt der Studienarbeit ist die sogenannte Evaluation. Dabei wird das fertige Spiel erneut anhand aller Requirements überprüft. Ist das Ergebnis positiv, kann die Implementierung des Spiels als erfolgreich angesehen werden.
	Im nächste Schritt werden nun die Lernziele mit dem tatsächlichen Lernerfolg durch das Spielen des Spiels verglichen. Dies kann beispielsweise durch verschiedene Testversuche mit Kindern und deren Eltern durchgeführt werden. Ist auch dies erfolgreich, kann die gesamte Studienarbeit als voller Erfolg verbucht werden.

	Somit bleibt als letzter Schritt der Durchführung lediglich das Fazit. Dabei werden die Ergebnisse der Evaluation erneut aufgegriffen und bewertet. Das Fazit wird gefolgt von einem kurzen Ausblick, welcher den Fortgang des Spiels betrachten soll.