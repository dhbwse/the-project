% !TEX root = projectmanagement.tex
Im Bereich der Projektplanung geht es darum, dem Projekt einen genauen Rahmen zu geben. Dieser reicht von den geplanten Terminen und Meilensteinen, über den Umgang mit potentiellen Risiken, bis hin zur Planung der benötigten Arbeitsmittel, der gewünschten qualitätssichernden Maßnahmen, und eine groben Planung für die Durchführung des Projekts.
% !TEX root = Projektmanagement/projectmanagement.tex
2016-10-19	Themenbekanntgabe
2016-11-03	erstes gespräch mit berkling
2016-11-06	Team Meeting \#1
2016-11-10	layout/outline absprechen, einleitung fertig
2016-11-17	?
2016-11-24	Grundlagen fertig
2016-12-01	?
2016-12-08	?
2016-12-15	?
2016-12-22	40\% text (28pages)
2016-12-29	?
2017-01-15	first draft: 80\% text  (56pages)
??	presentation of your final working tool, most likely to a software engineering course of the year below
2017-05-1	feedback round - final for Berkling
2017-05-15	Final

\subsection{Risikomanagement}
\begin{tabular}{|p{4cm}|p{3cm}|p{3.5cm}|p{3.5cm}|}
\hline
{\textbf{Wahrscheinlichkeit / Schwere}} & \multicolumn{1}{c|}{\textit{\textbf{gering}}}  & \multicolumn{1}{c|}{\textit{\textbf{kritisch}}}  	& \multicolumn{1}{c|}{\textit{\textbf{katastrophal}}} \\ \hline
\textit{\textbf{(Sehr) wahrscheinlich}} &
\cellcolor[HTML]{FCFF2F}Teammitglied keine Zeit für Meetings 	& \cellcolor[HTML]{FE0000}Missverständnisse mit dem Betreuer & \cellcolor[HTML]{FE0000}schwerer Fehler in Zeitplanung / Datenverlust     \\ \hline
\textit{\textbf{Möglich}}               &
\cellcolor[HTML]{34FF34}Missverständnisse im Team            	& \cellcolor[HTML]{F8FF00}Ausfall eines Teilnehmers durch Krankheit / Kunde ist nicht zufrieden / Spiel wird nicht gekauft / Persönliche Konflikte im Team & \cellcolor[HTML]{F8FF00}Gewählte Engine funktioniert nicht wie gedacht / Kompletter Hardwareausfall 				  \\ \hline
\textit{\textbf{(Sehr) unwahrscheinlich}}&
\cellcolor[HTML]{34FF34} & \cellcolor[HTML]{34FF34}Systemfehler in verwendeten Geräten & \cellcolor[HTML]{F8FF00}Exmatrikulation eines Teilnehmers                                               \\ \hline
\end{tabular}

Die Risiken, welche in grün markiert sind, haben nicht das Potential, das Projekt zu gefährden. Systemfehler in den verwendeten Geräten können in kürzester Zeit gefixt werden, alternativ stehen genügend Austauschgeräte für die Programmierung und den Test der Anwendung zur Verfügung. Da beide Teammitglieder gut miteinander auskommen können Misverständnisse im Team schnell und effektiv beseitigt werden.
Die Risiken, welche in gelb markiert sind haben das Potential, den Fortschritt und den Erfolg des Projekt zu gefährden, bzw. zu verzögern. Falls die Teammitglieder sich nicht auf Meetings einigen können, werden wichtige Aspekte der Kommunikation ignoriert. Dabei müsste dann auf Chats zurückgegriffen werden, was wiederum Potential für Missverständnisse liefert. Daher sollten beide Teilnehmer versuchen, sich auf Termine zu einigen.
Ein Ausfal eines Teilnehmers durch Krankheit ist nicht auszuschließen und muss dementsprechend über Mehrarbeit des gesunden Teammitglieds aufgeholt werden. Zur Vermeidung eines solchen Falls ist der Zeitrahmen des Projekts recht großzügig gewählt. Im Fall von persönlichen Konflikten zwischen den Teammitgliedern steht eine Schlichtung an erster Stelle. Da wie bereits erwähnt aber ein gutes Verhältnis zwischen beiden Teilnehmern herscht, sollte dies im Fall des Auftretens in annehmbarere Zeit aus der Welt geschafft sein. Für den Fall, dass das Spiel nicht gekauft wird bzw. die Kunden nicht zufrieden sind, ist das Projekt nicht gefährdet. Da keine großen Investitionen in das Projekt geflossen sind, ist ein negativer Ausgang des Projekts faktisch nicht möglich. Da durch eine negative Meinung der Kunden über das Spiel allerdings das Ziel der Arbeit verfehlt wurde (Spieler sollen etwas lernen), ist dies als kritisch zu betrachten.
Für den Fall, dass die gewählte Spielengine nicht wie gedacht funktioniert oder ein Teammitglied einen vollständigen Hardwareausfall erleidet, kann das Projekt stark zurückgeworfen werden. Aus diesem Grund ist dieses Risiko bei katastrophalen Folgen eingestuft. Da dieses Risiko zwar möglich ist, jedoch eher zu unwahrscheinlich tendiert, ist es ebenfalls als gelb eingestuft.
Eine Exmatrikulation eines Teammitglieds würde ebenfalls zu einem harten Rückfall des Projekts führen, da dies allerdings sehr unwahrscheinlich ist, wurde das Risiko ebenfalls als gelb eingestuft.
Die Risiken, welche das Projekt am ehesten gefährden können, sind als rot markiert. Da Missverständnisse mit dem Betreuer aufgrund verschiedener Ansichten sehr wahrscheinlich sind und diese kritische Auswirkungen haben können ist diese Einstufung sehr einfach gefallen. Zur Risikovermeidung wurde in diesem Fall sehr detailliert mit dem Betreuer über das Projekt gesprochen und alle Fakten klar ausgelegt.
Ein schwerer Fehler in der Zeitplanung ist ein nicht zu verachtendes Risiko, welches sehr wahrscheinlich auftritt und zu katastrophalen Folgen führen kann. Aus diesem Grund wurde zur Risikovermeidung ein genaues Netz aus Deadlines gelegt, an denen überprüft werden kann, ob der Zeitplan realisierbar ist. Im Fall eines Datenverlusts würde das Projekt zum Start zurückgeworfen. Um dies zu vermeiden wurde GitHub als Backup verwendet.
\pagebreak
\subsection{Arbeitsmittel}
Die für die Durchführung der Studienarbeit benötigten Arbeitsmittel belaufen sich auf:
\begin{itemize}
	\item{Entwicklungsrechner: Zur Entwicklung der Anwendung werden die Privatrechner der Projektteilnehmer verwendet. Da diese bereits vorhanden sind, müssen keine Maschinen gestellt werden.}
	\item{Game-Engines: Die Game-Engines, welche für die Entwicklung zur Verfügung stehen und noch evaluiert werden müssen sind kostenlos verfügbar. Die Kosten waren dabei ein Auswahlkriterium. Somit muss für die Game-Engines ebenfalls kein Geld aufgewendet werden.}
	\item{Models und Grafiken: Für Models und Grafiken wurde ein externer Grafiker hinzugezogen. Weiterhin können die Projektteilnehmer auf eine Sammlung aus Models und Grafiken zurückgreifen, welche im Zuge eines anderen Projekts bereits gekauft wurden. Die so entstehnden Kosten werden von der DHBW übernommen.}
\end{itemize}
Da die Abdeckung der Kosten durch die DHBW gewährleistet ist, muss lediglich darauf geachtet werden, den gesetzten Rahmen nicht zu überschreiten. Da der ausgewählte Grafiker nur einen vergleichsweise geringen Stundenlohn für seine Arbeit fordert ist dies grundsätzich gewährleistet.

\subsection{Maßnahmen zur Qualitätssicherung}
Zur Sicherung der Qualität während der Projektdurchführung werden mehrere Maßnahmen durchgeführt:
\begin{itemize}
	\item{Regelmäßige Tests: Zur Sicherung der Qualität des Programmcodes werden regelmäßige, automatisierte Tests realisiert. Diese können über Build-Tools wie Gradle oder Maven, sowie über Verwendung von Online-Diensten wie beispielsweise TravisCI realisiert werden. Dabei werden bei jedem Bau der Anwendung diese Tests automatisch durchgeführt.}
	\item{Code-Analyse: Mittels Code-Analyse Tools besteht die Möglichkeit, den Programmcode auf Qualitätsmerkmale zu analysieren. Diese Analysen werden ebenso automatisiert durchgeführt und die Ergebnisse werden den Projektteilnehmern in regelmäßigen Abständen zugesendet.}
	\item{Gegenlesen der Studienarbeitstexte: Wenn ein Textabschnitt der Studienarbeit nahezu fertiggestellt ist, wird dieser Text vom zweiten Projektteilnehmer, welcher logischerweise nicht der Verfasser ist, gegengelesen. Erst wenn beide Projektteilnehmer mit einem Textabschnitt zufrieden sind, wird er final in die Studienarbeit aufgenommen.}
\end{itemize}

\subsection{Zulieferungen}
Durch mehrere zwischenzeitliche Abgaben der bisherigen Studienarbeitsergebnisse bei Prof. PhD Kay Berkling wird der stetige Fortschritt des Projekts gewährleistet. Diese Abgaben werden zwar in unregelmäßigen Abständen, jedoch immer mit der Fertigstellung einer gewählten Thematik durchgeführt. Somit kann auch Prof. PhD. Berkling absichern, dass die Studienarbeit am Ende zu einem Erfolg wird.

\subsection{Durchführungsplanung}
Für die Zielgruppe des Spiels werden Personas ausgearbeitet, danach ein Gamification-Modell auf diese Personas zugeschnitten. Das Gamification-Modell soll Spieler anregen das Spiel länger zu spielen und so mehr zu lernen. Der Fokus liegt dabei stark auf der Verbindung von Gamification und spielerischem Lernen.
Das Spiel soll in einer gängigen Game-Engine entwickelt werden. Um eine passende Game-Engine zu wählen, wird eine Evaluation durchgeführt.
Die Entwicklung des Spiels geschieht über anfängliches Prototyping und eine anschließenden Implementierung in der Game-Engine. Am Ende der Implementierung soll das entstandene Spiel in die bereits vorhandene Lernplattform von Prof. PhD. Kay Berkling eingegliedert und über diese vertrieben werden.
\pagebreak