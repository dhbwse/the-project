Das letzte Kapitel dieses Dokuments beschäftigt sich mit dem Projektabschluss. Dazu gehören zum einen die Übergabe des Projekts an den Auftraggeber, und zum anderen eine Sicherung der gemachten Erfahrungen.

\subsection{Übergabe}
Wie bereits erwähnt, wurde eine Frist bis zum 15. Januar gesetzt, um 80\% des Textes der Studienarbeit erreicht zu haben. Dadurch soll gewährleistet werden, dass im Anschluss genug Zeit für die Implementierung des Spiels verbleibt und die Zeitplanung einfacher fällt.
Aus diesem Grund kann zum aktuellen Zeitpunkt keine (Teil-)Übergabe des Software stattfinden. Die zu dokumentierende Übergabe betrifft somit ausschließlich den Text der Studienarbeit.
Um die Übergabe vor zu bereiten wurde bereits zum aktuellen Zeitpunkt ein Drittleser beauftragt, den aktuellen Stand der Studienarbeit zu überprüfen. Die von diesem Drittleser gefundenen Fehler konnten korrigiert werden.
Die Übergabe der Studienarbeit geschieht in Form eines PDF, welches per E-Mail an Prof. PhD. Kay Berkling übersendet wird. Auf ein Übergabeprotokoll wird an dieser Stelle verzichtet, da es sich lediglich um eine Teilübergabe handelt.
Über die Frist bis zum 15. Januar sagt Prof. PhD. Kay Berkling: \enquote{Die Deadline war dafür da, dass Sie gut fertig werden.}. Da dies keine finale Übergabe ist, wird von seiten der Dozentin keine Bewertung erfolgen. Die Übergabe dient lediglich der Überprüfung des aktuellen Standes.

\subsection{Erfahrungssicherung}
Die während der Durchführung der Studienarbeit gemachten Erfahrungen beziehen sich hauptsächlich auf das Risikomanagement und den Umgang mit Risiken.
Wie im Risikomanagement erwähnt, bestand ein hohes Risiko bei der Zeitplanung und der Umsetzung in der vorgegeben Zeit. Da beide Projektteilnehmer im Laufe des Dezembers auf Grund von Klausuren und anderen Zwischenfällen wenig Zeit für das Projekt aufbringen konnten, wurde im Dezember wenig geleistet.
Das Eintreten dieses Risikos hat dazu geführt, dass ein erreichen der gesetzten Deadline nur schwer möglich war. Durch konzentriertes Durcharbeiten in den ersten beiden Januarwochen konnte die Deadline zwar erreicht werden, dies war jedoch mit viel Stress verbunden.
Um aus diesem Risiko zu lernen, wurde beschlossen, eine zeitige Abarbeitung der anfallenden Aufgaben anzustreben. Zudem wäre es klug gewesen, Absprache darüber zu halten, wie mit dem geringeren Zeitfenster im Dezember umgegangen wird.
Somit hätte der Stress im Januar deutlich reduziert werden können. Für die Zukunft wird also vorgemerkt, anfallende Probleme unmittelbar zu besprechen und nicht zu verschieben, um weiteren Stress zu vermeiden.
Außerdem konnte durch Einsatz der Regeln zur Textqualität die allgemeine Art des Schreibens beeinflusst werden. Ich für meinen Teil achte beim Verfassen von Texten sehr viel mehr darauf, die gesetzten Metriken und Regeln direkt einzuhalten, um die Notwendigkeit der nachträglichen Korrektur zu vermeiden.
