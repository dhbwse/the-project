% !TEX root = Projektmanagement/projectmanagement.tex
\textbf{Mehrere Brainstorming-Sessions:} (Themenbekanntgabe bis 2016-11-03)\\
In mehreren kleinen Brainstorming-Sessions zwischen den beiden Bearbeitern der Studienarbeit, welche noch vor dem Kickoff-Meeting abgehalten wurden, konnten diverse Grundsätze festgesetzt werden:
\begin{itemize}
	\item min 2h Spielspaß
	\item educational game: Nicht offensichtlich ein Lernspiel
	\item Programmieraufwand: drei Monate (all inclusive)
	\item casual game
	\item originelle Idee (kein Klon)
	\item offline spielbar
	\item kein RPG im engeren Sinne
\end{itemize}

Folgende Ideen zu Spielen wurden in weiteren Brainstorming-Sessions erarbeitet:
\begin{enumerate}
	\item Chemie / Physik Simulation:
	\begin{itemize}
		\item Laborumgebung
		\item Einfache Experimente durchführbar
		\item Fehler in Experimenten führt zu scheitern
		\item Steuerung der Hände des Laborarbeiters
		\item Idee wegen schwerer Durchführbarkeit (Partikelsystem etc) sowie geringem Lernerfolg verworfen
	\end{itemize}
	\item Rätselspiel:
	\begin{itemize}
		\item Aufgrund fehlernder weiterer Eingebung verworfen
	\end{itemize}
	\item Simulation von Motoren und Getrieben
	\begin{itemize}
		\item Spieler bekommt zu fahrende Strecke gezeigt (2D Ansicht)
		\item Auswahl verschiedener Motoren möglich
		\item Auswahl des Treibstoffes möglich
		\item Gangschaltung während der Fahrt möglich
		\item Informationstexte über Motoren sowie Treibstoffe und Übersetzungen der Zahnräder
		\item Zeitrennen / Erreichen des Ziels als Spielziel
		\item Idee nach dem Kickoff-Meeting vollständig verworfen aufgrund unpassender Lernumsetzung
	\end{itemize}
	\newpage
	\item Rennfahrspiel mit Verkehrsregelaufgaben und Mini Aufgaben:
	\begin{itemize}
		\item wilde renntour durch die stadt
		\item Polizeiverfolgung und verschiedenen Animationen des Scheiterns (Death Sells)
		\item zu lösende Aufgabe bei jedem Scheitern
		\item Minigame mit verschiedenen aufgaben wie:
		\begin{itemize}
			\item ampel: es muss gebremst werden (mögliche Fail Animationen, fährt in ein Auto rein, bleibt stehen aber explodiert, fährt die Ampel um)
			\item Polizei: es muss rechts rangefahren werden
			\item Sprungschantze: Gas geben statt bremsen
			\item Kurven links / rechts
		\end{itemize}
		\item Nach dem Kickoff-Meeting verworfen
	\end{itemize}
\end{enumerate}


\textbf{Erstes Team-Meeting:} (2016-11-06)\\
Nach dem Kickoff-Meeting und der genauen Definiton der Anforderungen und Wünsche konnte eine genaue Spielidee entwickelt werden.
Das Ergebnis des Team-Meetings ist die zu Anfang erwähnte Spielidee, welche maßgeblich durch die vorherigen Ideen der Brainstorming-Sessions beeinflusst wurde. Dabei wurden die Ergebnisse dieser Sessions zusammengefasst und auf das gewünschte gefiltert.
\newline