% !TEX root = Projektmanagement/projectmanagement.tex
\textbf{Erstes Feedbackgespräch:} (2016-11-10)\\
Im zweiten Gespräch mit Prof. PhD. Kay Berkling stand die Absprache des Outlines sowie der Gliederung der Arbeit im Vordergrund. Zudem soll die im ersten Team-Meeting erarbeitete Spielidee präsentiert und abgesegnet werden.

Nachdem in Zusammenarbeit mit Prof. PhD. Kay Berkling die Strukturierung der Studienarbeit überarbeitet wurde, bis alle beteiligten zufrieden waren, wurde auf die Spielidee eingegangen.
Dabei wurde die bereits erwähnte Idee vorgestellt und von Prof. Berkling als gut geeignet abgesegnet.
Ebenso wurde das Outline der Arbeit sowie der gesetzte Terminplan als gut befunden und azeptiert.

Weitere Gesprächsinhalte betrafen:
\begin{enumerate}
	\item Gamification: Dabei ist darauf zu achten, dass kein vor definiertes Modell herausgesucht und verwendet werden soll. Durch die spezifischen Anforderungen der Arbeit ist es notwendig, den Spieler als Persona zu definieren und die wichtigsten Gamificationaspekte für diese Persona zu evaluieren
	\item Spielegrafik: Aufgrund des baldigen Kassenschlussees Ende November ist es wichtig, dass unter Umständen benötigte Modells für die Spielegrafik (dazu zählen generelle grafische Elemente sowie 3D Modelle) in nächster Zeit über die zuständigen Personen der DHBW gekauft werden müssen
	\item Verwendung von \enquote{youtrack}: Anstatt wie ursprünglich geplant JIRA zu verwenden, wurden wir von Prof. Berkling gebeten, die neue Software youtrack zu verwenden und zu berichten, wie diese im Vergleich zu JIRA verwendet werden konnte.
\end{enumerate}


