% !TEX root = dokumentation.tex
\section*{Abstract}
Dass Lernspiele funktionieren, ist mittlerweile eine akzeptierte Ansicht. Dennoch fehlen professionell umgesetzte Lernspiele, welche die komplette Bandbreite an Wissen vermitteln. Eine Kombination aus vielen Lernspielen könnte dies erreichen und so eine Schulausbildung erweitern bzw. teilweise ersetzen.
Diese Arbeit handelt von der Entwicklung eines Lernspieles, dabei werden Konzepte der Gamification, Personas, Psychologie und Game Design generell verwendet. Für die Arbeit werden gängige Lernmethoden und klassische Videospiele analysiert. Das Spiel wird konzipiert, entwickelt und über die Entwicklung hinweg bereits kontinuierlich evaluiert. Der Spielspaß steht dabei im Vordergrund der Entwicklung.
Ergebnis der Arbeit ist ein Rennspiel für Android. Das Spiel unterstützt beliebige Lernmaterialien, eignet sich aber vor allem für schnell abrufbares Faktenwissen.

\vfill

\section*{Abstract}
It is an accepted view that educational games work. However, there is still a lack of professionally implemented educational games, which teach the complete range of knowledge and thus are able to possibly replace school education.
This work deals with the development of an educational game, in which concepts of gamification, personas, psychology and generally game design are used. In this work, common learning methods and classical video game are analyzed. The game is designed, developed and continuously evaluated development. The Fun is the focus of the development.
A racing game for Android is the result of this work. Our game supports arbitrary learning materials, but is best suited for fast-retrievable factual knowledge.

\vfill\vfill\newpage
