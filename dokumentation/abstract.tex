% !TEX root = dokumentation.tex
\section*{Abstract}
Dass Lernspiele funktionieren ist mittlerweile eine akzeptierte Ansicht. Es fehlt jedoch noch an professionell umgesetzten Lernspielen, welche die komplette Bandbreite an Wissen vermitteln und so möglicherweise eine Schulausbildung ersetzen könnten.
Diese Arbeit handelt von der Entwicklung eines Lernspieles, dabei werden Konzepte der Gamification, Personas, Psychologie und Game Design generell verwendet. Für die Arbeit wurden gängige Lernmethoden, als auch klassische Videospiele analysiert. Das Spiel wird konzipiert, entwickelt und über die Entwicklung hinweg bereits kontinuierlich evaluiert. Der Spielspaß steht dabei im Vordergrund der Entwicklung.
Ergebnis der Arbeit ist ein Rennspiel für Android. Das Spiel unterstützt beliebige Lernmaterialien, eignet sich aber vor allem für schnell abrufbares Faktenwissen.

\vfill

\section*{Abstract}
That learning games work is now an accepted view. However, there is still a lack of professionally implemented learning games which can provide the complete range of knowledge and thus possibly replace a school education.
This work deals with the development of a learning game, in which concepts of gamification, personas, psychology and game design are generally used. For the work, common learning methods as well as classical video game were analyzed. The game is designed, developed and continuously evaluated over the development. The game is the focus of the development.
Result of the work is a racing game for Android. The game supports any learning materials, but is suitable for fast-retrievable factual knowledge.

\vfill\vfill\newpage
