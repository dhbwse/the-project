% !TEX root = dokumentation.tex
\section*{Abstract}
Schon 1980 fragten sich Forscher, was genau Spaß an Computerspielen macht und wie diese Eigenschaften verwendet werden können, um Lernen generell Spaßig zu gestalten.\footcite{learn-game-history} Mittlerweile ist dass Lernspiele funktionieren eine akzeptierte Ansicht. Es fehlt trotzdem noch an professionell umgesetzten Lernspielen welche die komplette Bandbreite an Wissen vermitteln und so möglicherweise eine Schulausbildung ersetzen könnten.
Diese Arbeit handelt von der Entwicklung eines Lernspieles, dabei werden Konzepte der Gamification, Personas, <TODO> und <TODO> verwendet.
Als Resultat \dots

Das Oberthema der vorliegenden Studienarbeit ist das spielerische Lernen. Die Arbeit zeigt an einem Beispiel, dass Lernen von Schulstoff auch mit Spielspaß vereinbar ist. Dazu wurden sowohl gängige Lernmethoden, als auch klassische Videospiel analysiert. Im Nachhinein wurde betrachtet, wie diese beiden Themenbereiche kombiniert werden können. Unter Verwendung von Gamification und psychologischen Grundlagen für Spiele wurde im ersten Teil der Arbeit ein Spiel für Android Endgeräte gestaltet, welches den Spielspaß in den Vordergrund stellt. Beim Spielen des Spiels lernen die Kinder etwas über einen ausgewählten Lernstandard, wie beispielsweise Mathemtik der zweiten Klasse. Der zweite Teil der Arbeit beschäftigt sich mit der Entwicklung und den technischen Hintergründen dieser Anwendung. 
Die vorliegende Studienarbeit ist einerseit interessant für Lehrer als Informationsquelle über spielerisches Lernen, kann aber auch von Entwicklern als Informationsquelle über Game Design und Spieleentwicklung herangezogen werden.

\vfill

\section*{Abstract}
Already in the 1980s...

\vfill\newpage
