% !TEX root = dokumentation.tex
\section{Evaluation}
    Im Rahmen der Evaluation sollen die in der Konzipierung festgesetzten Kriterien mit der tatsächlichen Implementierung verglichen werden. Dabei wird analysiert, ob die gesetzten Kriterien erfüllt sind. Somit ist eine Bewertung der Studienarbeit auf Basis der Aufgabenstellung und den definierten Kriterien möglich.
    Eventuelle Abweichungen zu den Kriterien werden im Rahmen der Evaluation erläutert und begründet. Zusätzlich soll betrachtet werden, welche Projektschritte nicht bzw. anders als geplant abgelaufen sind

\subsection{Analyse bezüglich der psychologischen Grundlagen}
Im folgenden Absatz soll die im Kapitel \ref{ssec:psycho-grundlagen} aufgestellte These diskutiert und mit dem fertigen Spiel abgeglichen werden. Die These lautet:
\enquote{Durch eine Kombination aller vier genannten Merkmale (Long-term Learning, Social Interaction, Spielspaß, Attention \& Challenge) entsteht für den Spieler ein persönliches Engagement. Durch die Lernkurve und den ansteigenden Schwierigkeitsgrad fühlt sich der Spieler herausgefordert. Der Spielspaß unterstützt diesen Effekt, da der Spieler das Spiel gerne spielt. Während der Spieler also auf die Herausforderungen des Spiels konzentriert ist und dabei Spaß empfindet, fällt der Lerneffekt und damit die Verbesserung der akademischen Fähigkeiten nicht auf.}
Bevor eine exakte Analyse dieser These in Verbindung mit der Anwendung möglich ist, muss vermerkt werden, dass der Mehrspielermodus aus zeitlichen Gründen nicht implementiert werden konnte. Eine genauere Betrachtung der Kriterien und der Spielidee befindet sich im Kapitel \ref{ssec:kriterien}. Somit kann der Aspekt der Social Interaction nicht untersucht werden.
In der finalen Version des Spiels ist eine Lernkurve gegeben. Der Spieler beginnt mit einem Hilfsagenten und wird über ein Einführungslevel an alle relevanten Steuerungselemente herangeführt. Dabei werden die einzelnen Steuerungselemnete getrennt voneinander erläutert, sodass der verwendete Funktionsumfang im Laufe des Hilfsagenten stetig ansteigt. Je weiter der Spieler im Spiel voranschreitet, erhöht sich der Schwierigkeitsgrad der Level. Somit bleibt auch für geübte Spieler stets eine Herausforderung in den höheren Levelpaketen verfügbar. Im Rahmen der durchgeführten Playtests \ref{sec:playtest} hat sich gezeigt, dass die Spieler Spaß am Testen der Anwendung haben und das Spiel gern spielen. Durch die häufige Wiederholung von Aufgabentypen wird eine optimale Nutzung des \enquote{Long-term Learnings} erreicht. Der Spieler prägt sich das Vorgehen zum lösen der Aufgaben ein und festigt das Wissen des Stoffes. Durch eine anfänglich sehr lange Anzeigezeit der Aufgaben erhält der Spieler die Möglichkeit, sich an das Lösen von Aufgaben während des Rennens zu gewöhnen. Mit weiterem Spielfortschritt verkürzt sich die Zeit, die zum Lösen der Aufgaben zur Verfügung steht. Das bedeutet einen weiteren Anstieg des Schwierigkeitsgrads und führt dazu, dass der Spieler stetig gefordert wird und keine Langeweile eintritt.
Analyse der Playtests
Die Playtests für die Studienarbeit werden mit dem Fokus auf Qualität durchgeführt. Quantitative Untersuchungen sind für Playtests einer Android-Anwendung ungeeignet, da die Anwendung vielen Personen zur Verfügung gestellt werden müsste. Zudem müssten die Testpersonen beobachtet werden, um ein angemessenes Qualitätslevel zu erreichen. Deshalb werden zur Durchführung der einzelnen Playtest-Schritte kleine Gruppen von zwei bis drei Personen ausgewählt. Ein wichtiges Kriterium für die Auswahl der Testpersonen ist, dass diese das Spiel noch nicht kennen bzw. noch nicht als Testperson verwendet wurden.
Den Testpersonen werden in jedem Testdurchlauf fest definierte Fragen gestellt.
\begin{enumerate}
    \item{Wie viel Spaß hattest du beim Spielen?}
    \item{Wie intuitiv empfindest du die Steuerung?}
    \item{Wie empfindest du die gestellten Aufgaben?}
    \item{Hast du Verbesserungsvorschläge?}
\end{enumerate}
Die Fragen sind sehr offen formuliert, um einen Dialog mit den Testpersonen zu ermöglichen. Zusätzlich werden die Testpersonen wie bereits erwähnt beim Spielen beobachtet. So kann beispielsweise betrachtet werden, wieviel Zeit nötig ist um die Steuerung zu verstehen. Dabei wird auf jegliche Erklärungen verzichtet. Versteht der Spieler die Steuerung sehr schnell, so kann von hoher Intuitivität ausgegangen werden. Über die Frage nach Verbesserungsvorschlägen wird konkret abgefragt, welche Anforderungen einzelne Spieler ann ein Rennspiel dieser Art haben. Viele Änderungen am ursprünglichen Konzept basieren auf Vorschlägen von Testspielern.
Auch wenn die Wahl dieses Vorgehens keinen wissenschaftlichen Standards entspricht, haben diese Playtests dennoch stark zur Verbesserung der Qualität der Anwendung beigetragen. Da die Zeit für ausführliche Umfragen und große Breitentests des Spiels nicht vorhanden ist, wird das Vorgehen der durchgeführten Playtests als richtungsweisend und hilfreich empfunden. Somit ist die Wahl des Playtestings im Grunde genommen die richtige Entscheidung und hat die Entwicklung des Spiels maßgeblich beeinflusst.

\subsection{Analyse der Kriterien zur Umsetzung}
Bei der Analyse der Kriterien der Spielidee soll deren Erfüllung in der finalen Anwendung ausgewertet werden.
Zum Besseren Verständnis der Erläuterungen werden die entsprechenden Kriterien erneut erwähnt.
\begin{itemize}
    \item{Das Spiel soll levelbasiert sein. - Dieses Kriterium ist erfüllt. Im Spiel sind voneinander unabhängige Levelpakete vorhanden. Lediglich innerhalb der Levelpakete existieren Abhängigkeiten. Um das Pokalrennen freizuschalten müssen die Qualifikationsrennen des Levelpakets abgeschlossen werden.}
    \item{Der Umfang ist auf zwei bis drei Stunden Spielspaß ausgelegt. - Dieses Kriterium wird in der aktuellen Version des Spiels nicht erfüllt. Durch auftretende Zeitprobleme ist eine Entwicklung von ausreichend Levelpaketen zur Erfüllung der Zeitrichtline nicht möglich. Der Fokus der Entwicklung liegt auf der Funktionsfähigkeit des Spiels.}
    \item{Durch modulare Gestaltung kann das Spiel erweitert werden. - Die Erweiterbarkeit des Spiels ist vollständig erfüllt. Zum Hinzufügen weiterer Levelpakete ist ausschließlich die Gestaltung der einzelnen Level notwendig. Die Grundfunktionalität des Spiels ist gegeben und ist in neuen Leveln ebenso funktionsfähig.}
    \item{Der Spielspaß steht im Vordergrund. - Wie im Rahmen der Playtests bereits erwähnt, kann bestätigt werden, dass die Testpersonen Spaß am Spiel haben. Für eine genauere Auswertung des Spaßfaktors ist jedoch eine ausführliche Umfrage nötig. }
    \item{Der Spieler lernt unbewusst nebenbei und hat Spaß am Spiel. - Die befragten Testpersonen empfinden die Lerninhalte nicht als \enquote{Hauptbestandteil} des Spiels.}
    \item{Das Spiel soll \enquote{für zwischendurch} geeignet sein. - Durch eine Levellänge von durchschnittlich zwei Minuten kann das Spiel jederzeit gestartet und ein Level schnell absolviert werden. Der Spieler wird nicht genötigt, viel Zeit in den einzelnen Leveln zu verbringen.}
    \item{Die Spielidee soll originell sein und kein anderes Spiel kopieren. - Eine ausführliche Recherche bezüglich ähnlichen Spielen ergibt kein Ergebnis. Unter dem Namen \enquote{Schreibspiel Autorennen} existiert ein Spiel, mit welchem die Spieler schnelles Schreiben mit Tastaturen erlernen sollen. Die Unterschiede dieses Spiels zur entwickelten Anwendung sind jedoch so weitreichend, dass eine Ähnlichkeit zur entwickelten Anwendung ausgeschlossen wird.}
    \item{Im Gegensatz zu Spielen von \enquote{arcademics.com} soll das Spiel offline verfügbar sein. - Abgesehen von einem initalen Download benötigt das entwickelte Spiel keine Internetverbindung. Mit einer zukünftigen Entwicklung eines Mehrspielermodus wird zwar eine Internetverbindung nötig, ist jedoch nicht für den Zugriff auf den Einzelspielermodus relevant.}
    \item{Das Spiel distanziert sich von gängigen Lernmethoden, vermittelt aber dennoch ausgewählte Lernstandards. - Über die Möglichkeit, Fragen und Antworten im JSON Format in die Anwendung zu übergeben, kann der Lernstandard frei definiert werden. In der derzeiten Version des Spiels sind Fragen aus dem Lernstandard Mathematik erste und zweite Klasse integriert. Diese können von Lehrern, welche das Spiel für ihre Schulklassen nutzen möchten, entsprechend angepasst werden. Die Wahl der Lernmethode über \enquote{Hineinfahren in die korrekte Lösung} ist keine gängige Lernmethode, somit ist das Kriterium erfüllt.}
    \item{Anpassung an die Zielgruppe aus Kapitel \ref{ssec:personadef} - Durch die Verwendung von kindgerechten Grafikelementen und einer vergleichsweise einfachen Steuerung kann das Spiel von Kindern im Grundschulalter problemlos gespielt werden. Auch Personen höheren Alters sind in der Lage das Spiel zu spielen, auch wenn die Eingewöhungszeit deutlich höher ausfällt, als bei Kindern. Dies haben die Playtests ergeben.}
\end{itemize}
Die obrige Auflistung zeigt sehr deutlich, dass die gesetzten Anforderungen zu einem sehr großen Teil erfüllt sind. Lediglich der gesetzte Umfang kann nicht erfüllt werden, dies ist jedoch über das spätere Hinzufügen weiterer Levelpakete noch erreichbar.

\subsection{Analyse Lernziele}
Durch die Wahl der Weggabelungen für die Realisierung der Aufgaben, sind immer zwei Antwortmöglichkeiten gegeben. Da eine dieser Antworten korrekt und die zweite Antwort falsch ist, kann Faktenwissen optimal vermittelt werden. Die gestellten Fragen aus dem Lernstandard Mathematik der ersten und zweiten Klasse besteht aus Addition und Subtraktion. Diese Aufgaben besitzen exakt ein korrektes Rechenergebnis und können so auf die möglichen Wege der Weggabelung abgebildet werden. Durch die Wiederholung der Aufgaben verteilt über viele verschiedene Level ist ein auswendiglernen der Ergebnisse nicht möglich. Der SPieler muss die Aufgaben korrekt lösen um im Spiel voran zu kommen. Da Spieler ein Spiel in der Regel erfolgreich abschließen möchten, werden die Spieler zum lernen motiviert.

\subsection{Analyse Funktionale und nicht funktionale Anforderungen}
Im Bereich der funktionalen und nicht funktionalen Anforderungen sind einige Features vorgestellt worden, welche in der finalen Anwendung nicht oder anders umgesetzt wurden. Die Anforderungen, welche vollständig erfüllt sind, werden an dieser Stelle nicht betrachtet. An dieser Stelle soll erläutert werden, mit welcher Begründung bestimmte Anforderungen nicht umgesetzt sind.
\begin{itemize}
    \item{Hilfsagent beim Spielstart: Von der ursprünglichen Planung des Hilfsagenten, dem Spieler im Einführungslevel über wortlose Bilder die Steuerung zu erklären, wird abgesehen. An Stelle dieses aufwändigen Tutorials wird ein Video verwendet, in dem gezeigt wird, wie ein Spieler in eines der Level hineinfährt. Dabei sind alle wichtigen Steuerungselemente klar erkennbar. Der geplante Hilfsagent wird aus zeitgründen nicht umgesetzt, da die getätigte Zeitschätzung für die Implementierung des Hilfsagenten zu kurz ausgefallen ist.}
    \item{Optionsmenü als eigener Level: Das Optionsmenü ist nicht, wie ursprünglich geplant als eigener Level implementiert. Da die Möglichkeit, im Menü einen zusätzlichen Level zu erreichen, in jedem Levelpaket verfügbar sein müsste, würde dies die fahrbaren Menüs überladen. Das Optionsmenü ist als zusätzlich einblendbarer Bildschirm innerhalb der Level verfügbar.}
    \item{Level wählen: Im Entwurf des Spiels soll dem Spieler ein Auswahlbildschirm mit den Funktionen \enquote{Als Zeitrennen starten.} und \enquote{Als Rennen gegen den Computergegner starten.} angeboten werden. Da die Entwicklung einer KI auf Grund der Komplexität gestrichen wurd, entfällt die Notwendigkeit für diesen Auswahlbildschirm.}
    \item{Onlinespiel starten: Die Funktionen eines Onlinespiels sind zwar im Entwurf des Spiels vorgesehen, jedoch wird aus mehreren Gründen darauf verzichtet, das Onlinespiel zu ermöglichen:}
    \begin{itemize}
        \item{Bei einem Mehrspielermodus ist eine dauerhafte Bereitstellung eines Servers notwendig. Dieser Server muss gepflegt und gewartet werden. Dieser auf lange Sicht anfallende Zusatzaufwand wird von beiden Projektteilnehmern nicht gewollt.}
        \item{Zur Vermeidung von anzüglicher Sprache und Beleidigung müsste die Kommunikation zwischen den Spielern überwacht werden.}
        \item{Durch die bereits erwähnte fehlerhafte Zeiteinschätzung wird der Fokus zunächst auf die essenziellen Anteile der Anwendung gelegt.}
    \end{itemize}
    \item{\label{zusätzlicheAufgaben}Zusätzliche Aufgaben lösen: Das zur Verfügung stellen von zusätzlichen Aufgaben im Sinne eines \enquote{Learn-to-Win} Systems bringt das Spiel potentiell zu sehr in die Richtung klassischer Lernspiele. Der Spieler soll die Münzen zum erwerben von Belohnungen erhalten, indem er die Level erneut spielt. Da in den Leveln ebenfalls Aufgaben implementiert sind, bleibt der Lernerfolg nicht aus.}
    \item{Integration in die Lernplattform: Durch das Fehlen wichtiger Schnittstellen auf Seiten der Lernplattform konnte eine Integration in Selbige nicht stattfinden.}
    \item{Aufgabenverwaltung durch Aufgabensteller: Das Spiel wird in vollem Umfang nur für einen Lernstandard veröffentlicht. Durch eine Schnittstelle zur Lernplattform könnte der Lernstandard variabel definiert und bei Anwendungsstart übergeben werden, da diese Schnittstelle jedoch nicht existiert, wird auf die Aufgabenverwaltung verzichtet.}
    \item{Minimale Bildwiederholrate von $24$ Bildern pro Sekunde: Diese Anforderung kann leider nicht für alle Endgeräte erreicht werden. Wenn das Endgerät die technischen Gegebenheiten eines \emph{HTC ONE M8} überschreitet, gilt diese Anforderung als erfüllt.}
    \item{Maximale Speicheranforderung von unter $300$ MB: Das Spiel in der aktuellen Version erfüllt diese Bedingung. Das installierte Spiel nimmt derzeit eine Speicherplatzgröße von $280$ MB ein. Ein Erweitern der Anwendung um zusätzliche Level kann jedoch leicht zu einer Überschreitung dieser $300$ MB Grenze führen.}
\end{itemize}
Alle Anforderungen, welche in diesem Kapitel nicht erwähnt werden, können als grundsätzlich erfüllt angenommen werden.

\subsection{Analyse Spielmechanik und sonstige Eigenschaften}
Bei der Analyse der Spielmechanik und sonstigen Eigenschaften werden besonders die Abweichungen von der ursprünglich geplanten Mechanik betrachtet. Aspekte des Spiels, welche exakt so implementiert werden konnten wie sie geplant waren, werden an dieser Stelle nicht erneut erwähnt.
\begin{itemize}
    \item{Zusatzaufgaben im Multiplayer: Durch die fehlende Implementierung eines Multiplayers, findet dieser Abschnitt der Spielmechanik keinen Weg in die Anwendung. Sollte die Anwendung weiter gepflegt oder weiterentwickelt werden, kann ein Multiplayer in dieser Form umgesetzt werden.}
    \item{Verdienen der Ingame-Währung: Im Laufe der Entwicklung ist aufgefallen, dass eine Implementierung der täglich beschränkten Zusatzaufgaben (wie bereits in \ref{zusätzlicheAufgaben} erwähnt) das Spiel zu sehr in die Richtung klassischer Lernspiele bringt. Aus diesem Grund kann keine Ingame-Währung auf diesem Weg erworben werden.}
    \item{Kaufbare Gegenstände für Ingame-Währung: Im Bereich der optischen Upgrades, welche für Ingame-Währung gekauft werden können, ergeben sich Änderungen. Bei einer Implementierung verschiedener Gegenstände (wie beispielsweise einem Läufer oder einem Ball), müssen die physikalischen Eigenschaften dieser Objekte angepasst werden. Da im Programmcode grundsätzlich von den Eigenschaften eine Fahrzeugs ausgegangen wird, ist dies nicht ohne Weiteres möglich. Um zusätzlichen Aufwand für optische Upgrades betreiben zu müssen und um sicher zu stellen, dass alle Fahrzeuge/Gegenstände ein identisches Verhalten zeigen, wird auf diese Gegenstände verzichtet. Stattdessen werden dem Nutzer weitere Fahrzeugmodelle zur Verfügung gestellt.}
\end{itemize}

\subsection{Verwendung eines Vorgangs-/ Projektverfolgungssystems}
Auch wenn im Rahmen der Aufgabendefinition die Anforderung an ein Vorgangs- bzw. Projektverfolgungssystem gestellt wurde, haben sich die beiden Projektteilnehmer geeinigt, kein solches System zu verwenden.
Im Rahmen der Vorlesung \enquote{Angewandtes Projektmanagment} wurden Methoden zur Erstellung eines Projektmanagement-Dokuments vorgestellt. Auf Basis dieser Methoden und Regeln wurde ein solches Dokument erzeugt. Das Dokument befindet sich im Anhang <TODO Anhang verlinken> und beschreibt das Vorgehen der Arbeit und des Projektmanagements. In diesem Dokument enthalten ist eine exakte Beschreibung aller relevanten Themen des Projektmanagements.

\subsection{Evaluationsfazit}
Im Laufe der Entwicklung wurden viele Änderungen am ursprünglichen Plan vorgenommen. Diese Anpassungen sind einerseits darauf zurückzuführen, dass geplante Elemente nicht auf diese Weise umgesetzt werden konnten. Andererseits basieren viele getätigte Anpassungen auf den ursprünglichen Fehlern in der Zeitplanung. Für zukünftige Projekte ist eine bessere, realistischere Zeitplanung besonders wichtig. So kann beispielsweise ein Verschieben der Abgabefrist zukünftig vermieden werden.
Keine dieser Änderungen nimmt gravierenden Einfluss auf die grundsätzliche Projektidee. Die zu Beginn der Arbeit definierten Kriterien konnten fast vollständig erfüllt werden. Die in diesem Kapitel genannten Anpassungen beeinflussen den geplanten Zweck des Spiels nicht und beschreiben lediglich technische Anpassungen. Somit wird die Aufgabe grundsätzlich als vollständig erfüllt betrachtet.


