% !TEX root = dokumentation.tex
\section{Konzipierung}
Das folgende Kapitel beschäftigt sich mit der Konzipierung des Spiels. Dabei werden zunächst die erarbeiteten und vorgegebenen Kriterien zusammengefasst, mehrere Spielideen anhand dieser Kriterien überprüft und anschließend eine finale Spielidee entwickelt. Im weiteren Verlauf des Kapitels werden Functional und Non-Functional Requirements erörtert und in Use-Cases unterteilt. Anhand dieser Requirements wird dann im Kapitel \ref{sec:impl} Implementierung das Spiel entwickelt. Weiterhin werden in diesem Kapitel die Spielregeln verfasst und weitere Elemente, wie die Steuerung oder Computergegner betrachtet. Abschließend soll erarbeitet werden, wie die im Grundlagenteil erfassten Lernziele in Verbindung mit dem Spiel umgesetzt werden können.
\subsection{Kriterien für die Spielidee}\label{ssec:kriterien}
	Auf Basis der in der Aufgabenstellung genannten Kriterien und den Grundlagen wird ein Kriterienkatalog entwickelt. Dieser wird mit zusätzlichen eigenen Kriterien erweitert.
	\begin{enumerate}
		\item{Das Spiel soll levelbasiert sein, d.h. dass voneinander unabhängige Spielabschnitte vorhanden sind.}
		\item{Der Umfang des Spiels ist auf zwei bis drei Stunden Spielspaß ausgelegt. Durch modulare Gestaltung (Levelbasiert) kann das Spiel im Nachhinein erweitert werden.}
		\item{Das Spiel ist ein \enquote{educational game}, der Spielspaß steht im Vordergrund. Die Spieler sollen unbewusst nebenbei lernen.}
		\item{Das Lernen selbst soll Spaß machen. Das Spiel soll die Spieler nicht zum Lernen nötigen, sondern das Lernen als festen Bestandteil in das Geschehen integrieren.}
		\item{Das Spiel soll \enquote{casual} sein, d.h.\@ ein Spiel \enquote{für zwischendurch} sein.}
		\item{Der vollständige Programmieraufwand für das Spiel ist auf 3 Monate Arbeit ausgelegt. Somit bleiben weitere volle drei Monate für das Verfassen der Arbeit.}
		\item{Ziel ist, eine originelle Spielidee zu entwickeln, sodass das Spiel kein Klon eines bereits existierenden Spiels ist.}
		\item{Als Beispiele zur Orientierung dient die Webseite \enquote{arcademics.com}, welche Spiele der gewünschten Art beinhaltet, jedoch eine Internetverbindung benötigt. }
		\item{Im Gegensatz zu \enquote{arcademics.com} soll das Spiel abgesehen vom initialen Download offline verfügbar sein.}
		\item{Das Spiel stellt keine gängigen Lernmethoden zur Verfügung. Dennoch sollen ausgewählte Lernstandards (z.B. Mathematik der zweiten Klasse) vermittelt werden.}
		\item{Die Zielgruppe für das Spiel ist durch die Personas aus Kapitel \ref{ssec:personadef} gegeben. Nach Möglichkeit soll das Spiel und die zugehörige Lernplattform auch Flüchtlingskindern zur Verfügung gestellt werden.}
	\end{enumerate}

	Die Entstehung dieser Kriterien ist teilweise auf die Ergebnisse von Brainstorming zurückzuführen. Durch das Brainstorming können die Erwartungen der Projektteilnehmer an das Spiel spezifiziert werden. Somit ist eine genaue Trennung der Kriterien aus dem Brainstorming und der Aufgabenstellung nur schwer möglich. Die Zusammenfassung aller Kriterien dient lediglich der Übersichtlichkeit.

\subsection{Lernziele Definition}\label{ssec:lern-def}
	Als Lernziel wird die Vermittlung von Faktenwissen gewählt. Das bedeutet, dass zu jeder gestellten Frage eine eindeutige Antwort existiert. Mögliche Themengebiete sind Mathematik, Naturwissenschaften, Sprachwissen oder auch Geschichte. So muss für das Spiel lediglich eine Liste von Fragen mit korrekten und möglichen falschen Antworten gepflegt werden.
	Die Lernstrategie soll durch Wiederholung von Aufgaben und spielerischem Anregen zum Lernen umgesetzt werden.

\subsection{Brainstorming \& Game Design}\label{ssec:idee}
	In mehreren Gesprächen werden Ideen für das zu entwickelnde Spiel gesammelt. Dieses Kapitel hat zum Zweck, den Ablauf der Ideenfindung in eine Reihenfolge zu bringen und diese zu präsentieren.
	Die Ideen werden anhand der Kriterien aus Kapitel \ref{ssec:kriterien}, aber auch nicht eindeutigen, subjektiven Kriterien beurteilt. Die subjektiven Kriterien sind beispielsweise das \enquote{Eigeninteresse an der Idee} und die \enquote{Lust die Idee umzusetzen}.
	\begin{enumerate}
		\item{Chemie / Physik Simulation:}
		\begin{itemize}
			\item{Die Simulation findet in einer Laborumgebung statt. Dabei steuert der Spieler den Laborarbeiter (bzw. aus der Sicht des Laborarbeiters).}
			\item{Der Spieler führt einfache Experimente aus den Bereichen Chemie und Physik durch.}
			\item{Fehler, welche dem Spieler in den Experimenten passieren, führen zum Scheitern des Versuchs.}
			\item{Der Spieler hat die Möglichkeit, die Hände des Laborarbeiters zu steuern.}
			\item{Alternativ kann der Spieler über den Touchscreen mit den Utensilien im Labor interagieren.}
			\item{Die Idee wird verworfen, da eine realistische Steuerung von Experimenten aus Chemie und Physik nur schwer über einen Touchscreen umzusetzen ist. Weiterhin ist die für diese Art von Spiel benötigte Physiksimulation ungeeignet für die Zielplattform. Außerdem stellt diese Idee nicht den gewünschten Lernerfolg für die gewünschte Zielgruppe zur Verfügung. }
		\end{itemize}
		\item{Rätselspiel:}
		\begin{itemize}
			\item{Der Spieler soll eine Art Rätsel lösen, welches automatisch in unterschiedlichen Schwierigkeitsgraden generiert werden kann.}
			\item{Die Idee wird aufgrund eines fehlenden Rätselkonzepts, fehlender weiterer Ideen und großer Einschränkung der Zielgruppe verworfen.}
		\end{itemize}
		\item{Simulation von Motoren und Getrieben:}
		\begin{itemize}
			\item{Vor Spielbeginn bekommt der Spieler eine Strecke angezeigt. Diese Strecke ist in 2D und zeichnet sich durch Höhenunterschiede aus.}
			\item{Der Spieler wählt auf Basis der Strecke aus einer Sammlung verschiedener Motoren den Besten aus. Beispielsweise ist ein 2-Takt Motor für steile Berge besser geeignet als ein 4-Takt Motor.}
			\item{Der Spieler bekommt mehrere Treibstoffe (Benzin, Diesel, \dots) zur Auswahl und kann entsprechend der Strecke und des Motors wählen.}
			\item{Während der Fahrt hat der Spieler die Möglichkeit, über Schaltflächen den ausgewählten Gang des Getriebes zu wechseln.}
			\item{Bei der Auswahl von Treibstoffen, Motoren und Gangschaltungen werden dem Spieler Informationstexte angezeigt und vorgelesen.}
			\item{Um das Level bzw. das Spiel abzuschließen, muss die Strecke in der vorgegebenen Zeit absolviert werden.}
			\item{Die Idee wird nach dem Kickoff-Meeting vollständig verworfen, da die Umsetzung der Lernziele nicht in gewünschter Form möglich ist.}
		\end{itemize}
		\item{Rennfahrspiel Version 1:}
		\begin{itemize}
			\item{Der Spieler nimmt an einer wilden Renntour durch die Stadt teil.}
			\item{Der Spieler wird von der Polizei verfolgt. Sollte er gestellt werden, werden verschiedene Animationen des Scheiterns gezeigt.}
			\item{Sollte der Spieler scheitern, kann eine Aufgabe absolviert werden, um einen weiteren Versuch zu erhalten.}
			\item{Diese Aufgaben / Minigames sind beispielsweise:}
			\begin{itemize}
				\item{Ampel: Der Spieler muss rechtzeitig reagieren, um an einer roten Ampel anzuhalten.}
				\item{Polizei: Der Spieler muss die Anweisungen der Polizei rechtzeitig befolgen.}
				\item{Sprungschanze: Der Spieler benötigt die richtige Geschwindigkeit, um den Sprung perfekt zu absolvieren.}
			\end{itemize}
			\item{Die Idee wird nach dem Kickoff-Meeting vollständig verworfen, da die Umsetzung der Lernziele nicht in gewünschter Form möglich ist.}
		\end{itemize}
	\end{enumerate}
	Aus diesen Ideen und in Verbindung mit den Informationen aus dem Kickoff-Meeting konnte die finale Spielidee entwickelt werden. Dabei wird die Idee eines Rennspiels beibehalten und so angepasst, dass eine Umsetzung der Lernziele gut umsetzbar ist.
	\begin{enumerate}
		\item{Das Rennspiel wird dem Spieler aus der \enquote{Top-Down}-Ansicht präsentiert. Dabei ist die Kamera leicht schräg auf die Rennstrecke gerichtet, sodass die Modelle der Fahrzeuge dreidimensional sind.}
		\item{Als Steuerung nutzt der Spieler ein Steuerkreuz, sowie einen Button für Gas und einen Button für Bremse. Somit muss der Spieler sowohl lenken, als auch seine Geschwindigkeit steuern.}
		\item{Bei der Auswahl der spielbaren Objekte ist kein Fokus auf Realismus gesetzt. Beispielhafte spielbare Dinge sind:}
		\begin{itemize}
			\item{Ein rollender Ball und ein rollender Würfel (welcher logischerweise ungleichmäßige Bewegungen vollführt).}
			\item{Ein rennender Mensch oder ein rennendes Tier (welche die gleiche Geschwindigkeit wie Autos erreichen).}
			\item{Ein Auto oder ein Flugzeug, welches sich in Bodennähe aufhält und auf die Gegebenheiten der Strecke ebenso reagieren muss.}
		\end{itemize}
		\item{Für die verschiedenen Spielmodi stehen folgende Ideen zur Auswahl:}
		\begin{itemize}
			\item{Im Online Mehrspieler-Modus haben die Spieler die Möglichkeit, gegen bis zu drei weitere Spieler anzutreten. Gewinner ist dabei, wer zuerst das Ziel erreicht.}
			\item{Die Level für den Spielfortschritt sind als Zeitrennen umgesetzt. Dabei muss der Spieler das Ziel in der vorgegebenen Zeit erreichen. Am Spielende überschüssige Zeit kann unter Umständen in zusätzliche Belohnungen verwandelt werden.}
			\item{Die Level für den Spielfortschritt werden gegen computergesteuerte Rennautos gefahren. Um den Level zu beenden, muss der Spieler als erster das Ziel erreichen.}
		\end{itemize}
		\item{Die Strecken werden in Level-Paketen ausgeliefert. Diese Level-Pakete müssen von dem Spieler durch Abschluss des vorherigen Level-Pakets freigeschaltet werden.}
		\begin{itemize}
			\item{Ein Level-Paket besteht aus zwei normalen Strecken und einem Pokalrennen.}
			\item{Um an einer finalen Strecke teilnehmen zu können, muss der Spieler eine Teilnahmegebühr bezahlen. Verliert der Spieler dieses Rennen, wird die Gebühr erneut fällig.}
			\item{Mit dem Abschluss der zwei normalen Strecken wird das Pokalrennen verfügbar. Mit dem Abschluss des Pokalrennens wird das nächste Level-Paket freigeschaltet.}
			\item{Durch dieses Konzept ist Erweiterbarkeit gewährleistet.}
		\end{itemize}
		\item{Auf den Strecken werden verschiedene Abzweigungen möglich sein, bei welchen der Spieler die Auswahl zwischen zwei Wegen hat.}
		\begin{itemize}
			\item{Die Auswahl des \enquote{richtigen} Wegs wird über Aufgaben realisiert. Dabei wird dem Spieler rechtzeitig eine Aufgabe auf dem Bildschirm gezeigt.}
			\item{Beide möglichen Wege werden mit einer Lösung versehen. Eine dieser Lösungen ist die Lösung der vorher gezeigten Aufgabe. Dieser Weg gilt als der Richtige.}
			\item{Wählt der Spieler den falschen Weg, erhält der Spieler einen Nachteil. Im Fall des Zeitrennens kann dies z.B.\@ der Verlust von einigen Sekunden sein.}
		\end{itemize}
		\item{Im Mehrspieler-Modus müssen die Spieler Aufgaben absolvieren, um das Rennen beginnen zu können.}
			\begin{itemize}
				\item{Sobald die Aufgabe gelöst ist, kann ein Spieler losfahren. Somit können sich Spieler über schnelles Absolvieren der Aufgaben einen Vorteil im Rennen erspielen.}
				\item{Können die Aufgaben nicht gelöst werden, darf der Spieler nach einer Zeitverzögerung dennoch losfahren. Spieler, welche die Aufgaben gelöst haben, haben im Rennen dadurch einen Vorteil.}
				\item{Falscheingaben resultieren in einer neuen Aufgabe, um schummeln zu verhindern.}
			\end{itemize}
		\item{Der Spieler erhält die Möglichkeit, zusätzliche Aufgaben auf freiwilliger Basis zu lösen. Als Belohnung wird eine Ingame-Währung eingeführt, die für folgende Aktionen genutzt werden kann:}
		\begin{itemize}
			\item{Für ein Rennen gegen Computergegner kann sich der Spieler eine bessere Startposition erkaufen.}
			\item{Für ein Zeitrennen kann sich der Spieler einen Zeitbonus erkaufen.}
			\item{Der Spieler kann die Teilnahmegebühr für die Pokalrennen sowohl über die Ingame-Währung als auch über das Absolvieren von Aufgaben bezahlen.}
			\item{Der Spieler kann weitere Fahrzeuge / spielbare Objekte für die Ingame-Währung kaufen. Alternativ ist über diese Währung auch ein Farbwechsel der bisher gekauften Objekte oder ein Wechsel des Spielersymbols (für den Mehrspieler-Modus) denkbar.}
		\end{itemize}
	\end{enumerate}

\subsection{Funktionale und nicht funktionale Anforderungen}\label{ssec:requirements}
	In diesem Unterkapitel werden die Anforderungen an das Spiel dokumentiert. Begonnen wird mit einer Übersicht der Funktionalitäten, dann folgt eine detaillierte Beschreibung der Anforderungen der einzelnen Funktionen. Abschließend werden generelle nicht funktionale Anforderungen definiert.
	Das Spiel wird hauptsächlich so gesteuert wie auch die Rennen gefahren werden: mit einem Auto. Das heißt, dass die Menüs auch mit einem Auto angesteuert werden.
	Für das Spiel werden zwei Aktoren benötigt:
	\begin{description}
		\item[Spieler]{ Die Person, welche das Spiel auf dem Handy installiert hat und spielt. }
		\item[Aufgabensteller]{ Die Person, welche neue Aufgaben für das Spiel zusammenstellt, bestehende editiert oder auch Aufgaben entfernt. Die Person benötigt eine Methode die Aufgaben zu aktualisieren. }
	\end{description}

	Liste aller Funktionen und eine Beschreibung für einen Überblick der Funktionalität, für weitere Informationen siehe Kapitel \ref{ssec:ucd} und \ref{sssec:spielmechanik}:
	\begin{description}
		\item[Spielstart Hilfsagent]{ Beim ersten Starten wird dem Nutzer geholfen, sich in das Spiel einzufinden. Dies wird über wortlose Bilder erreicht, welche verschwinden, sobald der Nutzer die Steuerung ausprobiert hat und ein Rennen gefahren ist. Der Hilfsagent kann auch abgebrochen werden. }
		\item[Level-Pakete wechseln]{
			Der Spieler kann die Level-Pakete rotieren, jedes besitzt zwei Level als Qualifikationsrennen (normale Rennkarten) und einen Level, für welchen eine Teilnahmegebühr anfällt (Pokalrennen).
			Jedes Level-Pack ist in einem eigenen Stil thematisiert, wobei die Karten für die Pokalrennen außergewöhnlicher sein sollen.
		}
		\item[Optionen ändern]{ Über einen speziell als Optionen gekennzeichneten \enquote{Level} gelangt der Nutzer auf einen gesonderten Bildschirm wo folgende Funktionen zur Verfügung stehen:
			\begin{description}
				\item[Lautstärke der Musik einstellen]{ Der Nutzer kann mittels Schieberegler die Lautstärke der Hintergrundmusik, im Spiel wie in den Menüs, einstellen. Die Lautstärke wird in Echtzeit verändert. }
				\item[Lautstärke der Soundeffekte einstellen]{ Der Nutzer kann mittels Schieberegler die Lautstärke von Soundeffekten anpassen. Wird der Schieberegler losgelassen, wird ein Effekt vorgespielt, um dem Nutzer Feedback zu geben.}
				\item[Hilfsagent erneut starten]{ Der Hilfsagent wird neu gestartet und kann erneut gespielt werden.}
			\end{description}
		}
		\item[Level wählen]{ Der Spieler fährt mit dem Auto auf einen Level, darauf erscheint ein Auswahlbildschirm für folgende zwei Optionen. Falls der Level ein Pokalrennen ist, muss der Spieler eine Teilnahmegebühr zahlen.
			\begin{description}
				\item[als Rennen gegen die Zeit starten]{ In diesem Spielmodus ist eine Zeit vorgegeben, innerhalb welcher der Spieler den Level bestehen soll. Ebenso sind zwei kleinere Zeiten vorgegeben, welche den Spieler herausfordern, sich mehr anzustrengen. }
				\item[als Rennen gegen Computergegner starten (Optional)]{ In diesem Spielmodus fährt der Spieler gegen computergesteuerte Autos. Die Auswertung funktioniert über das Übliche Rang-Verfahren: die besten drei Spieler werden hervorgehoben und gekürt, nur der Erste gewinnt.}
			\end{description}
			Gestartete Spiele können jederzeit abgebrochen, pausiert und neugestartet werden.
		}
		\item[Onlinespiel starten (Optional)]{ Ebenso wie die Optionen ist dies ein speziell gekennzeichneter \enquote{Level}, in welchem Rennen gegen zufällig ausgewählte Spieler gefahren werden können. }
		\item[Extra Aufgaben Lösen]{ Ein Icon am Rand des Bildschirms, welches ermöglicht, Münzen durch Lösen von Aufgaben zu sammeln. }
		\item[Einkaufen]{ Wie bereits im Rahmen der Ingame-Währung erwähnt, kann der Spieler beispielsweise Fahrzeuge und Farben erwerben.
			\begin{description}
				\item[Gegenstand Kaufen]{ Der Spieler erhält im Austausch für Ingame-Währung den Artikel, dieser wird dann auch ausgewählt. Im Ingame-Shop existieren Gegenstände, welche verbraucht werden, zum Beispiel Zeitboni und Gegenstände, die dem Spieler verfügbar bleiben, wie Fahrzeuge. }
				\item[Gegenstand Auswählen]{ Ein bereits bezahlter Gegenstand, welcher nicht verbraucht werden kann, wird aktiviert. So kann beispielsweise zwischen zwei bereits gekauften Fahrzeugen gewechselt werden.}
			\end{description}
		}
		\item[Spiel Beenden]{ Beendet das Spiel. }
		\item[Aufgabenverwaltung durch Aufgabensteller]{
			Der Aufgabensteller kann folgende Aktionen machen:
			\begin{itemize}
				\item{ Der Aufgabensteller kann Aufgaben ansehen und verändern. }
				\item{ Der Aufgabensteller kann Aufgaben entfernen. }
				\item{ Der Aufgabensteller kann Aufgaben hinzufügen. }
				\item{ Der Aufgabensteller kann ein Spiel mit einem geändertem Aufgabensatz erzeugen.	}
			\end{itemize}
		}
	\end{description}
	Alle Funktionalitäten lassen sich in einem Use-Case Diagramm abbilden, welches die Autoren bei der Entwicklung durch vereinfachte und übersichtliche Koordination und Aufgabenaufteilung unterstützt.
	\label{ssec:ucd}\figur[\textwidth]{usecases.pdf}{Use-Case Diagramm}

	\subsubsection{Einschränkungen}
		Zeitrennen und Rennen gegen Computergegner sollen auch ohne Internetverbindung verfügbar sein.
		Die Menüstruktur soll so flach wie möglich gehalten werden, keine verschachtelten Menübäume.
	\subsubsection{Nicht funktionale Anforderungen}
		\begin{itemize}
			\item{ Minimale Bildwiederholrate von 24 Bilder pro Sekunde. }
			\item{ Maximale Speicheranforderung von unter 300MB. }
			\item{ Spielmusik soll nicht stocken. }
			\item{ Das Spiel soll in unter einer Sekunde auf Nutzereingaben reagieren. }
		\end{itemize}

\subsection{Eigenschaften des Spiels}\label{sssec:spielmechanik}
	Im folgenden Kapitel sollen die Grundlagen des Spiels erarbeitet und abgewogen werden, sodass ein breites Spektrum an Regeln und Eigenschaften des Spiels erarbeitet werden kann. Diese Sammlung von Eigenschaften und Regeln kann im späteren Verlauf der Arbeit als Kriterienkatalog zur Auswahl einer Game-Engine verwendet werden. Dabei sollen zunächst für jede Kategorie verschiedene Möglichkeiten erklärt werden, um im Anschluss die für die Spielidee beste Möglichkeit auszuwählen.

	\subsubsection*{Regeln für den Spielfortschritt:}
		Zum Durchspielen des Spiels muss der Spieler eine Reihe von Level-Paketen abschließen. Die genaue Anzahl an zu spielenden Paketen ist variabel und kann jederzeit um ein zusätzliches Paket erweitert oder gekürzt werden. Die zu treffende Entscheidung an dieser Stelle ist, ob der Spieler die einzelnen Level-Pakete bereits von Beginn an auswählen kann oder sich diese im Laufe des Spiels freispielen muss. Ein Freispielen der einzelnen Level-Pakete gibt die Möglichkeit, einen Schwierigkeitsgrad einzuführen. Somit können zum Beispiel zuerst alle Pakete mit Schwierigkeitsgrad \enquote{Einfach} verfügbar sein. Mit erfolgreichem Abschluss aller einfachen Pakete würden dann die Pakete mit Schwierigkeit \enquote{Mittel} freigeschaltet.
		Die vollständige Verfügbarkeit aller Level-Pakete von Beginn an, hat für den Spieler den Vorteil der freien Wahl. Allerdings ist eine Umsetzung von unterschiedlichen Schwierigkeitsgraden unter Umständen nicht möglich. Durch sofortige vollständige Verfügbarkeit bleibt eine Lernkurve aus und der Spieler kann schnell die Motivation am Spiel verlieren.
		Aus den genannten Punkten ergibt sich die Entscheidung für das Freispielen. Somit ist eine Lernkurve gewährleistet. Für das Abschließen des Spiels muss der Spieler alle vorangegangenen Level erfolgreich gespielt haben. Dies gewährleistet dem Spieler einen \enquote{roten Faden durch das Spiel} und ermöglicht einen linearen Spielfortschritt.

	\subsubsection*{Regeln innerhalb der Level-Pakete:}
		Ein Level-Paket besteht aus insgesamt drei Rennen. Diese sind aufgeteilt in zwei Qualifikationsrennen und ein Pokalrennen. Für den Fortschritt innerhalb eines Level-Pakets kann entweder eines der Qualifikationsrennen abgeschlossen werden, um das Pokalrennen freizuschalten oder beide Qualifikationsrennen in entweder festgelegter oder beliebiger Reihenfolge müssen abgeschlossen werden, um das Pokalrennen verfügbar zu machen.
		Für den Umfang des Spiels wird eine Größe von drei Rennen pro Level-Paket festgelegt. Der Abschluss eines einzelnen Qualifikationsrennens für die Freischaltung des Pokalrennens ist zwar grundsätzlich möglich, führt jedoch zu einer Minderung des Spielumfangs. Um dies zu vermeiden, besteht zumindest die Anforderung, beide Qualifikationsrennen abschließen zu müssen. Die Reihenfolge der Qualifikationsrennen spielt im Bereich der Entwicklung keine Rolle. Dieser Aspekt ist insofern zu vernachlässigen. Für den Spieler kann eine sofortige Verfügbarkeit beider Qualifikationsrennen ein gewisses \enquote{Freiheitsgefühl} auslösen. Aus diesem Grund werden mit Beginn eines Level-Pakets beide Qualifikationsrennen sofort verfügbar.

	\subsubsection*{Gewinnbedingungen eines Rennens:}\label{gewinnbedinungen}
		Zur Durchführung der Einzelspieler Rennen stehen grundsätzlich zwei Möglichkeiten zur Auswahl. Die erste Möglichkeit ist, dass der Spieler gegen computergesteuerte Rennteilnehmer antritt. Mit realisierung dieser Alternative ist ein erhöhter Entwicklungsaufwand für die Steuerungslogig der Computergegner verbunden. Weiterhin werden dadurch Punkte, wie beispielsweise die Kollision mit anderen Rennteilnehmern notwendig. Um ein Rennen zu gewinnen, muss der Spieler als erster die Ziellinie überqueren.
		Die zweite Möglichkeit wäre die Realisierung der Rennen über eine vorgegebene Zeit. Das so entstehende Zeitrennen muss in der vorgegeben Zeit absolviert werden. Überschüssige Zeit verfällt und kann für folgende Rennen nicht verwendet werden.
		Unter Beachtung der erwähnten Punkte und mit Blick auf den gesetzten Zeitrahmen wird von der Entwicklung von Computergegnern abgesehen. Somit bleibt für die Realisierung des Rennens das Zeitrennen übrig.
		Das richtige Beantworten von Aufgaben sollte ebenfalls in die Gewinnbedingungen einfließen, wenn keine einzige Aufgabe korrekt ist, verliert der Spieler. Die konkrete Umsetzung befindet sich in Tabelle \ref{pokal-tabelle}.

	\subsubsection*{Zusatzaufgaben während eines Rennens:}\label{par:aufgaben}
		Der Spieler soll während eines Rennens Zusatzaufgaben erhalten. Eine Möglichkeit für die Umsetzung dieser Zusatzaufgaben konnte während des Brainstormings erarbeitet werden. Diese hat sich im Laufe der Zeit weiterentwickelt und wird weiter verfeinert. Auf eine ausführliche Erläuterung des exakten Werdegangs wird an dieser Stelle verzichtet.
		Bei der erarbeiteten Umsetzung bekommt der Spieler während des Rennens eine Aufgabe auf einer freien Stelle des Bildschirms angezeigt. Nach einer vorgegebenen Zeit X (diese variiert anhand der Schwierigkeit der Aufgabe oder dem Verlauf der Strecke) verschwindet diese Aufgabe und die Rennstrecke führt zu einer Weggabelung. Beide Wege sind mit Schildern versehen, auf denen jeweils mögliche Lösungen der Aufgabe angezeigt werden. Auf einem der Schilder steht somit die korrekte Lösung, auf dem anderen Schild stehen falsche Lösungen. Der Spieler muss nun sein Auto auf den richtigen Weg lenken.
		Eine weitere Möglichkeit wäre, das Rennen an einem bestimmten Punkt zu pausieren und eine Aufgabe anzuzeigen. Die gemessene Zeit des Zeitrennens läuft dabei weiter. Erst mit Eingabe der korrekten Lösung wird das Rennen fortgesetzt und der Spieler darf weiterfahren. Dies würde jedoch den Spielfluss maßgeblich unterbrechen und somit zu einem Verlust an Spielspaß führen. Somit fällt die Wahl für die Zusatzaufgaben auf die Weggabelungen im Rennen.

	\subsubsection*{Belohnungen / Strafen bei Zusatzaufgaben:}
		Da die Zusatzaufgaben fester Bestandteil des Spiels und zur Übermittlung des Lernerfolgs nötig sind, muss ein Absolvieren bzw. Nichtabsolvieren der Aufgaben auch Auswirkungen auf das Spiel haben. Für die Zusatzaufgaben innerhalb der Rennen ergeben sich zwei mögliche Lösungen.
		Die erste Möglichkeit zur Lösung dieses Problems ist, die Weggabelung der Strecke in zwei Streckenabschnitte mit identischer Länge zu führen und die Belohnungen / Strafen über die benötigte Zeit zu regeln. Mit der bestrafenden Methode würde ein Wählen des falschen Wegs zu einem Abzug in der verbleibenden Zeit führen, was ein erfolgreiches Absolvieren des Rennens maßgeblich erschwert. Ein Wählen des richtigen Wegs hätte in diesem Fall keine Auswirkungen auf den Spieler.
		Bei der belohnenden Methode würde ein Wählen des richtigen Wegs einen Zeitbonus und somit eine Erleichterung für den Spieler bedeuten und der falsche Weg hätte keine Auswirkungen. Eine Verbindung beider Methoden ist selbstverständlich möglich.
		Die zweite Möglichkeit besteht darin, an den Weggabelungen unterschiedliche Streckenabschnitte zu positionieren. Somit würde die falsche Wahl des Wegs zu einem längeren Streckenabschnitt führen, was den Spieler wichtige Zeit kostet.
		Da eine fundierte wissenschaftliche Entscheidung bezüglich der Wahl einer der genannten Methoden sehr schwer fällt, wird in der Umsetzung des Spiels darauf verzichtet.
		Die Anzahl an gelösten Aufgaben ist somit unabhängig von der erreichten Zeit im Rennen. Für die Auswertung eines Rennens wird in der Implementierung eine Entscheidungstabelle erstellt, welche aus der Anzahl korrekter Aufgaben und der gefahrenen Zeit einen Pokal auswählt.
		Als Belohnung und Bestrafung wird eine Sammlung von Reaktionstönen fungieren, beispielsweise ein Jubeln, oder Händeklatschen als positive Rückmeldung und Ausbuhen als negative Rückmeldung.

	\subsubsection*{Zusatzaufgaben im Mehrspieler-Modus:}
		Für die Implementierung der Zusatzaufgaben im Mehrspieler-Modus werden zunächst die Zusatzaufgaben aus dem Einzelspieler-Modus übernommen. Das bedeutet, dass auch im Mehrspieler-Modus Weggabelungen existieren werden, die Einfluss auf das Rennergebnis haben. Diese werden für jeden Spieler einzeln berechnet, sodass ein Abschauen von dem Vordermann nicht möglich ist. Zusätzlich zu diesen Aufgaben sollen weitere Aufgaben zu Beginn des Rennens eingeführt werden. Dabei besteht die Möglichkeit, einen \enquote{Start-Timer} einzuführen. Dieser würde eine festgelegte Anzahl an Sekunden herunter zählen, nach welchen das Rennen für jeden Spieler startet. Diejenigen Spieler, die allerdings in einer kürzeren Zeit eine Aufgabe lösen können, dürfen früher starten und erhalten somit einen Vorteil fürs Rennen.
		Durch diese Lösung ist gewährleistet, dass jeder Spieler starten kann, ungeachtet ob die Aufgabe lösbar ist. Durch eine exakte Optimierung dieser Zeitspanne kann gewährleistet werden, dass sowohl Können beim Fahren sowie das schnelle Lösen von Aufgaben zum Sieg beitragen.

	\subsubsection*{Umsetzung einer Ingame-Währung:}
		Um dem Spieler die Möglichkeit zu geben, Boni und optische Upgrades zu erwerben, soll dem Spiel eine Ingame-Währung beigefügt werden. Potentiell sind drei Möglichkeiten für den Erwerb dieser Währung denkbar.
		\begin{enumerate}
			\item{ Erwerb der Währung über reale Zahlungen. Dies ist die Einnahmequelle für Spiele nach dem \enquote{freemium} Modell\footcite[Seite 8]{freemium}. }
			\item{ Verdienen der Währung über das Gewinnen von Rennen. Diese Siegerprämie kann als zusätzliche Motivation für den Spieler dienen, Level-Pakete zu wiederholen um zusätzliche Währung zu verdienen. }
			\item{ Lösen von zusätzlichen Aufgaben. }
		\end{enumerate}
		Die erste genannte Möglichkeit der Bezahlung mit realen Transaktionen wird an dieser Stelle verworfen, da das Spiel nicht gewinnorientiert ist. Außerdem kann somit das Risiko von versehentlichen Belastungen der elterlichen Kreditkarten verhindert werden.
		Die beiden übrigen Möglichkeiten bieten dem Spieler Motivation und vergrößern den potentiellen Lernerfolg. Somit wird für die Umsetzung des Spiels eine Kombination aus Siegerprämie und (täglich beschränkten) Zusatzaufgaben implementiert.

	\subsubsection*{Verlassen des vorgegebenen Wegs:}\label{par:streckendesign}
		Jedes Rennen verfügt über einen vorgegeben Weg, welcher von Start bis zum Ziel führt. Dieser vorgegebene Weg verbreitert sich um breite Randstreifen zur fertigen Strecke. Ein Verlassen der Straße muss in gewisser Weise Auswirkungen auf das Fahrzeug des Spielers haben, um das Fahren auf der Strecke zu begünstigen. Denkbar wäre, das Fahrzeug des Spielers merklich zu verlangsamen, falls dieser die Strecke verlässt. Zusätzlich sollte dem Spieler eine Warnung zum Zurückkehren auf die Strecke angezeigt werden. Weiterhin kann das Fahrzeug des Spielers automatisiert auf die Strecke zurückgesetzt werden, falls der Spieler den vorgegebenen Weg für eine gewisse Zeit X verlässt. Gleiches gilt, falls ein Spieler sein Fahrzeug wendet und die Strecke in entgegengesetzter Richtung fährt. Dabei soll das Fahrzeug jedoch nicht verlangsamt werden. Für die Implementierung wird vorrangig eine Verlangsamung des Fahrzeugs gewählt.
		Somit hat ein Verlassen der Strecke einen merklichen Nachteil im Zeitrennen, da die Verlangsamung viel Zeit kostet. Auf ein automatisches Zurücksetzen wird verzichtet, damit der Spieler aus seinen Fehlern lernt, da er selbstständig auf die Strecke zurückfahren muss. Durch Level-Design kann ein Verlassen des vorgegebenen Wegs großteils vermieden werden, neben der Rennstrecke befinden sich Blockierelemente und der Level beschränkt sich auf einen breiten Streifen neben der Straße.

	\subsubsection*{Hindernisse auf der Strecke:}\label{par:streckendesign2}
		Um den Schwierigkeitsgrad der Rennen zusätzlich zu erhöhen, besteht die Möglichkeit, Hindernisse in den Streckenverlauf einzubauen. Für eine Kollision mit einem solchen Hindernis muss geklärt werden, welches Verhalten das Fahrzeug zeigt. Die beiden naheliegenden Möglichkeiten für eine Kollisionsreaktion sind das vollständige Stoppen (Abprallen) und das Verlangsamen des Fahrzeugs.
		Ein vollständiges Stoppen des Fahrzeugs würde eine Art Unterbrechung des Spielflusses bedeuten. Da dies, wie bereits erwähnt, nach Möglichkeit vermieden werden soll, scheidet diese Möglichkeit weitgehend aus. Ein Verlangsamen des Fahrzeugs kann entweder als Verlangsamung um einen Prozentsatz X (basierend auf der aktuellen Geschwindigkeit) oder mittels einer Verlangsamung der Geschwindigkeit auf einen Festen Wert umgesetzt werden.
		Aus Gründen der Plattform kann eine detaillierte Physiksimulation nicht umgesetzt werden. Stattdessen sollen Hindernisse, welche sich auf der Strecke befinden, mit dem Fahrzeug interagieren (vgl. weggestoßen werden). Für Objekte, welche sich neben der Strecke befinden, wird trotz der Unterbrechung des Spielflusses ein vollständiges Stoppen gewählt. Somit wird der Spieler weiterhin dazu animiert, die Strecke nicht zu verlassen.

	\subsubsection*{Kaufbares für Ingame-Währung:}
		Im Spiel sind zwei Kategorien von kaufbaren Dingen denkbar. Die erste Kategorie sind optische Upgrades. Diese beziehen sich vor allem auf die verwendbaren Fahrzeuge und Gegenstände. So kann der Spieler sein von Beginn an verfügbares Auto beispielsweise gegen einen Ball, einen Läufer, einen Rennwagen oder ähnliches austauschen. Zu beachten ist dabei, dass alle Fahrzeuge die gleiche Geschwindigkeit haben und somit kein Vorteil für den Mulitplayer erkauft werden kann. Die zweite Kategorie sind Boni, welche auf den Einzelspieler-Modus bezogen funktionieren. Dabei bekommt der Spieler die Möglichkeit, einen Zeitbonus (von beispielsweise 10 Sekunden) zu erkaufen, um das nächste Rennen zu erleichtern. Diese Möglichkeit besteht sowohl für die Qualifikations-, sowie für die Pokalrennen und ist im Mehrspieler-Modus nicht verfügbar.

	\subsubsection*{Steuerung:}
		Weiterhin besitzen die grundsätzliche Benutzeroberfläche, sowie die Steuerung Relevanz. Dabei soll der Fokus an dieser Stelle weniger auf dem Design der Oberfläche oder des Menüs, sondern mehr auf einer Beschreibung der vorhandenen Knöpfe und Funktionen liegen.
		Die Steuerung des Fahrzeugs könnte in mehreren Varianten realisiert werden, mittels:
		\begin{description}
			\item[Neigungssensor]{ Durch Neigung des Handys fährt das Auto entsprechend Links oder Rechts, dabei entscheidet der Grad der Neigung, wie stark die Kurve gefahren wird. Wichtig ist hier eine Kalibrierung des Sensors um den korrekten Nullpunkt zu verwenden. Ebenso sollte der Bereich, welcher das Auto geradeaus fahren lässt, eine gewisse Breite aufweisen, so sorgt leichtes Wackeln nicht zum Kurvenfahren. Die generelle Empfindlichkeit sollte für Anwender konfigurierbar sein. }
			\item[Lenkrad auf dem Touchscreen]{ Auf dem Bildschirm wird ein Lenkrad aus der Sicht des Fahrers gezeigt, der Nutzer kann das Lenkrad drehen und Links- oder Rechtskurven bewirken. }
			\item[Direkteingabe über den Bildschirm]{ Mit dem Finger wird der Bildschirm berührt, das Spiel berechnet dann die gewünschte Richtung und lässt das Auto an die zuletzt berührte Stelle fahren. In manchen Implementierungen werden bei dieser Art der Steuerung auch Hindernisse umgangen,\footnote{Beispielsweise in \enquote{League of Legends}.} Dies ist in Rennspielen nicht empfehlenswert, da die Schwierigkeit im Fahren an sich besteht.  }
			\item[Digitaler Joystick]{ Ein Joystick, welcher relativ zum Auto interpretiert wird. Nach rechts ziehen lässt das Auto eine Rechtskurve fahren. Nach unten ziehen bremst und kann auch rückwärtsfahren bewirken. }
			\item[Digitaler Joystick (nur Richtung)]{ Ein Joystick, welcher mit zwei Achsen die Fahrtrichtung direkt angibt. Die Richtung sollte relativ zur Kamera interpretiert werden, sodass ein nach rechts ziehen des Joysticks auch einem nach rechts fahren entspricht. }
		\end{description}
		Lenkung durch den Neigungssensor wird verworfen, weil dadurch die Aufgaben schwerer zu lesen sind.
		Ein Lenkrad auf dem Bildschirm nimmt zu viel Platz auf dem Bildschirm ein.
		Direkteingabe über den Bildschirm ist zu einfach zu steuern und wird bei höheren Geschwindigkeiten schnell untauglich.
		Ein Joystick, welcher gleichzeitig Gas gibt und die Fahrzeugrichtung bestimmt, vereinfacht die Steuerung zu arg und vermindert das Gefühl ein Auto zu fahren.

		Aus diesen Gründen wird eine Steuerung mit digitalem Joystick, welcher die Fahrtrichtung übernimmt und Pedalen für Gas und Bremse gewählt. Rückwärts fahren soll dabei nur aus dem Stillstand durch ein Lenken nach hinten und Gas geben möglich sein.
		Um die Geschwindigkeit des Fahrzeugs zu halten, muss das Gaspedal durchgehend gedrückt werden.
		Ein Loslassen des Gaspedals lässt das Fahrzeug bis zum Stillstand rollen. Ein Betätigen des Bremspedals führt zu einer deutlich schnelleren Reduzierung der Geschwindigkeit.
		Ein Beispiel für digitale Joysticks, welche in Mobilen-Spielen relativ üblich sind, ist in der folgenden Abbildung zu sehen.

		\figur{dpad.png}{Digitaler Joystick auch D-Pad unten links. Abbildung \cite{easports}\footnotemark}%
		\footcitetext[\url{http://www2.ea.com/uk/fifa-11-iphone/images/f784dabc26b7b210VgnVCM2000001165140aRCRD}]{easports}

	\subsubsection*{Heads-Up Display:}\label{HUD}
			Das Heads-Up Display beschreibt die eingeblendeten Anzeigen und Interaktionselemente während das Spiel läuft.
			Die wichtigsten Elemente sind hier die Steuerelemente: Gas, Bremse und digitaler Joystick gefolgt von dem Menübutton und dem Neustartknopf.
			Hinzu kommen die angezeigte Aufgabe und die beiden Antwortmöglichkeiten, eine Übersichtskarte, welche den Streckenverlauf anzeigt, und eine Zeitanzeige, welche den zeitlichen Rahmen klar macht.
			Bei den verschiedenen Elementen soll farblich klar sein, mit welchen interagiert werden kann und mit welchen nicht.

	\subsubsection*{Kameraperspektive:}
		Grundsätzlich sind für Rennspiele drei Kameraperspektiven denkbar. Diese sind:
		\begin{itemize}
			\item{ Die Kamera zeigt das Innere des Fahrzeugs und schaut durch die Frontscheibe. Diese Perspektive wird auch \enquote{First-Person} genannt. }
			\item{ Die Kamera befindet sich hinter dem Fahrzeug, zeigt somit das Fahrzeug und die Strecke. Dies wird auch \enquote{Third-Person} oder Vogelperspektive genannt. }
			\item{ Die Kamera befindet sich über der Strecke und dem Fahrzeug. Dies wird auch \enquote{Top-Down View} genannt. }
		\end{itemize}
		Da das Spiel für Mobile-Endgeräte entwickelt wird, ist eine Ansicht innerhalb des Autos nicht gut geeignet, da die Steuerungselemente sowie die Finger des Spielers die Sicht zu sehr einschränken. Somit fällt diese Option weg.
		Befindet sich die Kamera hinter dem Fahrzeug, ist die Sicht auf die kommende Strecke bei besonders kurvigen Strecken stark eingeschränkt. Da ein frühzeitiges Sehen des Streckenverlaufs hilft, gleichzeitig die gestellten Aufgaben zu lösen und das Fahrzeug zu steuern, fällt die Entscheidung an dieser Stelle auf die Ansicht von oben.
		Der Spieler kann so die Strecke früh erkennen und sich auf die kommenden Kurven einstellen, ohne dabei die Konzentration für die gestellten Aufgaben zu verlieren. Der Kamerawinkel wird leicht geneigt sein, um die Modelle plastischer zu gestalten und 3D-Modelle verwenden zu können.

	\subsubsection*{Musik und Sound:}
		Um den Umfang des Spiels zu vervollständigen, müssen an dieser Stelle noch Musik und Sound im Allgemeinen betrachtet werden. Wie im Brainstorming erwähnt, soll das Spiel eine Hintergrundmusik erhalten. Zudem werden Töne für Kollisionen, Fahrzeuge und bestimmte Momente im Spiel (Start, Ziel und Lösen einer Aufgabe) benötigt.
		Dazu soll der Spieler die Möglichkeit bekommen, im Spiel Einstellungen an der Tonausgabe vorzunehmen. Dazu gehören vor allem die Lautstärkeeinstellung und die Möglichkeit, den Sound vollständig zu deaktivieren. Dabei soll zwischen der Spielmusik, den Fahrzeuggeräuschen und den Umgebungsgeräuschen unterschieden werden.

\subsection{Entwicklung Play-Persona \label{ssec:personadef}}
	Wie in Kapitel \ref{ssec:persona} beschrieben, werden für die Entwicklung von Play-Personas die Spielmechaniken und Fähigkeiten der Spieler benötigt. Im Falle des zu entwickelnden Rennspiels ergeben sich zwei Kernkompetenzen, nach welchen die Spieler unterschieden werden können. Diese sind:
	\begin{description}
		\item[Rennen fahren]{Die Kompetenz des Rennenfahrens tätigt eine Aussage darüber, wie gut bzw. schlecht ein Spieler beim Steuern des Fahrzeugs ist. Diese Fähigkeit kann durch Rundenzeiten bzw. durch Platzierungen in Mehrspieler-Modus-Rennen gemessen werden.}
		\item[Aufgaben lösen]{Durch die Kompetenz des Aufgabenlösens wird abgedeckt, ob ein Spieler die gestellten Aufgaben richtig lösen kann, wie viel Prozent der Aufgaben korrekt gelöst sind und wie viel Zeit der Spieler dafür benötigt.}
	\end{description}
	Daraus ergeben sich folgende 4 Personas:

	\begin{tabl}{lcc}{Play-Persona Matrix}
		\toprule
			Name/Eigenschaft & Rennen fahren & Aufgaben lösen \\
		\midrule
			Anfänger & - & - \\
			Rennfahrer & + & - \\
			Lehrer & - & + \\
			Experte & + & + \\
		\bottomrule
	\end{tabl}

	\begin{description}
		\item[Die Grundschule Helmsheim]\hfill\\
		Helmsheim ist eine Kleinstadt in Deutschland mit knapp 6000 Einwohnern. Die Grundschule Helmsheim liegt am Stadtrand und ist eine der beiden Grundschulen in der Stadt. Die Grundschule ist recht fortschrittlich, da sie zwar staatlich betrieben, jedoch durch ein ortsansässiges IT-Unternehmen gesponsert wird. Durch dieses Sponsoring hat die Schule die Möglichkeit, neue Technologien einzusetzen und so beispielsweise educational games für den Unterricht zu verwenden.
		\item[Florian Klein: Der Anfänger]\hfill
		\begin{description}
			\item[Familiäres Umfeld]{Florian ist sieben Jahre alt und lebt mit seinen Eltern und zwei Geschwistern am Stadtrand. Seine Geschwister sind Andy(10) und Laura(4). Florians Vater arbeitet bei einem ortsansässigen IT Unternehmen, seine Mutter ist Hausfrau. Gelegentlich passt Florian auf seine kleine Schwester auf und macht regelmäßig seine Hausaufgaben aus der Schule, weitere Aufgaben im Haushalt hat er jedoch nicht.}
			\item[Schulisches Umfeld]{Florian besucht, genauso wie sein Bruder, die Helmsheimer Grundschule. Dort ist er aktuell in der zweiten Klasse. Florian befindet sich im schulischen Durchschnitt und findet keinen besonderen Spaß am Lernen.}
			\item[Interessen und Hobbys]{Florian ist ein begeisterter Auto-Fan. Er spielt gerne mit Spielzeugautos, sieht Autosendungen im Fernsehen und will später unbedingt Rennfahrer werden. Er hat sehr viel Spaß daran, Rennen zu fahren und hat bei seinem Bruder ein Auto-Rennspiel gesehen, dass er nun auch spielt. Außerdem spielt er in seiner Freizeit Fußball. Er hat eine Abneigung gegen kleine Autos (Smarts) und Mädchen.}
			\item[Motivation]{Die Motivation Florians liegt im Ehrgeiz, seinen Bruder zu schlagen. Er möchte bessere Rundenzeiten im Rennen erreichen und so lange immer besser im Spiel werden, bis er seinen Bruder im Mehrspieler-Modus besiegen kann.}
			\item{\enquote{Mama, Mama, guck mal ein rotes Rennauto!}}
		\end{description}
		\item[Joachim Wolf: Der Lehrer]\hfill
		\begin{description}
			\item[Familiäres Umfeld]{Joachim ist 58 Jahre alt und lebt mit seiner Frau Ulrike(56) in Helmsheim. Seine Tochter Natalie(25) wohnt nicht mehr bei ihnen, da sie für ihr Studium umgezogen ist. Joachim hat noch keine Enkelkinder. Während er nicht in der Grundschule ist, muss er den Unterricht für kommende Tage vorbereiten, Klassenarbeiten korrigieren und ab und an ein Seminar für Lehrer besuchen.}
			\item[Schulisches Umfeld]{Joachim hat Lehramt in den Fächern Geschichte und Geographie studiert, beschloss jedoch mit 41 das Gymnasium zu verlassen und an der Grundschule weiter zu arbeiten. Seitdem arbeitet er bei der Grundschule Helmsheim.}
			\item[Interessen und Hobbys]{In seiner Freizeit liest Joachim sehr viel. Vor allem Romane und Krimis interessieren ihn besonders. Er fährt viel Fahrrad und nutzt jede mögliche Gelegenheit, um im nahe gelegenen Stadtpark Schach mit seinen Freunden zu spielen. Zudem gibt er aus Hobby Nachhilfe und beschafft sich somit einen kleinen Nebenverdienst.}
			\item[Motivation]{Die größte Motivation für Joachim ist, den Kindern möglichst viel beizubringen. Dafür möchte er für die Kinder ein besseres Lernumfeld schaffen, damit die Kinder Spaß am Lernen haben. Um dies zu erreichen, möchte er mit seinen Klassen das spielerische Lernen ausprobieren. Er hofft, dass die Kinder freiwillig lernen (spielen), wenn sie auch Spaß daran haben.}
			\item{\enquote{Wissen sollte Spaß machen, um es in jungen Jahren nach vorn zu bringen.}}
		\end{description}
		\item[Laura Dietz: Die Rennfahrerin]\hfill
		\begin{description}
			\item[Familiäres Umfeld]{Laura ist 12 Jahre alt und lebt zusammen mit ihrer Mutter im Nachbarhaus von Joachim. Lauras Eltern leben getrennt. Laura ist ein Einzelkind und muss, da ihre Mutter arbeiten geht, regelmäßig im Haushalt helfen und mit dem Hund gehen. Zudem wurde sie von ihrer Mutter zur Nachhilfe von Joachim angemeldet. Sie geht dort hin, um ein gutes Verhältnis zu ihrer Mutter zu wahren, auch wenn sie die Zeit viel lieber mit ihrem dreizehnjährigen Freund und ihren Spielen verbringen würde.}
			\item[Schulisches Umfeld]{Sie besucht die fünfte Klasse einer Realschule in Helmsheim, interessiert sich allerdings nicht sonderlich fürs Lernen. Da sie die Vorschule besucht hat, gehört sie zu den Ältesten in ihrer Klasse. Da ihre Mutter schon früher nicht viel Zeit mit Laura verbringen konnte, hat Laura viel gespielt. Sie hat über diese Spiele viel gelernt und hat deshalb ein grundsätzliches Desinteresse an klassischen Lernformen.}
			\item[Interessen und Hobbys]{Laura verbringt gerne viel Zeit mit ihrem Handy oder am Computer. Sie nutzt beides hauptsächlich zum Spielen und zum Chatten mit ihrem Freund. Sie möchte endlich eine Spielekonsole haben, um ihre Leidenschaft zum Spielen noch weiter verfolgen zu können. Wenn sie nicht gerade spielt, trifft sie sich mit ihrem Freund.}
			\item[Motivation]{Laura wurde von Joachim gebeten, das Lernspiel auszuprobieren, da Joachim weiß, wie gern Laura spielt. Sie ist wenig motiviert, viel zu lernen, möchte jedoch Joachim den Gefallen tun, da ihre Mutter ja für die Nachhilfe mit Joachim zahlt.}
			\item{\enquote{Nur noch fünf Minuten spielen, Mama?}}
		\end{description}
		\item[Andy Klein: Der Experte]\hfill
		\begin{description}
			\item[Familiäres Umfeld]{Andy ist zehn Jahre alt und Bruder von Florian. Zu seinen Aufgaben gehören das Lernen und das Machen der Hausaufgaben, jedoch hat Andy daran deutlich mehr Spaß als Florian.}
			\item[Schulisches Umfeld]{Er besucht die vierte Klasse der Helmsheimer Grundschule und ist ein motivierter Schüler. Er verfolgt zur Zeit zwei schulische Ziele: Zum einen möchte er zum Klassenbesten in Mathematik werden, zum anderen bereitet er sich schon jetzt auf die weiterführende Schule vor, um einen möglichst guten Abschluss zu erreichen. }
			\item[Interessen und Hobbys]{Andy ist wissenschaftlich sehr interessiert. Besonders angetan ist er von Physik und Chemie. Er spielt gern mit seinem Chemiebaukasten, sowie andere Physik- und Chemie-, sowie Geschicklichkeitsspiele. Ab und an spielt er nebenbei allerdings auch Action- bzw. Rennspiele. Er geht einmal wöchentlich nach der Schule in einen Physik-Club for Kids, wo kleinere Experimente mit den Kindern durchgeführt werden.}
			\item[Motivation]{Seine Klasse wird in Mathematik von Joachim unterrichtet, daher sollen alle Kinder der Klasse das Lernspiel ausprobieren. Da er Klassenbester in Mathematik werden möchte, geht er mit besonders viel Enthusiasmus an das Spiel heran, um schnell möglichst hohe Erfolge zu erzielen. Außerdem macht ihm das Spielen Spaß, was zusätzlich zum Weiterspielen motiviert.}
			\item{\enquote{Mit Spielen Lernen, macht ja schon Spaß!}}
		\end{description}
	\end{description}

\subsection{Gamification Modell}\label{ssec:gamification-modell}
	Das Spiel soll durch einen gewissen Schwierigkeitsgrad herausfordernd aber trotzdem anfängerfreundlich sein. Ein Punktesystem und freiwilliges Bearbeiten von Aufgaben ermöglicht Spielern zusätzliche Funktionalitäten zu erkunden und dadurch fürs Lernen belohnt zu werden. Durch schwierig zu erreichende Gold-Pokale können Level längerfristig anspruchsvoll für Spieler sein.
	Durch Hinweise auf diese Zusatzfunktionen und Extras soll der Spieler kontinuierlich daran erinnert werden, dass er mehr erkunden könnte, dabei sollen diese Hinweise nicht störend oder aufdringlich sein.
	Die verschiedenen Personas sollen dabei wie folgt angesprochen werden:
	\begin{description}
		\item[Florian Klein]{
			Durch einen leichten Einstieg und viel verrückten Rennspaß kann der Anfänger viel lernen. Durch eine einfache Steuerung, viele Extras und auch Freunde bleibt er längere Zeit am Spiel und somit am Lernen begeistert.
		}
		\item[Joachim Wolf]{
			Der Lehrer hat die Möglichkeit durch Aufgaben und nicht Reaktionszeit oder Renntalent zu glänzen. So kann auch der Lehrer mit seinen Schülern mithalten und sich in Geschicklichkeit und Reaktionszeit üben.
		}
		\item[Laura Dietz]{
			Die Rennfahrerin stört sich wohl am meisten an den Aufgaben, kann aber im Rennen durch geschicktes Fahren wieder aufholen. Sie hat Spaß an den Extras, solange sie dafür nicht zu viel arbeiten muss.
			Durch anspruchsvollere Pokalrennen soll sie überzeugt werden, die Aufgaben zu lösen.
		}
		\item[Andy Klein]{
			Der Experte erkundet das Spiel schnell und hat später durch den Mehrspieler-Modus die Möglichkeit sich mit anderen zu vergleichen und zusammen Spaß zu haben. So bleibt auch für den Experten das Spiel interessant.
			Für ihn ist besonders wichtig, jeden Level mit der besten Wertung abzuschließen. Das sollte nicht zu schwer gemacht werden, um die anderen zu inkludieren.
		}
	\end{description}
	Spieler sollen für den ersten Platz durch Zeigen des Pokals nach einem Sieg belohnt werden. Sie werden durch eine Anzeige der erreichten Pokale im Menu darauf hingewiesen, in welchen Leveln noch Gold-Pokale fehlen und in welchen diese bereits erreicht sind. So werden die Spieler dazu animiert, diejenigen Level in denen ein Gold-Pokal fehlt, erneut zu spielen um diesen zu erreichen.
	Die Schwierigkeitsgrade der Level sollten innerhalb der Level-Pakete aufsteigend angeordnet sein. Dieses Prinzip kann auch auf die Levelpakete übertragen werden und somit das \enquote{easy to learn, hard to master} Prinzip unterstützen.\footcite[Bushnell's Law]{easy-to-learn-hard-to-master} Eine gewisse Konsistenz oder Kontinuität ist wichtig um das mentale Modell des Spielers nicht zu brechen, die Balance zwischen Konsistenz und Abwechslung ist hier zu treffen.
	Das Zusammenspiel auch einiger hier nicht erwähnten Dinge\footnote{Nicht alle Effekte die auf den Spieler wirken können vorher durchgeplant werden, einige werden auch ohne Absicht in das Spiel gelangen.} sorgt dann schließlich für den resultierenden Effekt Gamification.


\subsection{Lernziele Umsetzung}
	Die Lernstrategie ist Wiederholung und spielerisches Einsetzen von Aufgaben. Dazu sollen die Pokalrennen etwas Herausforderndes und Besonderes sein. Nutzer sollen durch Gamification motiviert werden mehr und länger zu spielen (siehe \ref{ssec:gamification-modell}).
	Die Definition der Lernziele befindet sich in Kapitel \ref{ssec:lern-def}.
	Wie schon in den Kapiteln \ref{ssec:spielidee}, \ref{ssec:idee} und \ref{ssec:requirements} beschrieben existieren mehrere Möglichkeiten, um Aufgaben in das Spiel einzubauen. Wichtig ist, dass für die Teilnahme an Pokalrennen Aufgaben gelöst werden müssen. Von den Aufgaben innerhalb der Rennen abgesehen, sind weitere Zusatzaufgaben optional, werden jedoch psychologisch gefördert (siehe \ref{ssec:psycho-grundlagen}).
	\begin{description}
		\item[Aufgabe zum Start des Rennens]{
			Eine Aufgabe um \enquote{den Motor zu starten} nimmt etwas Spannung aus dem Start, passt aber in das Spielkonzept und stört deshalb möglicherweise weniger.
			Für den Einzelspielermodus wird dies nicht umgesetzt, für einen potentiellen Mehrspielermodus sollen Aufgaben zum Rennstart hinzugefügt werden.
		}
		\item[Aufgaben in Form von Weggabelungen]{
			Am oberem Rand des Bildschirms taucht die Frage auf und kurz später eine Weggabelung, beispielsweise ein Tor mit zwei Durchgängen. Durch so eine spielerische Nutzung von Aufgaben kann ein Spieler motiviert werden zu lernen.
		}
		\item[Aufgaben für Münzen]{
			Münzen können später in normalen Rennen für witzige aber auch hilfreiche Dinge eingesetzt werden. Hilfreiche Dinge sind unter anderem ein Zeitbonus oder eine erhöhte Geschwindigkeit. Außerdem sollen auch Dinge verfügbar sein, welche dem Spieler keinen Vorteil bringen. Diese können beispielsweise optische Verbesserungen oder andere Fahrzeuge sein.
		}
		\item[Teilnahmegebühr für Pokalrennen]{
			Eine Teilnahmegebühr für Pokalrennen bringt nicht nur den Spieler zum Lernen, sondern setzt die Pokalrennen von den normalen Leveln ab und macht sie so zu etwas besonderem. Die für die Gebühr benötigten Münzen kann der Spieler durch wiederholtes Spielen der Level verdienen. Somit steigt der Lernerfolg durch Wiederholung weiter an. Durch einen besonders hervorgehobenen Leveleingang innerhalb des Menü-Levels soll der Spieler zusätzlich darauf aufmerksam gemacht werden.
		}
	\end{description}
	Im Optimalfall werden die verschiedenen Alternativen durch Spieltests empirisch belegt.

	Die Idee ein Punktekonto für Aufgaben zu erstellen ist von dem \enquote{pay-to-win}\footcite{pay-to-win} Modell  abgeleitet und \enquote{learn to win} getauft.
	Ein Spiel ist \enquote{pay-to-win} wenn dieses durch kaufen von Spielgegenständen oder sonstigen Verbesserungen Spielern deutliche Vorteile gewährt:
	\begin{quote}
		\vspace{\baselineskip}\hfill\begin{minipage}{0.96\textwidth}
			[\dots] to succeed, the player really needs to buy additional ingame enhancements, [\dots] making the game a pay-to-win game; [\dots]
		\end{minipage}
		\attrib{2016 Kimppa, Heimo und Harviainen\footcite[Seite 1]{pay-to-win}}
	\end{quote}
	Andy Hartup nennt diese Taktik auch \enquote{pay-to-advance}.\footcite{free-to-play_is_pay-to-win} Die Schwierigkeit liegt darin, die Balance zu zwischen zu viel und zu wenig Bonus zu treffen.

