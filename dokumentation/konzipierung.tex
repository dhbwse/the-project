% !TEX root = dokumentation.tex
\section{Konzipierung}
Das folgende Kapitel beschäftigt sich mit der Konzipierung des Spiels an sich. Dabei werden zunächst die erarbeiteten und vorgegebenen Kriterien zusammengefasst, mehrere Spielideen anhand dieser Kriterien überprüft und anschließend eine finale Spielidee entwickelt. Im weiteren Verlauf des Kapitels werden Functional und non-functional Requirements erörtert und in Use-Cases unterteilt. Anhand dieser Requirements wird dann im Kapitel \ref{sec:impl} Implementierung das Spiel entwickelt. Weiterhin werden in diesem Kapitel die Spielregeln verfasst und weitere Elemente, wie die Steuerung oder computergesteuerte Gegner (e.g. Bots) betrachtet. Abschließend soll betrachtet werden, wie die im Grundlagenteil erfassten Lernziele in Verbindung mit dem Spiel umgesetzt werden können.
\subsection{Kriterien für die Spielidee}\label{ssec:kriterien}
	<TODO:Fazite aus vorgehenden Kapiteln klar einarbeiten>
	Auf Basis der in der Aufgabenstellung genannten Kriterien und mit Erweiterung durch eigens erarbeitete Kriterien konnte ein Kriterienkatalog entwickelt werden. Damit eine vollständige Zusammenfassung aller Kriterien möglich ist, werden die Kriterien aus der Aufgabenstellung erneut aufgegriffen.\footnote{<TODO: nein>}
	\begin{enumerate}
		\item{Das Spiel soll levelbasiert sein, d.h. dass voneinander unabhängige Spielabschnitte vorhanden sind.}
		\item{Der Umfang des Spiels ist auf zwei bis drei Stunden Spielspaß ausgelegt. Durch modulare Gestaltung (Levels) kann das Spiel im Nachhinein erweitert werden.}
		\item{Das Spiel ist ein \enquote{educational game}, der Spielspaß steht im Vordergrund. Die Spieler sollen unbewusst nebenbei lernen.}
		\item{Aus Punkt 3 folgt: das Lernen selbst soll Spaß machen. Dies soll erreicht werden, indem Lernen ein fester Bestandteil des Spiels ist und nicht als Aufgabe, zu der die Spieler genötigt werden.}
		\item{Das Spiel soll \enquote{casual} sein, d.h.\@ ein Spiel \enquote{für zwischendurch} sein.}
		\item{Der vollständige Programmieraufwand für das Spiel ist auf 3 Monate Arbeit ausgelegt. Somit bleiben weitere volle drei Monate für das Verfassen der Arbeit.}
		\item{Ziel ist, eine originelle Spielidee zu entwickeln, sodass das Spiel kein Klon eines bereits existierenden Spiels ist.}
		\item{Als Beispiele zur Orientierung wurde \enquote{arcademics.com} genannt, welche Spiele der gewünschten Art beinhaltet, jedoch eine Internetverbindung benötigt. Daraus ergibt sich, dass das Spiel auch offline spielbar sein soll.}
		\item{Bei frühen Brainstorming-Sessions konnte sich darauf geeinigt werden, dass das Spiel in keinem Fall ein \gls{RPG} sein soll.}
		\item{Es besteht keine Notwendigkeit der Einhaltung der Lernstandards. Die Spieler sollen mit dem Spiel Spaß haben und das Spiel auch spielen wollen.}
		\item{Die Zielgruppe für das Spiel ist durch die Personas aus Kapitel \ref{ssec:persona} gegeben. Nach Möglichkeit soll dieses Spiel und die zugehörige Lernplattform auch Flüchtlingskindern zur Verfügung gestellt werden.}
	\end{enumerate}

	Die Entstehung dieser Kriterien ist teilweise auf die Ergebnisse aus dem Brainstorming\footnote{<TODO: Brainstorming erklären und verlinken>} zurückzuführen. Durch das Brainstorming konnten die Erwartungen der Projektteilnehmer an das Spiel weiter spezifiziert werden. Somit ist eine genaue Trennung der Kriterien aus dem Brainstorming nur schwer möglich und dient lediglich der Übersichtlichkeit.

\subsection{Lernziele Definition}\label{ssec:lern-def}
	Als Lernziel haben sich die Autoren auf die Vermittlung von Faktenwissen geeinigt. Soll heißen beliebige Fragen mit einer eindeutigen Antwort. Mögliche Themengebiete sind so unbeschränkt unter anderem Mathematik, Naturwissenschaften, Sprachwissen(Übersetzen wie Grammatik) oder auch Geschichte. So muss für das Spiel nur eine Liste von Fragen mit Antwort und möglichen falschen antworten gepflegt werden.
	Die Lernstrategie soll durch Wiederholung von Aufgaben und spielerischem Anregen zum Lernen umgesetzt werden.

\subsection{Brainstorming \& Game Design}\label{ssec:idee}
	Über mehrere Gespräche hinweg wurden Ideen für das zu entwickelnde Spiel gesammelt. Dieses Kapitel hat zum Zweck, den Ablauf der Ideenfindung in eine Reihenfolge zu bringen und diese zu präsentieren.
	\begin{enumerate}
		\item{Chemie / Physik Simulation:}
		\begin{itemize}
			\item{Die Simulation findet in einer Laborumgebung statt. Dabei steuert der Spieler den Laborarbeiter (bzw. aus der Sicht des Laborarbeiters).}
			\item{Der Spieler führt einfache Experimente aus den Bereichen Chemie und Physik durch.}
			\item{Fehler, welche dem Spieler in den Experimenten passieren, führen zum Scheitern des Versuchs.}
			\item{Der Spieler hat die Möglichkeit, die Hände des Laborarbeiters zu steuern.}
			\item{Alternativ kann der Spieler über den Touchscreen mit den Utensilien im Labor interagieren.}
			\item{Die Idee wurde verworfen, da eine realistische Steuerung von Experimenten aus Chemie und Physik nur schwer über einen Touchscreen umzusetzen ist. Weiterhin ist die für diese art von Spiel benötigte Physiksimulation ungeeignet für die Zielplattform. Außerdem stellt diese Idee nicht den gewünschten Lernerfolg für die gewünschte Zielgruppe zur Verfügung. }
		\end{itemize}
		\item{Rätselspiel:}
		\begin{itemize}
			\item{Der Spieler soll eine Art Rätsel lösen, welche automatisch in unterschiedlichen Schwierigkeitsgraden generiert werden kann.}
			\item{Die Idee wurde aufgrund eines fehlenden Rätselkonzepts, fehlender weiterer Ideen und großer Einschränkung der Zielgruppe verworfen.}
		\end{itemize}
		\item{Simulation von Motoren und Getrieben:}
		\begin{itemize}
			\item{Vor Spielbeginn bekommt der Spieler eine Strecke angezeigt. Diese Strecke ist in 2D und zeichnet sich durch Höhenunterschiede aus.}
			\item{Der Spieler wählt auf Basis der Strecke aus einer Sammlung verschiedener Motoren den Besten aus. Beispielsweise ist ein 2-Takt Motor für steile Berge besser geeignet als ein 4-Takt Motor.}
			\item{Der Spieler bekommt mehrere Treibstoffe (Benzin, Diesel, \dots) zur Auswahl und kann entsprechend der Strecke und des Motors wählen.}
			\item{Während der Fahrt hat der Spieler die Möglichkeit, über Schaltflächen den ausgewählten Gang des Getriebes zu wechseln.}
			\item{Bei der Auswahl von Treibstoffen, Motoren und Gangschaltungen werden dem Spieler Informationstexte angezeigt und vorgelesen.}
			\item{Um das Level bzw. das Spiel abzuschließen, muss die Strecke in der vorgegebenen Zeit absolviert werden.}
			\item{Die Idee wurde nach dem Kickoff-Meeting vollständig verworfen, da die Umsetzung der Lernziele nicht in gewünschter Form möglich ist.}
		\end{itemize}
		\item{Rennfahrspiel Version 1:}
		\begin{itemize}
			\item{Der Spieler nimmt an einer wilden Renntour durch die Stadt teil.}
			\item{Der Spieler wird von der Polizei verfolgt, sollte er gestellt werden bekommt er verschiedene Animationen des Scheiterns gezeigt.}
			\item{Sollte der Spieler scheitern, kann er eine Aufgabe absolvieren um einen weiteren Versuch zu erhalten. Das ist dreimal möglich.}
			\item{Diese Aufgaben / Minigames sind beispielsweise:}
			\begin{itemize}
				\item{Ampel: Der Spieler muss rechtzeitig reagieren, um an einer roten Ampel anzuhalten.}
				\item{Polizei: Der Spieler muss die Anweisungen der Polizei rechtzeitig befolgen. Anweisungen können z.B.\@ das \enquote{Rechts ran fahren}sein.}
				\item{Sprungschanze: Der Spieler benötigt die richtige Geschwindigkeit, um den Sprung perfekt zu absolvieren.}
			\end{itemize}
			\item{Die Idee wurde nach dem Kickoff-Meeting vollständig verworfen, da die Umsetzung der Lernziele nicht in gewünschter Form möglich ist.}
		\end{itemize}
	\end{enumerate}
	Aus diesen Ideen und in Verbindung mit den Informationen aus dem Kickoff-Meeting konnte die finale Spielidee entwickelt werden. Dabei wurde die Idee eines Rennspiels beibehalten und so angepasst, dass eine Umsetzung der Lernziele gut umsetzbar ist.
	\begin{enumerate}
		\item{Das Rennspiel wird dem Spieler aus der \enquote{Top-Down}-Ansicht präsentiert. Dabei ist die Kamera leicht schräg auf die Rennstrecke gerichtet, sodass die Modelle der Fahrzeuge dreidimensional sind.}
		\item{Als Steuerung nutzt der Spieler ein Steuerkreuz, sowie einen Button für Gas und einen Button für Bremse. Somit muss der Spieler sowohl lenken, als auch seine Geschwindigkeit steuern.}
		\item{Bei der Auswahl der spielbaren Objekte ist kein Fokus auf Realismus gesetzt. Beispielhafte spielbare Dinge sind:}
		\begin{itemize}
			\item{Ein rollender Ball und ein rollender Würfel (welcher logischerweise ungleichmäßige Bewegungen vollführt).}
			\item{Ein rennender Mensch oder ein rennendes Tier (welche die gleiche Geschwindigkeit wie Autos erreichen).}
			\item{Ein Auto oder ein Flugzeug, welches sich in Bodennähe aufhält und auf die Gegebenheiten der Strecke ebenso reagieren muss.}
		\end{itemize}
		\item{Für die verschiedenen Spielmodi stehen folgende Ideen zur Auswahl:}
		\begin{itemize}
			\item{Im Online Multiplayer haben die Spieler die Möglichkeit, gegen bis zu drei weitere Spieler anzutreten. Gewinner ist dabei logischerweise, wer zuerst das Ziel erreicht.}
			\item{Die Level für den Spielfortschritt sind als Zeitrennen umgesetzt. Dabei muss der Spieler das Ziel in der vorgegebenen Zeit erreichen. Am Spielende überschüssige Zeit kann unter Umständen in zusätzliche Belohnungen verwandelt werden.}
			\item{Die Level für den Spielfortschritt werden gegen computergesteuerte Bots statt, um das Level zu beenden, muss der Spieler als erster das Ziel erreichen.}
		\end{itemize}
		\item{Die Strecken werden in Level-Paketen ausgeliefert. Diese Level-Pakete müssen von dem Spieler durch Abschluss des vorherigen Level-Pakets freigeschaltet werden.}
		\begin{itemize}
			\item{Ein Level-Paket besteht aus zwei normalen Strecken und einem Finalrennen.}
			\item{Um an einer finalen Strecke teilnehmen zu können, muss der Spieler eine Startgebühr bezahlen. Verliert der Spieler dieses Rennen, wird die Startgebühr erneut fällig.}
			\item{Mit dem Abschluss der zwei normalen Strecken wird das Finalrennen verfügbar. Mit dem Abschluss des Finalrennen wird das nächste Level-Paket freigeschaltet.}
		\end{itemize}
		\item{Auf den Strecken werden verschiedene Abzweigungen möglich sein, bei welchen der Spieler die Auswahl zwischen zwei Wegen hat.}
		\begin{itemize}
			\item{Die Auswahl des \enquote{richtigen} Wegs wird über Aufgaben realisiert. Dabei wird dem Spieler rechtzeitig eine Aufgabe auf dem Bildschirm gezeigt.}
			\item{Beide möglichen Wege werden mit einer Lösung versehen. Eine dieser Lösungen ist die Lösung der vorher gezeigten Aufgabe. Dieser Weg gilt als der Richtige.}
			\item{Wählt der Spieler den falschen Weg, erhält der Spieler einen Nachteil. Im Fall des Zeitrennens kann dies z.B.\@ der Verlust von einigen Sekunden sein.}
		\end{itemize}
		\item{Im Multiplayer müssen die Spieler Aufgaben absolvieren, um das Rennen beginnen zu können.}
			\begin{itemize}
				\item{Sobald die Aufgabe gelöst ist, kann ein Spieler losfahren. Somit können sich Spieler über schnelles Absolvieren der Aufgaben einen Vorteil im Rennen erspielen.}
				\item{Können die Aufgaben nicht gelöst werden, darf der Spieler nach einer gewissen Zeit dennoch losfahren. Spieler, welche die Aufgaben gelöst haben, haben im Rennen dadurch einen Vorteil.}
				\item{Falscheingaben resultieren in einer neuen Aufgabe, um \enquote{cheating} zu verhindern.}
			\end{itemize}
		\item{Der Spieler erhält die Möglichkeit, zusätzliche Aufgaben auf freiwilliger Basis zu lösen. Als Belohnung wird eine Ingame-Währung eingeführt, welche unter Anderem für folgendes genutzt werden kann:}
		\begin{itemize}
			\item{Für ein Rennen gegen computergesteuerte Bots kann sich der Spieler eine bessere Startposition erkaufen.}
			\item{Für ein Zeitrennen kann sich der Spieler einen Zeitbonus erkaufen.}
			\item{Der Spieler kann die Teilnahmegebühr für die Finalrunden sowohl über die Ingame-Währung als auch über das Absolvieren von Aufgaben bezahlen.}
			\item{Der Spieler kann weitere Fahrzeuge / spielbare Objekte für die Ingame-Währung kaufen. Alternativ ist über diese Währung auch ein Farbwechsel der bisher gekauften Objekte oder ein Wechsel des Spielersymbols (für den Multiplayer) denkbar.}
		\end{itemize}
	\end{enumerate}
	<TODO:Berkling "Besser wäre hier gewesen, Kriterien aufzustellen, die Ideen anhand der Kriterien zu evaluieren und in einer Tabelle vergleichen und dann eine wählen.">

\subsection{Funktionale und nicht funktionale Anforderungen}\label{ssec:requirements}
	In diesem Unterkapitel werden die Anforderungen an das Spiel dokumentiert. Begonnen mit einer Übersicht der Funktionalitäten, dann eine detaillierte Beschreibung der Anforderungen der einzelnen Funktionen. Es folgen generelle nicht funktionale Anforderungen.
	Das Spiel wird hauptsächlich so gesteuert wie auch die Rennen gefahren werden: mit einem Auto. Das heißt, dass die Menüs auch mit einem Auto angesteuert werden.
	Für das Spiel werden zwei Aktoren benötigt:
	\begin{description}
		\item[Spieler]{ Die Person, welche das Spiel auf dem Handy installiert hat und spielt. }
		\item[Aufgabenstellerin]{ Die Person, welche neue Aufgaben für das Spiel zusammenstellt, bestehende editiert oder auch Aufgaben entfernt. Die Person benötigt auch eine Methode die Aufgaben zu aktualisieren. }
	\end{description}

	Liste aller Funktionen und eine Beschreibung für einen Überblick der Funktionalität, für weitere Informationen siehe Kapitel \ref{ssec:ucd} und \ref{sssec:spielmechanik}:
	\begin{description}
		\item[Spielstart Hilfsagent]{ Beim ersten Starten wird dem Nutzer geholfen sich in das Spiel einzufinden. Dies wird über wortlose Bilder gemacht, welche verschwinden, sobald der Nutzer die Steuerung probiert hat und eine Runde rennen gefahren ist. Der Hilfsagent kann auch abgebrochen werden. }
		\item[Level-Gruppe wechseln]{
			Der Spieler kann die Level-Gruppen rotieren, jede besitzt zwei Level als Qualifikationsrennen (normale Rennkarten) und einen Level für welchen ein gewisser Preis an gelösten Aufgaben gezahlt werden muss (Finalrennen).
			Jede Level-Gruppe ist in einem gewissen Stil durch thematisiert wobei die Karten für die Finalrennen etwas außergewöhnlicher sein sollen.
		}
		\item[Optionen ändern]{ Über einen speziell als Optionen gekennzeichneten \enquote{Level} gelangt der Nutzer auf einen gesonderten Bildschirm wo folgende Funktionen zur Verfügung stehen:
			\begin{description}
				\item[Lautstärke der Musik einstellen]{ Der Nutzer kann mittels Schieberegler die Lautstärke der Hintergrundmusik, im Spiel wie in den Menüs, einstellen. Die Lautstärke wird in Echtzeit verändert. }
				\item[Lautstärke der Soundeffekte einstellen]{ Der Nutzer kann mittels Schieberegler die Lautstärke von Soundeffekten anpassen. Wird der Schieberegler losgelassen, wird ein Effekt vorgespielt, um damit der Nutzer Feedback erhält. }
				\item[Hilfsagent erneut starten]{ Hierdurch wird der Hilfsagent erneut gestartet. }
				\item[<TODO:Andere Optionen?>]{ Beschreibung }
			\end{description}
		}
		\item[Level wählen]{ Der Spieler fährt mit dem Auto auf ein Level, darauf erscheint ein Auswahlbildschirm für folgende zwei Optionen. Falls der Level ein Finalrennen ist, muss der Spieler eine Teilnahmegebühr zahlen.
			\begin{description}
				\item[als Rennen gegen die Zeit starten]{ In diesem Spielmodus ist eine Zeit vorgegeben, innerhalb welcher der Spieler den Level bestehen soll. Ebenso sind zwei kleinere Zeiten vorgegeben welche den Spieler herausfordern sich mehr anzustrengen. }
				\item[als Rennen gegen Computergegner starten]{ In diesem Spielmodus fährt der Spieler gegen computergesteuerte Autos um die Wette.<TODO:Formulierung?> Die Auswertung funktioniert über das Übliche erster zweiter dritter Platz verfahren <TODO:Name?>. }
			\end{description}
			Gestartete Spiele können jederzeit abgebrochen, pausiert und neugestartet werden.
		}
		\item[Onlinespiel starten]{ Ebenso wie die Optionen ist dies ein speziell gekennzeichneter \enquote{Level} welcher es ermöglicht mit Zufälligen anderen ein Rennen zu fahren. }
		\item[Extra Aufgaben Lösen]{ Ein Icon am Rand des welches es ermöglicht Punkte durch Aufgabenlösen zu sammeln. }
		\item[Spiel Beenden]{ Beendet das Spiel. }
		\item[Aufgabenverwaltung durch Aufgabenstellerin]{
			Die Aufgabenstellerin kann folgende Aktionen machen:
			\begin{itemize}
				\item{ Die Aufgabenstellerin kann Aufgaben ansehen und manipulieren. }
				\item{ Die Aufgabenstellerin kann Aufgaben entfernen. }
				\item{ Die Aufgabenstellerin kann Aufgaben hinzufügen. }
				\item{ Die Aufgabenstellerin kann entweder \\
					die Aufgaben in allen Spielen aktualisieren \\
					oder \\
					Ein Spiel mit einem geändertem Aufgabensatz erzeugen. <TODO:absprechen>
				}
			\end{itemize}
		}
	\end{description}
	Alle Funktionalitäten lassen sich in einem Use Case Diagramm abbilden welches die Autoren bei der Entwicklung Beispielsweise durch vereinfachte und übersichtliche Koordination und Aufgabenaufteilung unterstützt.
	\label{ssec:ucd}\figur[\textwidth]{usecases.pdf}{Use Case Diagramm}

	\subsubsection{Einschränkungen}
		Zeitrennen und Rennen gegen Computergegner sollen auch ohne Internetverbindung verfügbar sein.
		Die Menüstruktur soll so flach wie möglich gehalten werden, keine verschachtelten Menübäume.
	\subsubsection{Nicht funktionale Anforderungen}
		\begin{itemize}
			\item{ Minimale Bildwiederholrate von 24 Bilder pro Sekunde. }
			\item{ Maximale Speicheranforderung von unter 40MB. }
			\item{ Spielmusik soll nicht stocken. }
			\item{ Das Spiel zu starten soll unter einer Sekunde dauern. }
			\item{ <TODO:Mehr?> }
		\end{itemize}

\subsection{Grundsätzliche Eigenschaften des Spiels}
	Im folgenden Kapitel sollen die Grundlagen des Spiels erarbeitet und abgewogen werden, sodass ein breites Spektrum an Regeln und Eigenschaften des Spiels erarbeitet werden kann. Diese Sammlung von Eigenschaften und Regeln kann im späteren Verlauf der Arbeit als Kriterienkatalog zur Auswahl einer Game-Engine verwendet werden. Dabei sollen zunächst für jede Kategorie verschiedene Möglichkeiten erklärt werden, um im Anschluss die für die Spielidee beste Möglichkeit auszuwählen.
	<TODO:überarbeiten und mit Grundlagen auf Linie bringen. (Interessant ist hier wo sich unser Ansatz von der Wissenschaft unterscheidet)>

	\subsubsection{Spielmechanik}\label{sssec:spielmechanik}
	In den Bereich der Spielmechanik fallen all jene Regeln und Eigenschaften, welche das Spielgeschehen (Die Spielerfahrung des Spielers) beeinflussen und diese ausmachen.

		\paragraph{Regeln für den Spielfortschritt:}
		Zum Durchspielen des Spiels muss der Spieler eine Reihe von Level-Paketen abschließen. Die genaue Anzahl an zu spielenden Paketen ist variabel und kann jederzeit um ein zusätzliches Paket erweitert oder gekürzt werden. Die zu treffende Entscheidung an dieser Stelle ist, ob der Spieler die einzelnen Level-Pakete bereits von Beginn an auswählen kann oder sich diese im Laufe des Spiels freispielen muss. Ein Freispielen der einzelnen Level-Pakete gibt die Möglichkeit, einen Schwierigkeitsgrad einzuführen. Somit können zum Beispiel zuerst alle Pakete mit Schwierigkeitsgrad \enquote{Einfach} verfügbar sein. Mit erfolgreichem Abschluss aller einfachen Pakete würden dann die Pakete mit Schwierigkeit \enquote{Mittel} freigeschaltet.
		Die vollständige Verfügbarkeit aller Level-Pakete von Beginn an, hat für den Spieler den Vorteil der freien Wahl. Allerdings ist eine Umsetzung von unterschiedlichen Schwierigkeitsgraden unter Umständen nicht möglich. Durch sofortige vollständige Verfügbarkeit bleibt eine Lernkurve aus und der Spieler kann schnell die Motivation am Spiel verlieren.
		Aus den genannten Punkten ergibt sich die Entscheidung für das Freispielen. Somit ist eine Lernkurve gewährleistet. Für das Abschließen des Spiels (durch die Absolvierung des Final-Levels?) muss der Spieler alle vorangegangenen Level erfolgreich gespielt haben. Dies gewährleistet dem Spieler einen \enquote{roten Faden durch das Spiel} und ermöglicht einen linearen Spielfortschritt.

		\paragraph{Regeln innerhalb der Level-Pakete:}
		\footnote{<TODO:Berkling "Woher kommt die Idee?">}%
		Ein Level-Paket besteht aus insgesamt drei Rennen. Diese sind aufgeteilt in zwei Qualifikationsrennen und ein Finalrennen. Für den Fortschritt innerhalb eines Level-Pakets wäre es möglich, entweder eines der Qualifikationsrennen abzuschließen um das Finalrennen freizuschalten oder beide Qualifikationsrennen in entweder festgelegter oder beliebiger Reihenfolge abzuschließen um das Finalrennen verfügbar zu machen.
		Für den Umfang des Spiels wurde eine Größe von drei Rennen pro Level-Paket festgelegt. Der Abschluss eines einzelnen Qualifikaionsrennens für die Freischaltung des Finalrennens ist zwar grundsätzlich möglich, führt jedoch zu einer Minderung des Spielumfangs. Um dies zu vermeiden, besteht zumindest die Anforderung, beide Qualifikationsrennen abschließen zu müssen. Die Reihenfolge der Qualifikationsrennen spielt im Bereich der Entwicklung keine Rolle. Dieser Aspekt ist insofern zu vernachlässigen. Für den Spieler kann eine sofortige Verfügbarkeit beider Qualifikationsrennen ein gewisses \enquote{Freiheitsgefühl} auslösen <TODO:quelle>. Aus diesem Grund werden mit Beginn eines Level-Pakets beide Qualifikationsrennen sofort verfügbar.

		\paragraph{Gewinnbedingungen eines Rennens:}
		Zur Durchführung der Einzelspieler Rennen stehen grundsätzlich zwei Möglichkeiten zur Auswahl. Die erste Möglichkeit ist, dass der Spieler gegen computergesteuerte Rennteilnehmer (im folgenden Bots genannt) antritt. Zur Realisierung dieser Alternative ist die Entwicklung einer sogenannten KI <TODO: Erklärung?> nötig. Weiterhin werden dadurch Punkte, wie beispielsweise die Kollision mit anderen Rennteilnehmern notwendig. Um ein Rennen zu gewinnen, muss der Spieler als erster die Ziellinie überqueren.
		Die zweite Möglichkeit wäre die Realisierung der Rennen über eine vorgegebene Zeit. Das so entstehende Zeitrennen muss in der vorgegeben Zeit absolviert werden. Überschüssige Zeit verfällt und kann für folgende Rennen nicht verwendet werden.
		Unter Beachtung der erwähnten Punkte und mit Blick auf den gesetzten Zeitrahmen wird von der Entwicklung einer KI abgesehen. Somit bleibt als für die Realisierung des Rennens das Zeitrennen übrig.

		\paragraph{Zusatzaufgaben während eines Rennens:}
		Der Spieler soll während eines Rennens Zusatzaufgaben erhalten. Eine Möglichkeit für die Umsetzung dieser Zusatzaufgaben konnte während des Brainstormings erarbeitet werden. Diese hat sich im Laufe der Zeit weiterentwickelt und wurde verfeinert. Auf eine ausführliche Erläuterung des exakten Werdegangs wird an dieser Stelle verzichtet.
		Bei der erarbeiteten Umsetzung bekommt der Spieler während des Rennens eine Aufgabe auf einer freien Stelle des Bildschirms angezeigt. Nach einer vorgegebenen Zeit X (diese variiert anhand der Schwierigkeit der Aufgabe oder dem Verlauf der Strecke) verschwindet diese Aufgabe und die Rennstrecke führt zu einer Weggabelung. Beide Wege sind mit Schildern versehen, auf denen jeweils mögliche Lösungen der Aufgabe angezeigt werden. Auf einem der Schilder steht somit die korrekte Lösung, auf dem anderen Schild stehen falsche Lösungen. Der Spieler muss nun sein Auto auf den richtigen Weg lenken.
		Eine weitere Möglichkeit wäre es, das Rennen an einem bestimmten Punkt zu pausieren und eine Aufgabe anzuzeigen. Die gemessene Zeit des Zeitrennens läuft dabei weiter. Erst mit Eingabe der korrekten Lösung wird das Rennen fortgesetzt und der Spieler darf weiterfahren. Dies würde jedoch den Spielfluss maßgeblich unterbrechen und somit zu einem Verlust an Spielspaß führen. Somit fällt die Wahl für die Zusatzaufgaben auf die Weggabelungen im Rennen.

		\paragraph{Belohnungen / Strafen bei Zusatzaufgaben:}
		\marginpar{\tiny\raggedright\sloppy<TODO:Berkling "Was sind psychologische vor Nachteile bei Bestrafung und Belohnung? Wissen sie das? Gibt es darüber Literatur? Würde mich interessieren, auch ein Experiment, wie Leute auf die beiden Varianten reagieren wäre interessant\dots>"}
		Da die Zusatzaufgaben fester Bestandteil des Spiels und zur Übermittlung des Lernerfolgs nötig sind, muss ein Absolvieren bzw. Nichtabsolvieren der Aufgaben auch Auswirkungen auf das Spiel haben. Für die Zusatzaufgaben innerhalb der Rennen ergeben sich zwei mögliche Lösungen.
		Die erste Möglichkeit zur Lösung dieses Problems ist, die Weggabelung der Strecke in zwei Streckenabschnitte mit identischer Länge zu führen und die Belohnungen / Strafen über die benötigte Zeit zu regeln. Mit der bestrafenden Methodik würde ein Wählen des falschen Wegs zu einem Abzug in der verbleibenden Zeit führen, was ein erfolgreiches Absolvieren des Rennens maßgeblich erschwert. Ein Wählen des richtigen Wegs hätte in diesem Fall keine Auswirkungen auf den Spieler.
		Bei der belohnenden Methodik würde ein Wählen des richtigen Wegs einen Zeitbonus und somit eine Erleichterung für den Spieler bedeuten und der falsche Weg hätte keine Auswirkungen. Eine Verbindung beider Methoden ist selbstverständlich möglich.
		Die zweite Möglichkeit besteht darin, an den Weggabelungen unterschiedliche Streckenabschnitte zu positionieren. Somit würde die falsche Wahl des Wegs zu einem längeren Streckenabschnitt führen, was den Spieler wichtige Zeit kostet.
		<TODO: Methode wählen und Abschluss formulieren>

		\paragraph{Zusatzaufgaben im Multiplayer:}
		Für die Implementierung der Zusatzaufgaben im Mehrspieler-Modus werden zunächst die Zusatzaufgaben aus dem Einzelspieler-Modus übernommen. Das bedeutet, dass auch im Mehrspieler-Modus Weggabelungen existieren werden, die Einfluss auf das Rennergebnis haben. Diese werden für jeden Spieler einzeln berechnet, sodass ein Abschauen von dem Vordermann nicht möglich ist. Zusätzlich zu diesen Aufgaben sollen weitere Aufgaben zu Beginn des Rennens eingeführt werden. Dabei besteht die Möglichkeit, einen \enquote{Start-Timer} einzuführen. Dieser würde eine festgelegte Anzahl an Sekunden herunter zählen, nach welchen das Rennen für jeden Spieler startet. Diejenigen Spieler, die allerdings in einer kürzeren Zeit eine Aufgabe lösen können, dürfen früher starten und erhalten somit einen Vorteil fürs Rennen.
		Durch diese Lösung ist gewährleistet, dass jeder Spieler starten kann, ungeachtet ob die Aufgabe lösbar ist. Durch eine exakte Optimierung dieser Zeitspanne kann gewährleistet werden, dass sowohl Können beim Fahren sowie das schnelle Lösen von Aufgaben zum Sieg beitragen.

		\paragraph{Umsetzung einer Ingame-Währung:}
		Um dem Spieler die Möglichkeit zu geben, Boni und optische Upgrades zu erwerben, soll dem Spiel eine Ingame-Währung beigefügt werden. Potentiell sind drei Möglichkeiten für den Erwerb dieser Währung denkbar.
		\begin{enumerate}
			\item{ Erwerb der Währung über reale Zahlungen. Dies ist die Einnahmequelle für Spiele nach dem \enquote{freemium} Modell\footcite[Seite 8]{freemium}. }
			\item{ Verdienen der Währung über das Gewinnen von Rennen. Diese Siegerprämie kann als zusätzliche Motivation für den Spieler dienen, Level-Pakete zu Wiederholen um zusätzliche Währung zu verdienen. }
			\item{ Lösen von zusätzlichen Aufgaben. }
		\end{enumerate}
		Die erste genannte Möglichkeit der Bezahlung mit realen Transaktionen wird an dieser Stelle verworfen, da das Spiel nicht gewinnorientiert ist. Außerdem kann somit das Risiko von versehentlichen Belastungen der elterlichen Kreditkarten verhindert werden.
		Die beiden übrigen Möglichkeiten bieten dem Spieler Motivation und vergrößern den potentiellen Lernerfolg. Somit wird für die Umsetzung des Spiels eine Kombination aus Siegerprämie und (täglich beschränkten) Zusatzaufgaben implementiert.

		\paragraph{Verlassen des vorgegebenen Wegs:}
		Jedes Rennen verfügt über einen vorgegeben Weg, welcher von Start bis zum Ziel führt. Dieser vorgegebene Weg verbreitert sich um breite Randstreifen\footnote{<Anmerkung:Randstreifen hauptsächlich aus visuellen gründen? ich würde die Strecke so gestalten dass es kein vom weg abkommen gibt ;)>} zur fertigen Strecke. Ein verlassen der Straße muss gewisser Weise Auswirkungen auf das Fahrzeug des Spielers haben, um das Fahren auf der Strecke zu begünstigen. Denkbar wäre, das Fahrzeug des Spielers merklich zu verlangsamen, falls dieser die Strecke verlässt. Zusätzlich sollte dem Spieler eine Warnung zum Zurückkehren auf die Strecke angezeigt werden. Weiterhin kann das Fahrzeug des Spielers automatisiert auf die Strecke zurückgesetzt werden, falls es den vorgegebenen Weg für eine gewisse Zeit X verlässt. Gleiches gilt, falls ein Spieler sein Fahrzeug wendet und die Strecke in entgegengesetzter Richtung fährt. Dabei soll das Fahrzeug jedoch nicht verlangsamt werden. Für die Implementierung wird eine Kombination aus Anzeigen einer Warnung und Verlangsamung des Fahrzeugs gewählt. Somit hat ein Verlassen der Strecke einen merklichen Nachteil im Zeitrennen, da die Verlangsamung viel Zeit kostet. Auf ein automatisches Zurücksetzen wird verzichtet, damit der Spieler aus seinen Fehlern lernt, da er selbstständig auf die Strecke zurückfahren muss.

		\paragraph{Hindernisse auf der Strecke:}
		Um den Schwierigkeitsgrad der Rennen zusätzlich zu erhöhen, besteht die Möglichkeit auf Hindernisse zurückzugreifen, welche sich auf der Strecke befinden. Für eine Kollision mit einem solchen Hindernis muss geklärt werden, welches Verhalten das Fahrzeug zeigt. Die beiden naheliegenden Möglichkeiten für eine Kollisionsreaktion sind das vollständige Stoppen (Abprallen) und das Verlangsamen des Fahrzeugs.
		Ein vollständiges Stoppen des Fahrzeugs würde eine Art Unterbrechung des Spielflusses bedeuten. Da dies, wie bereits erwähnt, nach Möglichkeit vermieden werden soll, scheidet diese Möglichkeit aus. Ein Verlangsamen des Fahrzeugs kann entweder als Verlangsamung um einen Prozentsatz X (basierend auf der aktuellen Geschwindigkeit) oder mittels einer Verlangsamung der Geschwindigkeit auf einen Festen Wert umgesetzt werden. <TODO: Entscheiden und Erklären>

	\subsubsection{Sonstige Eigenschaften des Spiels}
	Im Bereich der sonstigen Eigenschaften sollen Eigenschaften betrachtet werden, die nicht direkt die Spielmechanik betreffen. Dazu gehören unter Anderem Kameraeinstellungen, das Optionsmenü und ähnliches.

		\paragraph{Kaufbares für Ingame-Währung:}
		Für die Spezifizierung der kaufbaren Dinge im Spiel<TODO:grammatik?>, können zwei Kategorien festgelegt werden. Die erste Kategorie sind optische Upgrades. Diese beziehen sich vor Allem auf die verwendbaren Fahrzeuge und Gegenstände. So kann der Spieler sein von Beginn an verfügbares Auto beispielsweise gegen einen Ball, einen Läufer, einen Rennwagen oder ähnliches austauschen. Zu beachten ist dabei, dass alle Fahrzeuge die gleiche Geschwindigkeit haben und somit kein Vorteil für den Mulitplayer erkauft werden kann. Die zweite Kategorie sind Boni, welche auf den Einzelspieler-Modus bezogen funktionieren. Dabei bekommt der Spieler die Möglichkeit, einen Zeitbonus (von beispielsweise 10 Sekunden) zu erkaufen, um das nächste zu erleichtern. Diese Möglichkeit besteht sowohl für die Qualifikations-, sowie für die Finalrennen und ist im Multiplayer nicht verfügbar.

		\paragraph{Heads-Up Display und Steuerung:}
		Weiterhin besitzen die grundsätzliche Benutzeroberfläche, sowie die Steuerung Relevanz. Dabei soll der Fokus an dieser Stelle weniger auf dem Design der Oberfläche oder des Menüs, sondern mehr auf einer Beschreibung der vorhandenen Knöpfe und Funktionen liegen.
		Die Steuerung der Fahrtrichtung des Fahrzeugs kann über zwei Wege realisiert werden.<TODO:Referenz human computer interaction> Der erste Weg wäre die Verwendung des Neigungssensors des Nutzergeräts. Dabei müsste der Spieler sein Gerät ähnlich einem Lenkrad in die entsprechende Richtung neigen, um eine Reaktion des Fahrzeugs hervorzurufen. Die zweite Möglichkeit wäre die Verwendung eines \enquote{Lenkrads}, wobei der Spieler mittels Bewegungen auf dem Touchscreen das Fahrzeug steuern kann. Da ein Steuern des Fahrzeugs mittels des Neigungssensors und ein gleichzeitiges Lesen und Lösen der auf dem Bildschirm angezeigten Aufgaben als zu schwierig erachtet wird, soll die Steuerung über Touchscreen-Eingaben realisiert werden. Hierbei eignet sich das für Mobile-Spiele übliche stufenlose Steuerkreuz am Besten. Somit ist dieses Steuerkreuz eine nötige Komponente auf dem Bildschirm.
		Für die Steuerung der Geschwindigkeit konnte sich im Rahmen des Brainstormings lediglich eine Möglichkeit durchsetzen. Hierbei sollen zwei \enquote{Pedale} verwendet werden, um das Fahrzeug zu beschleunigen bzw. zu bremsen. Um die Geschwindigkeit des Fahrzeugs zu halten, muss das \enquote{Gaspedal} durchgehend gedrückt werden. Ein Loslassen des Gaspedals führt zu einer stetigen Verringerung der Geschwindigkeit. Ein Betätigen des Bremspedals führt zu einer deutlich schnelleren Verringerung der Geschwindigkeit. Wird das Bremspedal nach Stillstand des Fahrzeugs erneut betätigt, fährt dieses mit verringerter Geschwindigkeit rückwärts. Aufgrund dieser Beschreibung ergibt sich, dass zwei Pedale als Komponenten auf dem Bildschirm benötigt werden.
		<TODO: Minikarte?>
		<Anmerkung: mögliche inputs (touch,beschleunigung,magnet/ausrichtung...) auflisten, darauf bereits genutzte steuerungen auflisten und dann herraussuchen (Quellen ;) ) >

		\paragraph{Kameraperspektive:}
		Grundsätzlich sind für Rennspiele drei Kameraperspektiven denkbar. Diese sind:
		\begin{itemize}
			\item{ Kamera zeigt das innere des Fahrzeugs und schaut durch die Frontscheibe. }
			\item{ Die Kamera befindet sich hinter dem Fahrzeug, zeigt somit das Fahrzeug und die Strecke. }
			\item{ Die Kamera befindet sich über der Strecke und dem Fahrzeug. }
		\end{itemize}
		Da das Spiel für Mobile-Endgeräte entwickelt wird, ist eine Ansicht innerhalb des Autos nicht gut geeignet, da die Steuerungselemente sowie die Finger des Spielers die Sicht zu sehr einschränken. Somit fällt diese Option weg.
		Befindet sich die Kamera hinter dem Fahrzeug, ist die Sicht auf die kommende Strecke bei besonders kurvigen Strecken stark eingeschränkt. Da ein frühzeitiges Sehen des Streckenverlaufs hilft, gleichzeitig die gestellten Aufgaben zu lösen und das Fahrzeug zu steuern, fällt die Entscheidung an dieser Stelle auf die Ansicht von oben.
		Der Spieler kann so die Strecke früh erkennen und sich auf die kommenden Kurven einstellen, ohne dabei die Konzentration für die gestellten Aufgaben zu erhalten. Der Kamerawinkel wird leicht geneigt sein, um die Modelle plastischer zu gestalten und 3D-Modelle verwenden zu können.

		\paragraph{Musik und Sound:}
		Um den Umfang des Spiels zu vervollständigen, muss an dieser Stelle noch die Musik und Sound im Allgemeinen betrachtet werden. Im Brainstorming wurde entschieden, dass das Spiel eine Hintergrundmusik erhalten soll. Zudem benötigen Kollisionen, die Fahrzeuge und bestimmte Momente im Spiel (Start, Ziel, Lösen einer Aufgabe) ein spezielles Sound-Pack.
		Dazu soll der Spieler die Möglichkeit bekommen, im Spiel Einstellungen an der Tonausgabe vorzunehmen. Dazu gehört vor Allem die Lautstärkeeinstellung sowie die Möglichkeit, den Sound vollständig zu deaktivieren. Dabei soll zwischen der Spielmusik, den Fahrzeuggeräuschen und den Umgebungsgeräuschen unterschieden werden.

\subsection{Entwicklung Play-Persona}
	Wie in Kapitel \ref{ssec:persona} beschrieben, werden für die Entwicklung von Play-Personas die Spielmechaniken und Fähigkeiten der Spieler benötigt. Im Falle des zu entwickelnden Rennspiels ergeben sich zwei Kernkompetenzen, nach welchen die Spieler unterschieden werden können. Diese sind:
	\begin{description}
		\item[Rennen Fahren]{Die Kompetenz des Rennfahrens tätigt eine Aussage darüber, wie gut bzw. schlecht ein Spieler beim Steuern des Fahrzeugs ist. Diese Fähigkeit kann durch Rundenzeiten bzw. durch Platzierungen in Multiplayer-Rennen gemessen werden.}
		\item[Aufgabenösen]{Durch die Kompetenz des Aufgabenlösens wird abgedeckt, ob ein Spieler die gestellten Aufgaben richtig lösen kann, wie viel Prozent der Aufgaben korrekt gelöst wurden und wie viel Zeit dafür benötigt wurde.}
	\end{description}
	Daraus ergeben sich folgende 4 Personas:

	\begin{tabl}{lcc}{Play-Persona Matrix}
		\toprule
			Name/Eigenschaft & Rennen Fahren & Aufgabenlösen \\
		\midrule
			Anfänger & - & - \\
			Rennfahrer & + & - \\
			Lehrer & - & + \\
			Experte & + & + \\
		\bottomrule
	\end{tabl}

	\begin{description}
		\item[Die Grundschule Helmsheim]\hfill\\
		Helmsheim ist eine Kleinstadt in Deutschland mit knapp 6000 Einwohnern. Die Grundschule Helmsheim liegt am Stadtrand und ist eine der beiden Grundschulen in der Stadt. Die Grundschule ist recht fortschrittlich, da sie zwar staatlich betrieben, jedoch durch ein ortsansässiges IT-Unternehmen gesponsert wird. Durch dieses Sponsoring hat die Schule die Möglichkeit, neue Technologien einzusetzen und so beispielsweise educational games für den Unterricht zu verwenden.
		\item[Florian Klein: Der Anfänger]<TODO: Bild!>\hfill
		\begin{description}
			\item[Familiäres Umfeld]{Florian ist sieben Jahre alt und lebt mit seinen Eltern und zwei Geschwistern am Stadtrand. Seine Geschwister sind Andy(10) und Laura(4). Florians Vater arbeitet bei einem ortsansässigen IT Unternehmen, seine Mutter ist Hausfrau. Gelegentlich passt Florian auf seine kleine Schwester auf und macht regelmäßig seine Hausaufgaben aus der Schule, weitere Aufgaben im Haushalt hat er jedoch nicht.}
			\item[Schulisches Umfeld]{Florian besucht, genauso wie sein Bruder, die Helmsheimer Grundschule. Dort ist er aktuell in der zweiten Klasse. Florian befindet sich im schulischen Durchschnitt und findet keinen besonderen Spaß am Lernen.}
			\item[Interessen und Hobbys]{Florian ist ein begeisterter Auto-Fan. Er spielt gerne mit Spielzeugautos, sieht Autosendungen im Fernsehen und will später unbedingt Rennfahrer werden. Er hat sehr viel Spaß daran, Rennen zu fahren und hat bei seinem Bruder ein Auto-Rennspiel gesehen, dass er nun auch spielt. Außerdem spielt er in seiner Freizeit Fußball. Er hat eine Abneigung gegen kleine Autos (Smarts) und Mädchen.}
			\item[Motivation]{Die Motivation Florians liegt im Ehrgeiz, seinen Bruder zu schlagen. Er möchte bessere Rundenzeiten im Rennen erreichen und so lange immer besser im Spiel werden, bis er seinen Bruder im Multiplayer besiegen kann.}
			\item{\enquote{Mama, Mama, guck mal ein rotes Rennauto!}}
		\end{description}
		\item[Joachim Wolf: Der Lehrer]<TODO: Bild!>\hfill
		\begin{description}
			\item[Familiäres Umfeld]{Joachim ist 58 Jahre alt und lebt mit seiner Frau Ulrike(56) in Helmsheim. Seine Tochter Natalie(25) wohnt nicht mehr bei ihnen, da sie für ihr Studium umgezogen ist. Joachim hat noch keine Enkelkinder. Während er nicht in der Grundschule ist, muss er den Unterricht für kommende Tage vorbereiten, Klassenarbeiten korrigieren und ab und an ein Seminar für Lehrer besuchen.}
			\item[Schulisches Umfeld]{Joachim hat Lehramt in den Fächern Geschichte und Geographie studiert, beschloss jedoch mit 41 das Gymnasium zu verlassen und an der Grundschule weiter zu arbeiten. Seitdem arbeitet er bei der Grundschule Helmsheim.}
			\item[Interessen und Hobbys]{In seiner Freizeit liest Joachim sehr viel. Vor allem Romane und Krimis interessieren ihn besonders. Er fährt viel Fahrrad und nutzt jede mögliche Gelegenheit, um im nahe gelegenen Stadtpark Schach mit seinen Freunden zu spielen. Zudem gibt er aus Hobby Nachhilfe und beschafft sich somit einen kleinen Nebenverdienst.}
			\item[Motivation]{Die größte Motivation für Joachim ist es, den Kindern möglichst viel beizubringen. Dafür möchte er für die Kinder ein besseres Lernumfeld schaffen, damit die Kinder Spaß am Lernen haben. Um dies zu erreichen, möchte er mit seinen Klassen das spielerische Lernen ausprobieren. Er hofft, dass die Kinder freiwillig lernen (spielen), wenn sie auch Spaß daran haben.}
			\item{\enquote{Wissen sollte Spaß machen, um es in jungen Jahren nach vorn zu bringen.}}
		\end{description}
		\item[Laura Dietz: Die Rennfahrerin]<TODO: Bild>\hfill
		\begin{description}
			\item[Familiäres Umfeld]{Laura ist 12 Jahre alt und lebt zusammen mit ihrer Mutter im Nachbarhaus von Joachim. Lauras Eltern leben getrennt. Laura ist ein Einzelkind und muss, da ihre Mutter arbeiten geht, regelmäßig im Haushalt helfen und mit dem Hund gehen. Zudem wurde sie von ihrer Mutter in der Nachhilfe von Joachim angemeldet. Sie geht dort hin, um ein gutes Verhältnis zu ihrer Mutter zu wahren, auch wenn sie die Zeit viel lieber mit ihrem dreizehnjährigen Freund und ihren Spielen verbringen würde.}
			\item[Schulisches Umfeld]{Sie besucht die fünfte Klasse einer Realschule in Helmsheim, interessiert sich allerdings nicht sonderlich fürs Lernen. Da sie die Vorschule besucht hat, gehört sie zu den Ältesten in ihrer Klasse. Da ihre Mutter schon früher nicht viel Zeit mit Laura verbringen konnte, hat Laura viel gespielt. Sie hat über diese Spiele viel gelernt und hat deshalb das grundsätzliche Desinteresse am Lernen.}
			\item[Interessen und Hobbys]{Laura verbringt gerne viel Zeit mit ihrem Handy oder am Computer. Sie nutzt beides hauptsächlich zum Spielen und zum Chatten mit ihrem Freund. Sie möchte endlich eine Spielekonsole haben, um ihre Leidenschaft zum Spielen noch weiter verfolgen zu können. Wenn sie nicht gerade spielt, trifft sie sich mit ihrem Freund.}
			\item[Motivation]{Laura wurde von Joachim gebeten, das Lernspiel auszuprobieren, da Joachim weiß, wie gern Laura spielt. Sie ist wenig motiviert, viel zu lernen, möchte jedoch Joachim den Gefallen tun, da ihre Mutter ja für die Nachhilfe mit Joachim zahlt.}
			\item{\enquote{Nur noch fünf Minuten spielen, Mama?}}
		\end{description}
		\item[Andy Klein: Der Experte]<TODO: Bild>\hfill
		\begin{description}
			\item[Familiäres Umfeld]{Andy ist zehn Jahre alt und Bruder von Florian. Zu seinen Aufgaben gehören, genau wie bei Florian das Lernen und das Machen der Hausaufgaben, jedoch hat Andy daran deutlich mehr Spaß als Florian.}
			\item[Schulisches Umfeld]{Er besucht die vierte Klasse der Helmsheimer Grundschule und ist ein motivierter Schüler. Er verfolgt zur Zeit zwei schulische Ziele: Zum einen möchte er zum Klassenbesten in Mathematik werden, zum anderen bereitet er sich schon jetzt auf die weiterführende Schule vor, um einen möglichst guten Abschluss zu erreichen. }
			\item[Interessen und Hobbys]{Andy ist wissenschaftlich sehr interessiert. Besonders angetan haben es ihm die Physik und die Chemie. Er spielt gern mit seinem Chemiebaukasten, sowie andere Physik- und Chemie-, sowie Geschicklichkeitsspiele. Ab und an spielt er nebenbei allerdings auch Action- bzw. Rennspiele. Er geht einmal wöchentlich nach der Schule in einen Physik-Club for Kids, wo kleinere Experimente für die Kinder durchgeführt werden.}
			\item[Motivation]{Seine Klasse wird in Mathematik von Joachim unterrichtet, daher sollen alle Kinder der Klasse das Lernspiel ausprobieren. Da er Klassenbester in Mathematik werden möchte, geht er mit besonders viel Enthusiasmus an das Spiel heran, um schnell möglichst hohe Erfolge zu erzielen. Außerdem macht ihm das Spielen Spaß, was zusätzlich zum Weiterspielen motiviert.}
			\item{\enquote{Mit Spielen lernen, macht ja schon Spaß!}}
		\end{description}
	\end{description}

\subsection{Gamification Modell}\label{ssec:gamification-modell}
	Das Spiel soll durch einen gewissen Schwierigkeitsgrad Herausfordernd aber trotzdem Anfängerfreundlich sein. Ein Punktesystem und freiwilliges bearbeiten von Aufgaben ermöglicht Spielern zusätzliche Funktionalitäten zu erkunden und dadurch fürs lernen belohnt zu werden.
	Durch Hinweise auf diese Zusatzfunktionen und Extras soll der Spieler kontinuierlich daran erinnert werden, dass er mehr erkunden könnte, dabei sollen diese Hinweise nicht störend oder aufdringlich sein.
	Die verschiedenen Personas sollen wie folgt angesprochen werden:
	\begin{description}
		\item[Florian Klein]{
			Durch einen leichten Einstieg und viel verrückten Rennspaß kann der Anfänger viel lernen. Durch eine einfache Steuerung, viele Extras und auch Freunde bleibt er längere Zeit am Spiel und somit am Lernen begeistert.
		}
		\item[Joachim Wolf]{
			Der Lehrer hat die Möglichkeit durch Aufgaben und nicht Reaktionszeit oder Renntalent zu glänzen. So kann auch der Lehrer Beispielsweise mit seinen Schülern mithalten und sich in Geschicklichkeit und Reaktionszeit üben.
		}
		\item[Laura Dietz]{
			Die Rennfahrerin stört sich wohl am meisten an den Aufgaben, kann aber im Rennen durch geschicktes Fahren wieder aufholen. Sie hat Spaß an den Extras solange sie dafür nicht zu viel arbeiten muss.
			Durch anspruchsvollere Finalrennen soll sie überzeugt werden die Aufgaben zu lösen.
		}
		\item[Andy Klein]{
			Der Experte erkundet das Spiel schnell und hat später durch den Mehrspielermodus die Möglichkeit sich mit anderen zu vergleichen und zusammen Spaß zu haben. So bleibt auch für den Experten das Spiel interessant.
		}
	\end{description}
	<TODO:nochmal zusammenfassen und ein zwei allgemeine Sätze>

\subsection{Lernziele Umsetzung}
	Die Lernstrategie ist Wiederholung und spielerisches einsetzen von Aufgaben. Dazu sollen die Finalrunden etwas Herausforderndes und Besonderes sein. Nutzer sollen durch Gamification motiviert werden mehr und länger zu spielen (siehe \ref{ssec:gamification-modell}).
	Die Lernziele wurden bereits in Kapitel \ref{ssec:lern-def} definiert.
	Wie schon in Kapitel \ref{ssec:spielidee}, \ref{ssec:idee} und \ref{ssec:requirements} beschrieben gibt es mehrere Möglichkeiten, wie Aufgaben im Spiel eingebaut werden können. Wichtig ist, dass die Aufgaben nur für Finalrunden-Teilnahme unbedingt gelöst werden müssen. Das Lösen von Aufgaben ist in allen anderen Fällen zwanglos, wird aber psychologisch gefördert (siehe \ref{ssec:psycho-grundlagen}).
	Die Idee ein Punktekonto für Aufgaben zu erstellen ist von dem \enquote{pay to win}\footnote{<TODO:Quelle>} Modell der Monetization\footcite{freemium} abgeleitet und \enquote{learn to win} getauft.
	\begin{description}
		\item[Aufgabe zum Start des Rennens]{
			Eine Aufgabe um \enquote{den Motor zu starten} nimmt möglicherweise etwas Spannung aus dem Start, passt aber ins Spielkonzept und stört deshalb möglicherweise weniger.
		}
		\item[Aufgaben in Form von Weggabelungen]{
			Am oberem Rand des Bildschirms taucht die Frage auf und kurz später eine Weggabelung, beispielsweise ein Tor mit zwei Durchgängen. Durch so eine spielerische Nutzung von Aufgaben kann ein Spieler motiviert werden zu lernen.
		}
		\item[Aufgaben für Punkte]{
			Punkte können später in normalen Rennen für witzige aber auch hilfreiche Dinge eingesetzt werden und haben so einen Sinn.
			Hilfreiche Dinge sind möglicherweise ein Vorsprung, Zeitbonus, optische Verbesserungen, besondere Autos,
		}
		\item[Aufgabe für Teilnahme an Finalrunde]{
			Eine Teilnahmegebühr für Finalrunden bringt nicht nur den Spieler zum Lernen, sondern setzt die Finalrunden von den normalen Level ab und macht sie so zu etwas besonderem. Durch ein besonders hervorgehobenen Leveleingang soll der Spieler darauf aufmerksam gemacht werden.
		}
	\end{description}
	Im Optimalfall werden die verschiedenen Alternativen durch Spieltests empirisch belegt.\footnote{<TODO:spieletests oder so>}

\subsection{Game-Engine Evaluation}\label{ssec:engineeval}
	Für die Evaluation, welche der betrachteten Engines am geeignetsten für die Umsetzung des Spiels ist, sollen zunächst die Kriterien erneut erwähnt und genauer beschrieben werden.
	\begin{description}
			\item[Androidfähig]{Dieses Kriterium beschreibt, ob die jeweilige Engine den Entwickler bei der Erstellung von Android-Anwendungen zu unterstützen. Die Bestwertung wird erziehlt, wenn die Engine standardmäßig dazu in der Lage ist, Android Anwendungen zu entwickeln und zu deployen. Benötigt die Engine eine separate Erweiterung für die Entwicklung von Android-Anwendungen, so erhält sie weniger Punkte. Ist die Engine nicht in der Lage, Android-Anwendungen zu entwickeln und zu deployen, werden keine Punkte vergeben.}
			\item[Lizenzen]{Mit dem Lizenz-Kriterium werden die verschiedenen Geschäftsmodelle beschrieben und kategorisiert. Eine Open-Source Engine erhält dabei die meisten Punkte. Gefolgt wird dies von kostenlosen Engines. Den Abschluss in dieser Evaluation bilden kostenpflichtige Engines. Da die Anwendung keine Gewinne einbringen soll, sondern kostenlos zur Verfügung gestellt wird und keine Ingame-Echtgeld-Transaktionen besitzt, spielt die Gewinnbeteiligung der Herausgeber der Game-Engines für diese Evaluation keine Rolle.}
			\item[Networking]{Das Networking beschreibt, in wechlcher Form und wie sehr eine Engine die Entwickler dabei unterstützt, einen Multiplayer-Modus für ein Spiel zu entwickeln. Relevant sind dabei sowohl die Möglichkeiten existierender Frameworks, die Bereitstellung von Matchmaking-Servern und die generelle Unterstützung bei der Entwicklung des Multiplayers. Engines, zu deren Funktionsumfang ein Tool für die Entwicklung von Mehrspieler-Modi gehört, belegen in diesem Fall die höchste Bewertung. Engines, welche dem Entwickler durch ihre Frameworks die Möglichkeit zur Mehrspieler-Entwicklung geben, belegen die zweithöchste Bewertung. Gefolgt werden sie von Engines, welche zwar für Mehrspieler-Entwicklung geeignet sind, dies jedoch nicht unmittelbar unterstützen. Abschließend folgen Engines, welche nicht für die Entwicklung eines Multiplayers geeignet sind.}
			\item[Funktionsumfang]{Durch den Funktionsumfang kann eine Aussage darüber getroffen werden, welche Möglichkeiten ein Entwickler bekommt, wenn er die jeweilige Game-Engine einsetzt. Game-Engines mit einem großen Funktionsumfang schneiden in dieser Kategorie besonders gut ab, kleinere Game-Engines mit geringerem Funktionsumfang erhalten eine schlechte Bewertung. Dieses Kriterium beschreibt somit lediglich den faktischen Umfang an Funktionen und trifft keine Aussage über die Lernkurve oder die Übersichtlichkeit einer Engine. Auch werden Zusatzfunktionen, welche durch Erweiterungen erreicht werden können, nicht betrachtet.}
			\item[Intuitivität]{Die Intuitivität beschreibt den Einstieg in eine Engine und die Lernkurve um den Funktionsumfang der Engine zu meistern. Engines, welche dem Nutzer einen klaren Leitfaden für den Einstieg geben, werden hierbei besser bewertet, als Engines die dies nicht können. Zudem werden Engines, welche dem Nutzer Beispiele, Tutorials und weiteres im Funktionsumfang mitliefern, besser bewertet. Die grundsätzliche Übersichtlichkeit der Engine und der Editoren wirkt sich auf die Intuitivität aus. Engines mit einem komplizierten / undurchsichtigen Aufbau erhalten somit eine schlechtere Bewertung. Die Einschätzung in diesem Kriterium basiert ausschließlich auf persönlichen Eindrücken. Für die Evaluation wurden alle genannten Engines installiert und betrachtet.}
			\item[Programmiersprache]{Die Programmiersprache ist das Kriterium mit dem geringsten Einfluss auf das Ergebnis dieser Evaluation. Die ausgewählten Engines verwenden entweder C\# oder C++ für die Implementierung der Spiele. Für Java-Entwickler ist der Einstieg in C\# durch die Ähnlichkeiten zu Java zwar erleichtert, jedoch bietet C++ dem Entwickler mehr Freiheiten. Da die Fähigkeiten der Projektteilnehmer sowohl in Java, als auch in C und Ruby ausgeprägt sind, kann die Lernkurve der Programmiersprache vernachlässigt werden. Auf Grund der Freiheiten des Programmierers erhalten C++ Engines einen kleinen Vorteil gegenüber C\# Engines.}
			\item[Zusätzliche Software]{Dieses Kriterium beschreibt die Notwendigkeit für weitere Anwendungen, um den Funktionsumfang vollständig nutzen zu können. Engines, welche keine weitern Anwendungen benötigen, behalten ihre Punktzahl. Engines, bei denen zusätzliche Anwendungen nötig sind, erhalten Maluspunkte.}
		\end{description}
		Für die Evaluation wird ein Punktesystem verwendet, welches über fünf Ausprägungen verfügt. Die Ausprägungen sind dabei wie folgt aufgestellt: $ ++ > + > \bigcirc > - > -- $
		Bei Eigenschaften, welche über weniger als fünf mögliche Einteilungen verfügen, kann dieses Punktesystem für eine Skalierung verwendet werden. Wird dieses Punktesystem auf Basis der zuvor beschriebenen Kriterien auf die vier Game-Engines angewandt und skaliert, entsteht folgende Ergebnismatrix:

	\begin{tabl}{lcccc}{Game-Engine Evaluation}
		\toprule
			Kriterium/Engine & Unreal-Engine & Unity & Cry-Engine & Irrlicht-Engine\\
		\midrule
			Androidfähig 			& $++$	& $++$ 		& $--$ 		& $\bigcirc$	\\
			Lizenzen 				& + 	& + 		& + 		& $++$		\\
			Networking				& + 	& $++$ 		& --		& --		\\
			Funktionsumfang 		& $++$ 	& + 		& $++$		& +			\\
			Programmiersprache 		& + 	& $\bigcirc$ 	& +			& +			\\
			Intuitivität 			& + 	& $++$ 		& +			& $++$		\\
			Zusätzliche Software	& -- 	& $\bigcirc$	& $\bigcirc$	& $\bigcirc$	\\
		\hline
			Gesamtpunktzahl			& 7 	& 8			& 2 		& 5			\\
		\bottomrule
	\end{tabl}

	Zur Erläuterung der Skalierung und der Zusammenfassung sollen nun die Plätze 1 und 2 genauer beleuchtet werden.
	Da die Cryengine und die Irrlicht-Engine mit größerem Abstand hinter der Unreal-Engine und dem Gewinner Unity zurückliegen, werden diese in der folgenden Analyse nicht betrachtet.
	\begin{description}
		\item[Androidfähig]{Sowohl die Unreal-Engine, als auch Unity bieten in ihrem Standardumfang die Möglichkeit, Android-Anwendungen zu erstellen und zu deployen. Da in diesem Fall eine der beiden Engines einen Vorteil gegenüber der Anderen hat, erhalten beide die volle Punktzahl.}
		\item[Lizenzen]{Auch im Fall der Lizenzen sind sie die Engines ebenbürtig. Beide Engines sind vollständig kostenlos nutzbar. Bei beiden Engines fällt zudem eine Gewinnbeteiligung von 5\% statt, welche jedoch wie bereits erwähnt vernachlässigt werden kann.}
		\item[Networking]{In der Kategorie Networking gewinnt Unity, da mittels \enquote{Unity Multiplayer} ein Tool im Funktionsumfang enthalten ist, welches die Entwicklung eines Mehrspielermodus stark vereinfacht. In diesem Tool sind Matchmaking-Server enthalten, welche lediglich auf einen eigenen Server aufgesetzt und im Anschluss direkt vewendet werden können. Die Unreal-Engine bietet für einen Multiplayer lediglich eine Unterstützung durch das Framework.}
		\item[Funktionsumfang]{Mit dem kostenlosen Lizenzmodell stellt Unity nur einen beschränkten Funktionsumfang zur Verfügung. Um den vollen Funktionsumfang zu erhalten ist bei Unity die Bezahlung eines Abo-Modells notwendig. Die Unreal-Engine stelle dem Nutzer durchgehend den vollen Umfang zur Verfügung. Aus diesem Grund erhält die Unreal-Engine in dieser Kategorie die volle Punkzahl. Da der von Unity zur Verfügung gestellte Funktionsumfang jedoch für die Entwicklung des Spiels ausreicht, erhält Unity in dieser Kategorie dennoch Punkte.}
		\item[Programmiersprache]{Im Bereich der Programmiersprachen erhält die Unreal-Engine einen Punkt, da für die Programmierung C++ verwendet wird. Unity bekommt wie in der Beschreibung der Sprachen erwähnt, keinen Punkt für die Verwendung von C\#.}
		\item[Intuitivität]{Werden die Unreal-Engine und Unity aus dem Aspekt der Intuitivität betrachtet, fällt schnell auf, dass der größere Funktionsumfang der Unreal-Engine zum Nachteil wird. Durch die vielen Funktionen wirkt die Engine überladen und kompliziert. Dies würde grundsätzlich eine Einordnung in einen deutlich niedrigeren Wertebereich rechtfertigen. Epic-Games, die Entwickler der Unreal-Engine, stellen jedoch eine Vielzahl von Video-Tutorials und Beispielen zur Verfügung, welche dabei helfen die komplexe Gesamtheit der Engine zu überblicken. Unity ist hingegen von der Oberfläche und dem Funktionsumfang übersichtlich und intuitiv nutzbar. Zudem stellt auch Unity Beispiele und Tutorials zur Verfügung. Daraus ergibt sich ein klarer Platz eins für Unity.}
		\item[Zusätzliche Software]{Unity benötigt für die Implementierung eines Spiels keine zusätzliche Software. Alle benötigten Komponenten sind in der Engine direkt enthalten und können über diese aufgerufen werden. Die Unreal-Engine hingegen besitzt keinen eigenen Code-Editor. Möchte ein Entwickler C++ Code schreiben, so wird Visual Studio von Microsoft dafür benötigt. Aus diesem Grund erhält die Unreal-Engine einen Minuspunkt.}
	\end{description}
	Wird die Gesamtheit der Evaluation betrachtet, so erreicht Unity einen Punkt mehr im gewählten Wertesystem, als die Unreal-Engine. Somit wird Unity das Mittel der Wahl und wird für die Implementierung des Spiels verwendet.
	<TODO: 1 Punkt mehr reicht nicht (nicht signifikant), überarbeiten>
