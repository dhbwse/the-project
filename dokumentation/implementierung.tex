% !TEX root = dokumentation.tex
\section{Implementierung}\label{sec:impl}
In diesem Kapitel wird die Entwicklung des Spieles beschrieben.
\subsection{Prototypeing}
	Als Basis wurde ein Beispiel Rennspiel verwendet\footnote{<TODO:paket nennen>}, dieses wurde im Lauf der Entwicklung entsprechend angepasst und erweitert.
	\begin{enumerate}[label=Prototyp \arabic*]
		\item{ Kopie des Beispiel Rennspiels. Kamera über Auto gesetzt. }
		\item{ Steuerung relativ zur Kamera Richtung. }
		\item{ Bremse hinzugefügt, Rückwärtsgang entfernt. }
		\item{ Benutzeroberfläche überarbeitet (Buttons), Tacho Version. }
	\end{enumerate}
\subsection{Entwicklung}
	\subsubsection{Level Design}
		Wie bereits in \ref{par:streckendesign} und \ref{par:streckendesign2} beschrieben sollen die Rennstrecken schlauchartig...
	\subsubsection{UI Design}
	\subsubsection{Physik und Steuerung}
	\subsubsection{}
\subsection{Implementation in Game-Engine X}
\subsection{Class Diagram, DB}
\subsection{Wie wurden gewisse Dinge umgesetzt}
\subsection{Integration in Lernplattform}
