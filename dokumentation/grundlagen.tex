% !TEX root = dokumentation.tex
\section{Grundlagen}
	In diesem Kapitel werden die Theoretischen Grundlagen der Arbeit aufgezeigt.
\subsection{Android als Zielplattform erklären}
\subsection{Game-Engines}
	Eine Game-Engine soll die Entwicklung von Spielen erleichtern. Dies funktioniert in dem Entwicklern Arbeit abgenommen wird Beispielsweise in dem eine einfach zu nutzende Physik-Komponente oder ein \gls{IDE} bereitgestellt wird. Eine Game-Engine stellt oft verwendete Funktionen Spielentwicklern zu Verfügung und beschleunigt die Entwicklung.
	\subsubsection*{Unity}
	\subsubsection*{Unreal}
	\subsubsection*{Cryengine}
	\subsubsection*{mehr?}
\subsection{Persona}\label{ssec:persona}
	Ein Persona ist eine Technik aus dem UI Bereich <TODO:Quelle> wo durch imaginäre Nutzer (Personas) die Useability verbessert werden soll. Ziel dieser Technik ist es einen möglichst konkreten Nutzer zu erstellen und die Software dann durch Imitation eines Persona's zu betrachten. Ergebnisse können unter anderen Bewertungen, Verbesserungen oder auch Entscheidungen sein.
	Persona's gehören zu den benutzerorientierten Designmethoden und sind einfach zu implementieren da kein echter Benutzer involviert wird.

	In der Wissenschaft sind Persona's umstritten.\footcite{persona-crit}\footcite{persona-crit2}\footcite{persona-crit3}
\subsection{Psychologische Grundlagen für Spiele}
\subsection{Gamification}
\subsection{Lernspiele}
	\subsubsection{Lernziele}
	\subsubsection{Lernstrategie}
