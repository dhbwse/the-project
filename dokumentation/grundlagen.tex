% !TEX root = dokumentation.tex
\section{Grundlagen}
	In diesem Kapitel werden die Theoretischen Grundlagen der Arbeit aufgezeigt.
\subsection{Android als Zielplattform erklären}
\subsection{Game-Engines}
	Eine Game-Engine soll die Entwicklung von Spielen erleichtern. Dies funktioniert in dem Entwicklern Arbeit abgenommen wird Beispielsweise in dem eine einfach zu nutzende Physik-Komponente oder ein \gls{IDE} bereitgestellt wird. Eine Game-Engine stellt oft verwendete Funktionen Spielentwicklern zu Verfügung und beschleunigt die Entwicklung.
	\subsubsection*{Unity}
	\subsubsection*{Unreal}
	\subsubsection*{Cryengine}
	\subsubsection*{mehr?}
\subsection{Persona}\label{ssec:persona}
	Ein Persona ist eine Technik aus dem Userexperience (UX) Bereich\footcite{persona}. Personas sollen durch imaginäre Nutzer (Personas) helfen die Nutzerfreundlichkeit zu verbessern. Durch Imitation von Nutzern werden Bewertungen, Verbesserungen oder Entscheidungen aus einer anderen Perspektive generiert. Die Nutzer werden dafür ausführlich beschrieben damit man sich die Person vorstellen kann. Oft werden Verhalten, Ziele, Aussehen, Erfahrungen, Name, Umfeld und weiteres definiert.
	Personas gehören zu den benutzerorientierten Designmethoden und sind einfach zu implementieren da kein echter Benutzer involviert wird.
	\subsubsection{Kritik an Personas}
		In der Wissenschaft sind Personas umstritten.\footcite{persona-crit}\footcite{persona-crit2}\footcite{persona-crit3}\footcite{persona-crit4} Die drei Argumente gegen Personas sind:
		\begin{enumerate}
			\item{ Fehlender Zusammenhang zwischen Nutzergruppe und Persona: Chapman und Milham\footcite{persona-crit} stellen die Frage: \enquote{Wie viele Nutzer werden von einem Persona abgebildet?}. Sie stellen auch stellen fest dass die Wahrscheinlichkeit dass ein Nutzer von einem Persona beschrieben wird abnimmt wenn dem Persona Details hinzugefügt werden. }
			\item{ Kritik an der Umsetzung: Der Berater Steve Portigal merkt an: \enquote{Personas werden missbraucht um von den Nutzern Abstand zu halten}\footcite[Übersetzt: Orginal \enquote{Personas are misused to maintain a “safe” distance from the people we design for [...]}]{persona-crit4}. }
			\item{ ??? }
		\end{enumerate}
	\subsubsection{Play-Personas}
		Die in der Arbeit eingesetzten Play-Personas\footcite{play-persona} sind eine Variante der Personas welche sich direkt auf die Spielmechaniken bezieht und so die Abdeckung der Nutzergruppe erhöht. Für jede zentrale Spielmechanik wird dabei ein Wertebereich vergeben (Beispielsweise: Akrobatik: \{gut, mittel, schlecht\}). Dann werden für alle Kombinationen Personas angelegt und ausgearbeitet. So ist für jede Kombination ein Persona vorhanden und alle Spielmechaniken abgedeckt.
\subsection{Psychologische Grundlagen für Spiele}
\subsection{Gamification}
\subsection{Lernspiele}
	\subsubsection{Lernziele}
	\subsubsection{Lernstrategie}
