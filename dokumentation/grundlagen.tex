% !TEX root = dokumentation.tex
\section{Grundlagen}
	In diesem Kapitel werden die Theoretischen Grundlagen der Arbeit aufgezeigt.
\subsection{Android als Zielplattform erklären}
\subsection{Game-Engines}
	Eine Game-Engine soll die Entwicklung von Spielen erleichtern. Dies funktioniert in dem Entwicklern Arbeit abgenommen wird Beispielsweise in dem eine einfach zu nutzende Physik-Komponente oder ein \gls{IDE} bereitgestellt wird. Eine Game-Engine stellt oft verwendete Funktionen Spielentwicklern zu Verfügung und beschleunigt die Entwicklung.
	\subsubsection{Unity}
	\subsubsection{Unreal}
	\subsubsection{Cryengine}
	\subsubsection{mehr?}
\subsection{Persona}\label{ssec:persona}
	Ein Persona ist eine Technik aus dem UI Bereich wo durch imaginäre Nutzer (Personas) die Useability verbessert werden soll.
\subsection{Gamification}
	modell selbst machen leicht mit top games der zielgruppe und zielgruppenanalyse
	Psychologische Grundlagen für Spiele?
	Vorstellung und generelles pro contra
\subsection{Lernziele definition}
