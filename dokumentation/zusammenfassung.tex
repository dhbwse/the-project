% !TEX root = dokumentation.tex
\section{Zusammenfassung}
Im letzten Kapitel der Studienarbeit soll zunächst ein abschließendes Fazit formuliert werden. Im Anschluss wird ein kurzer Ausblick gegeben, welcher über den weiteren Verlauf des Projekts informiert.

\subsection{Fazit}
<<<<<<< Updated upstream
	Dadurch bedingt das ein Fazit eher subjektiv ist haben wir uns entschieden, dass beide Autoren ein persönliches Fazit verfassen.
	\subsubsection{Fazit Mahler}
	\subsubsection{Fazit Zerulla}
=======
Wie dem Ergebnis der Evaluation entnommen werden kann, ist die Entwicklung des Spiels erfolgreich verlaufen. Trotz der bereits erwähnten Fehler im Zeitmanagement konnten die gestellten Anforderungen fast vollständig realisiert werden. 


Im Vergleich zur letzten Arbeit, in welcher der Fokus auf der Testphase der Softwareentwicklung
lag, betrachtete diese Arbeit den Entwicklungsprozess und alle Schritte,
welche zur Entwicklung von Software innerhalb der Fiducia & GAD IT AG notwendig
sind. Zusätzlich zu den nötigen Schritten sollten auch Informationswege betrachtet
werden, welche zur Übergabe von Aufgaben an andere Projektmitarbeiter / andere
Abteilungen dienen.
Die Möglichkeit, auf das Wissen aus der letzten Praxisphase zurückzugreifen konnte
umfangreich genutzt werden, da erneut mit den beiden gängigen Frameworks innerhalb
der Fiducia & GAD IT AG gearbeitet werden konnte. Durch die letzte Praxisphase
war auch ein thematischer Einstieg sehr viel einfacher und konnte schneller
erreicht werden, da das Grundwissen über die Anwendung und die Frameworks bereits
gegeben war. Durch die Entwicklungsaufgaben im Zuge der Erfüllung der EUHypothekarkredit-
Richtlinie konnte das bereits vorhandene Wissen weiter ausgebaut
und der Umgang mit den Frameworks in der Entwicklung vertieft werden.
Durch die aktive Beteiligung im Entwicklungsalltag der Abteilung konnte viel über die
Abläufe innerhalb des Unternehmens gelernt werden. Auch wurde durch die Übergabe
von Aufgaben viel Kontakt zu anderen Bereichen des Unternehmens, wie z.B. zu
PPM hergestellt werden, um die Hintergründe zu verstehen.
Durch die Loslösung vom strikten Zeitrahmen der Entwicklungs- und Testphasen konnte
eine Aufgabenstellung geschaffen werden, welche vom klassischen Zeitrahmen abweicht
und Weiterentwicklungen innerhalb der Testphase ermöglicht, ohne Unterbrechungen
durch Bugs und Tickets zu erhalten.
4 Fazit und Ausblick Seite 33
Die Durchführung der Entwicklungsaufgaben selbst kann als erfolgreich betrachtet
werden. Alle gestellten Aufgaben konnten zur Zufriedenheit der Verantwortlichen
umgesetzt werden und alle Code Reviews und Tests konnten nach kleinen Korrekturen
erfolgreich geschlossen werden. Die Arbeitspakete konnten alle den Status „Geschlossen“
erreichen und werden von den Testern betrachtet. Sollten die Tester keine Mängel
an der erstellten Software finden, werden die programmierten Erweiterungen für das
nächste Upgrade hinzugezogen und mit diesem Upgrade an die Banken ausgeliefert,
um von diesen eingesetzt zu werden.
Der Entwicklungsprozess selbst kann als klar strukturiert angesehen werden. Durch
die kürzliche Einführung von JIRA erhalten alle Arbeitspakete eine Protokollierung
und der aktuelle Status kann jederzeit abgerufen werden. Durch die Einteilung der
Entwicklung anhand des JIRA-Workflow werden auch in Zukunft die Tester anhand
protokollierter Arbeitspakete effizienter arbeiten, da klar ersichtlich ist, welche Änderungen
vorgenommen wurden und wann diese abgeschlossen (und somit testbar)
sind.
Alles in Allem kann somit der gesamte Praxiseinsatz inkl. der Erstellung dieser Praxisarbeit
als Erfolg verbucht werden und erfüllt die Anforderungen, welche zu Beginn an
diese Arbeit gestellt wurden.

>>>>>>> Stashed changes
\subsection{Ausblick}
	Da das Spiel leider nicht komplett fertig entwickelt werden konnte, ist der naheliegende Schritt dies zu tun. Level-Pakete können bereits mit wenigen Kenntnissen des Unity-Editor zusammengestellt werden, so kann das Spiel in Zukunft einfach erweitert werden. Die Anbindung an die Lernplattform ist ein weiterer wichtiger Schritt.
	Da das Lernmaterial vom Spiel unabhängig ist, könnte eine Sammlung für weitere Arbeiten öffentlich gestartet werden, davon profitiert die wissenschaftliche Gemeinde möglicherweise.
	Eine wissenschaftliche Evaluation des Spiels könnte mit dem für Lernspiele üblichem MEEGA\footcite{psych1} Modell angefertigt werden, ausführliche Studien bezogen auf die getroffenen Designentscheidungen sind ebenso interessant.
	Gegebenenfalls kann das Spiel mit Kooperation von Schulen getestet oder sogar produktiv eingesetzt werden.
