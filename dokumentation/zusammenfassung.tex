% !TEX root = dokumentation.tex
\section{Zusammenfassung}
Im letzten Kapitel der Studienarbeit soll zunächst ein abschließendes Fazit formuliert werden. Im Anschluss wird ein kurzer Ausblick gegeben, welcher über den weiteren Verlauf des Projekts informiert.

\subsection{Fazit}ss
Im Laufe der Entwicklung wurden viele Änderungen am ursprünglichen Plan vorgenommen. Diese Anpassungen sind einerseits darauf zurückzuführen, dass geplante Elemente nicht auf die geplante Weise umgesetzt werden können. Andererseits basieren viele getätigte Anpassungen auf den ursprünglichen Fehlern in der Zeitplanung. Für zukünftige Projekte ist eine bessere, realistischere Zeitplanung besonders wichtig. So kann beispielsweise ein Verschieben der Abgabefrist zukünftig vermieden werden.
Keine dieser Änderungen nimmt gravierenden Einfluss auf die grundsätzliche Projektidee. Die zu Beginn der Arbeit definierten Kriterien konnten fast vollständig erfüllt werden. Die in diesem Kapitel genannten Anpassungen beeinflussen den geplanten Zweck des Spiels nicht und beschreiben lediglich technische Anpassungen. Somit wird die Aufgabe grundsätzlich als vollständig erfüllt betrachtet.

	Dadurch bedingt das ein Fazit eher subjektiv ist haben wir uns entschieden, dass beide Autoren ein persönliches Fazit verfassen.
	\subsubsection{Fazit Mahler}
	\subsubsection{Fazit Zerulla}

Wie dem Ergebnis der Evaluation entnommen werden kann, ist die Entwicklung des Spiels erfolgreich verlaufen. Trotz der bereits erwähnten Fehler im Zeitmanagement konnten die gestellten Anforderungen fast vollständig realisiert werden. 




\subsection{Ausblick}
	Da das Spiel leider nicht komplett fertig entwickelt werden konnte, liegt nahe, die Entwicklung in der Zukunft fortzusetzen. Level-Pakete können bereits mit wenigen Kenntnissen des Unity-Editor zusammengestellt werden, so kann das Spiel in Zukunft einfach erweitert werden. Die Anbindung an die Lernplattform ist ein weiterer wichtiger Schritt.
	Da das Lernmaterial vom Spiel unabhängig ist, könnte eine Sammlung für weitere Arbeiten öffentlich gestartet werden, davon profitiert die wissenschaftliche Gemeinde möglicherweise.
	Eine wissenschaftliche Evaluation des Spiels könnte mit dem für Lernspiele üblichem MEEGA\footcite{psych1} Modell angefertigt werden, ausführliche Studien bezogen auf die getroffenen Designentscheidungen sind ebenso interessant.
	Gegebenenfalls kann das Spiel mit Kooperation von Schulen getestet oder sogar produktiv eingesetzt werden.
