% !TEX root = dokumentation.tex
\section{Zusammenfassung}
Im letzten Kapitel der Studienarbeit soll zunächst ein abschließendes Fazit formuliert werden. Im Anschluss wird ein kurzer Ausblick gegeben, welcher über den weiteren Verlauf des Projekts informiert.

\subsection{Fazit}
%Im Laufe der Entwicklung wurden viele Änderungen am ursprünglichen Plan vorgenommen. Diese Anpassungen sind einerseits darauf zurückzuführen, dass geplante Elemente nicht auf die geplante Weise umgesetzt werden können. Andererseits basieren viele getätigte Anpassungen auf den ursprünglichen Fehlern in der Zeitplanung. Für zukünftige Projekte ist eine bessere, realistischere Zeitplanung besonders wichtig. So kann beispielsweise ein Verschieben der Abgabefrist zukünftig vermieden werden.
%Keine dieser Änderungen nimmt gravierenden Einfluss auf die grundsätzliche Projektidee. Die zu Beginn der Arbeit definierten Kriterien konnten fast vollständig erfüllt werden. Die in diesem Kapitel genannten Anpassungen beeinflussen den geplanten Zweck des Spiels nicht und beschreiben lediglich technische Anpassungen. Somit wird die Aufgabe grundsätzlich als vollständig erfüllt betrachtet.

Im Laufe der Entwicklung wurden viele Änderungen am ursprünglichen Plan vorgenommen. Diese Anpassungen sind einerseits darauf zurückzuführen, dass Elemente nicht auf die ursprünglich geplante Weise umgesetzt werden können, andererseits basieren viele getätigte Anpassungen auf Fehlern in der Zeitplanung. Für zukünftige Projekte ist eine bessere, realistischere Zeitplanung besonders wichtig. So kann beispielsweise ein Verschieben der Abgabefrist zukünftig vermieden werden.
Keine dieser Änderungen nimmt gravierenden Einfluss auf die grundsätzliche Projektidee. Die zu Beginn der Arbeit definierten Kriterien konnten fast vollständig erfüllt werden. Die in der Evaluation genannten Anpassungen beeinflussen den geplanten Zweck des Spiels nicht und beschreiben lediglich technische Anpassungen. Somit wird die Aufgabe grundsätzlich als vollständig erfüllt betrachtet.
Wichtig ist, zu vermerken, dass das entwickelte Spiel alleinstehend keine Schulbildung vermitteln oder ersetzen kann. Im entwickelten Spiel wird Faktenwissen aus der Mathematik der ersten beiden Schulklassen abgefragt und vertieft. Das Verständnis der Kinder, dass \enquote{Zwei + Zwei = Vier} ist, bzw. wie diese Aufgaben zu lösen sind, kann nur schwer auf spielerischem Weg erläutert werden. Um das entwickelte Spiel zum Lernen einzusetzen, müssen die Kinder im Vorfeld die Zusammenhänge der Mathematik erklärt bekommen. Haben die Kinder verstanden, wie einfache Rechenoperationen funktionieren, kann das entwickelte Spiel zum Vertiefen des Wissens eingesetzt werden. So können die Kinder beispielsweise mit steigendem Fortschritt im Spiel ihre Fähigkeit verbessern, gestellte Aufgaben schnell korrekt lösen zu können.
Mit der Durchführung der Arbeit konnten die Autoren viel über spielerisches Lernen und Spieleentwicklung gelernt werden. Da generelles Interesse im Themenbereich der Computerspieleentwicklung besteht, ist diese Arbeit ein guter Einstieg in spätere private Projekte. Die Erfahrungen, welche bei der Entwicklung des Spiels gemacht wurden, können nahtlos in andere Projekte übertragen werden, um eventuell auftretende Probleme unmittelbar zu lösen oder sogar zu vermeiden. Außerdem konnte die Studienarbeit Aufschlüsse über die Plattformwahl für zukünftige Spiele geben. Im Rahmen der Arbeit ist deutlich geworden, dass bei der Spieleentwicklung für mobile Endgeräte besonders die Performance stark in den Vordergrund tritt. Für zukünftige Projekte wird voraussichtlich der PC als Plattform gewählt werden. Damit erhält der Entwickler größere Freiheit im Umgang mit Performance, da PCs leistungsstärker als mobile Endgeräte sind.
Mit der weitgehenden Fertigstellung des Spiels konnte der Nachwies erbracht werden, dass Spielspaß und Lernerfolg miteinander vereinbar sind. Das Spiel kann potentiell als \enquote{gutes Beispiel} vorangehen, um die Entwicklung zukünftiger Lernspiele zu beeinflussen. Besonders relevant wird der Einfluss auf die Industrie jedoch erst dann, wenn das Spiel in Kombination mit der Lernplattform und weiteren \enquote{educational games} dieser Art verfügbar wird.


\subsection{Ausblick}
	Da dem Spiel noch einige kleinere Komponenten, sowie mehr Level-Pakete fehlen, liegt nahe, die Entwicklung in der Zukunft fortzusetzen. Level-Pakete können bereits mit wenigen Kenntnissen des Unity-Editor zusammengestellt werden, so kann das Spiel in Zukunft einfach erweitert werden. Die Anbindung an die Lernplattform ist ein weiterer wichtiger Schritt.
	Da das Lernmaterial vom Spiel unabhängig ist, könnte eine Sammlung für weitere Arbeiten öffentlich gestartet werden, davon profitiert die wissenschaftliche Gemeinde möglicherweise.
	Eine wissenschaftliche Evaluation des Spiels könnte mit dem für Lernspiele üblichem MEEGA\footcite{psych1} Modell angefertigt werden, ausführliche Studien bezogen auf die getroffenen Designentscheidungen sind ebenso interessant.
	Gegebenenfalls kann das Spiel mit Kooperation von Schulen getestet oder sogar produktiv eingesetzt werden.
