% !TEX root = dokumentation.tex
\section{Zusammenfassung}
\subsection{Fazit}
	Dadurch bedingt das ein Fazit eher subjektiv ist haben wir uns entschieden, dass beide Autoren ein persönliches Fazit verfassen.
	\subsubsection{Fazit Mahler}
	\subsubsection{Fazit Zerulla}
\subsection{Ausblick}
	Da das Spiel leider nicht komplett fertig entwickelt werden konnte, ist der naheliegende Schritt dies zu tun. Level-Pakete können bereits mit wenigen Kenntnissen des Unity-Editor zusammengestellt werden, so kann das Spiel in Zukunft einfach erweitert werden. Die Anbindung an die Lernplattform ist ein weiterer wichtiger Schritt.
	Da das Lernmaterial vom Spiel unabhängig ist, könnte eine Sammlung für weitere Arbeiten öffentlich gestartet werden, davon profitiert die wissenschaftliche Gemeinde möglicherweise.
	Eine wissenschaftliche Evaluation des Spiels könnte mit dem für Lernspiele üblichem MEEGA\footcite{psych1} Modell angefertigt werden, ausführliche Studien bezogen auf die getroffenen Designentscheidungen sind ebenso interessant.
	Gegebenenfalls kann das Spiel mit Kooperation von Schulen getestet oder sogar produktiv eingesetzt werden.
