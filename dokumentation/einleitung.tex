% !TEX root = dokumentation.tex
\section{Einleitung}
Sowohl in der Spieleentwicklung, wie auch in vielen anderen Medien ergibt sich die Frage, welchen Mehrwert der Nutzer/Konsument erhält. Im Fall der Spieleentwicklung mag auf den ersten Blick der Eindruck entstehen: \enquote{Spielspaß ist Alles!}.
Allerdings trifft diese Aussage nicht die Realität. Häufig stehen auch Aspekte wie 
\begin{itemize}
	\item Kann der Spieler etwas fürs Leben lernen?
	\item Werden bestimmte Fähigkeiten des Spielers geschult?
	\item Kann das Spiel das Verhalten des Spielers beeinflussen?
\end{itemize}
im Fokus der Entwickler. Generell besteht oftmals das Interesse, die angesprochenen Aspekte mit dem Spielspaß zu kombinieren. Dies gilt besonders bei Spielen, die eher jüngere Altersgruppen als Ziel haben. Durch das spielerische Lernen können die Spieler Fähigkeiten schulen oder etwas fürs Leben lernen, während sie Spaß am Spiel haben. Der Spieler merkt unter Umständen nicht einmal, dass gerade Wissen vermittelt wird.

Das Ziel dieser Studienarbeit ist somit die Entwicklung eines Android basierten Lernspiels. Die Problemstellung liegt dabei darin, sowohl den Spielspaß, als auch den Lernerfolg zu vermitteln. Das erwartete Ergebnis ist zum einen das Release des fertigen Spiels sowie ein umfassendes Wissen über Gamification, Game Engines und Spieleentwicklung.
\subsection{Anforderung}
\subsection{Spielidee}
\subsection{Abgrenzung}
