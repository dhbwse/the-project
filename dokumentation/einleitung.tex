% !TEX root = dokumentation.tex
\section{Einleitung}
Sowohl in der Spieleentwicklung, wie auch in vielen anderen Medien ergibt sich die Frage, welchen Mehrwert der Nutzer/Konsument erhält. Im Fall der Spieleentwicklung mag auf den ersten Blick der Eindruck entstehen: \enquote{Spielspaß ist Alles!}.
Allerdings trifft diese Aussage nicht die Realität. Häufig stehen auch Aspekte wie 
\begin{itemize}
	\item Kann der Spieler etwas fürs Leben lernen?
	\item Werden bestimmte Fähigkeiten des Spielers geschult?
	\item Kann das Spiel das Verhalten des Spielers beeinflussen?
\end{itemize}
im Fokus der Entwickler.
\subsection{Anforderung}
\subsection{Spielidee}
\subsection{Abgrenzung}
