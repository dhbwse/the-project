% !TEX root = dokumentation.tex
\section{Einleitung}
	Üblicherweise geht es in der Spieleentwicklung darum dem Spieler eine bestimmte Emotion zu vermitteln, meistens ist dies Spaß oder Freude\footcite[Abstract]{persona}. Soziologen und Lehrer <TODO:Quelle? und genauer> hatten die Idee dass durch Spiele auch Wissen vermittelt werden kann. Wenn diese Lernspiele <TODO:Beispiele> betrachtet werden fällt oft auf dass der Spielspaß dabei vernachlässigt wurde.

	Kinder zeigen in einem gewissen Alter eine Abneigung gegen Schule und die Vermutung dass dies am Zwang liegt kommt auf<TODO:Quelle?>. Übertragen auf Lernspiele ist die These: wenn das Spiel stetig zum lernen auffordert aber keinen Gegenwert bringt wird dem Spiel auch mit Abneigung begegnet.
	Hier bleibt die Frage stehen ob lernen nicht effektiver wird wenn das Spiel auch Spaß macht.

	Durch das spielerische Lernen können die Spieler Fähigkeiten schulen oder etwas fürs Leben lernen, während sie Spaß am Spiel haben. Können die Lerninhalte erfolgreich mit dem Spielspaß kombiniert werden, merkt der Spieler unter Umständen nicht einmal, dass gerade Wissen vermittelt wird.

	Das Ziel dieser Studienarbeit ist die Entwicklung eines Lernspiels für Android. Die Problemstellung liegt dabei darin, sowohl den Spielspaß, als auch den Lernerfolg zu vermitteln. Das erwartete Ergebnis ist zum einen die Veröffentlichung des Spiels sowie die Erarbeitung und Anwendung von Wissen über Spieleentwicklung, Gamification und Game-Engines.

\subsection{Aufgabenstellung}
	Folgende Anforderungen konnten für die Durchführung der Studienarbeit und die Entwicklung der Software definiert werden:
	\begin{itemize}
		\item{ Die Zielgruppe für das Lernspiel besteht aus Kindern im Alter der ersten bis sechsten Klasse\footnote{Nach deutschem Schulsystem, also Kinder von circa 6 bis 12 Jahren.}. Nach Möglichkeit soll dieses Spiel und die zugehörige Lernplattform auch Flüchtlingskindern zur Verfügung gestellt werden. }
		\item{ Das Spiel soll Android als Zielplattform unterstützen. Dort befindet sich im mobilem Bereich die größte Nutzerbasis. }
		\item{ Es besteht keine Notwendigkeit der Einhaltung der Lernstandards. Die Spieler (Kinder) sollen mit dem Spiel Spaß haben und das Spiel auch spielen \enquote{wollen}. }
		\item{ Als Beispiel wurde \url{https://arcademics.com} genannt, eine Seite welche Spiele der gewünschten Art beinhaltet. }
		\item{ Das Spiel soll teil der Lernplattform von Prof. PhD. Kay Berkling werden und Lernerfolge an diese melden. }
		\item{ Das Spiel soll auch offline spielbar sein. Falls Lernerfolge anfallen sollen diese nachträglich an die Lernplattform gemeldet werden. }
		\item{ Der Spaß steht im Vordergrund und die Lernerfolge sollen unterbewusst stattfinden. }
	\end{itemize}
	Für die Umsetzung der Lerninhalte ist somit ein geeignetes Format zu wählen, welches den Spielspaß nicht mindert und dennoch wichtige Grundfertigkeiten vermittelt.
	Bei der Auswahl der Gamification-Modelle gilt besondere Achtsamkeit bei der Anfälligkeit der Zielgruppe gegenüber den gewählten Methoden. Somit soll das Spiel sowohl die Aufmerksamkeit des Spielers erreichen, sowie diese für einen gesetzten Zeitraum halten.

\subsection{Spielidee}
	Um zur endgültigen Spielidee zu gelangen wurden mehrere Ideen ausgearbeitet (siehe \ref{ssec:idee}) und gegen unsere Kriterien (siehe \ref{ssec:kriterien}) abgewogen.
	Die Finale Spielidee wurde so ein Rennspiel in welchem mit verschiedenen Dingen (Beispielsweise: ein Würfel, ein Auto oder eine Rakete) ein Rennen stattfinden sollen. Um Lerninhalte in ein Rennspiel einzubauen wurden mehrere Möglichkeiten ausgearbeitet:
	\begin{itemize}
		\item{ Aufgabe zum Start des Rennens. Hier kann der schnellste Spieler als erstes starten. Durch eine maximale Zeit und mehrere Versuche sollte Rücksicht auf Spieler genommen werden, welche es nicht schaffen die Aufgabe zu lösen. }
		\item{ Aufgaben in Form von Weggabelungen (wie beispielsweise Tore oder Schanzen). Eine Aufgabe erscheint rechtzeitig\footnote{Was in diesem Kontext rechtzeitig bedeutet wird noch definiert.} vor der Gabelung auf dem Bildschirm. Bei jeder Abzweigungsmöglichkeit steht eine Lösung zur Aufgabe. Fährt ein Spieler falsch  bekommt er einen Nachteil oder alle Spieler die korrekt fahren einen Vorteil. }
		\item{ Learn-To-Win: Spieler können durch lösen von Aufgaben virtuelles Geld verdienen und gewisse Vorteile im Spiel durch Einsatz des Geldes genießen. Diese Idee ist von dem Pay-To-Win Monetarisierungsmodell abgeleitet. }
	\end{itemize}
	Ein wichtiges Kriterium für die Spielidee war, dass Lerninhalte ein Teil vom Spiel sind und so auch das Lernen selbst durch Spaß angeregt wird.
	Das Spiel selbst soll in mindestens zwei Spielmodi spielbar sein:
	\begin{itemize}
		\item{ Online Multiplayer: Rennen fahren mit anderen echten Personen über das Internet. }
		\item{ Offline Zeitrennen: auf Strecken gegen die Zeit fahren. }
		\item{ Optional: Offline gegen Computer: wie das Onlinespiel nur mit Computergegnern. }
	\end{itemize}
	Auf Grafischer ebene soll das Spiel mit einer 3D Grafik entwickelt werden. Die Kamera Ansicht soll von oben in einem leichtem Winkel dem Gefährt folgen.

\subsection{Abgrenzung}
	Das im Rahmen der Studienarbeit entwickelte Spiel wird im App-Store zur Verfügung stehen und in der Lernplattform von Prof. PhD. Kay Berkling integriert. Auf weitere Aspekte von Vermarktung und Vertrieb wird im Rahmen der Studienarbeit verzichtet.
	Im Bereich der Spieleentwicklung wird auf eine eigenständige Entwicklung von Charaktermodellen und Grafiken verzichtet. Diese werden über externe Quellen eingekauft und in die Anwendung integriert.

	Lernaufgaben werden nicht im Rahmen der Arbeit erstellt, Aufgaben sollen durch eine Schnittstelle geladen werden.

\subsection{Vorgehen}
	Auf Basis der Spielidee werden Play-Personas\footcite{persona} ausgearbeitet, danach ein Gamification-Modell auf diese Personas zugeschnitten. Das Gamification-Modell soll Spieler anregen das Spiel länger zu spielen und so mehr zu lernen. Der Fokus liegt dabei stark auf der Verbindung von Gamification und spielerischem Lernen.
	Das Spiel soll in einer gängigen Game-Engine entwickelt werden. Um eine passende Game-Engine zu wählen, wird eine Evaluation durchgeführt (siehe \ref{ssec:engineeval}).
	Die Entwicklung des Spiels geschieht über anfängliches Prototyping und eine anschließenden Implementierung in der Game-Engine. Am Ende der Implementierung soll das entstandene Spiel in die bereits vorhandene Lernplattform von Prof. PhD. Kay Berkling eingegliedert und über diese vertrieben werden.
