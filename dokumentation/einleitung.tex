% !TEX root = dokumentation.tex
\section{Einleitung}
	Üblicherweise versuchen Spieleentwickler dem Spieler eine bestimmte Emotion zu vermitteln, meistens ist dies Spaß oder Freude.\footcite[Abstract]{persona} Dass durch bestimmte Spiele Wissen vermittelt werden kann, ist wohlbekannt.\footcite{generalgameandlearning} Wenn diese Lernspiele betrachtet werden, fällt oft auf, dass der Spielspaß dabei vernachlässigt wird (siehe \ref{sssec:lernspielanalyse}).

	Schon 1980 fragten sich Forscher, was genau Spaß an Computerspielen macht und wie diese Eigenschaften verwendet werden können, um Lernen generell Spaßig zu gestalten.\footcite{learn-game-history} Durch das spielerische Lernen können die Kinder Fähigkeiten schulen, Zusammenhänge erfassen und sich Wissen aneignen, während sie Spaß am Spiel haben. Können die Lerninhalte erfolgreich mit dem Spielspaß kombiniert werden, merkt der Spieler unter Umständen nicht einmal, dass gerade Wissen vermittelt wird.

	Das Ziel dieser Studienarbeit ist die Entwicklung eines Lernspiels für Android. Die Problemstellung liegt darin, Spielspaß und Wissensvermittlung zu kombinieren. Das erwartete Ergebnis ist zum einen die Veröffentlichung des Spiels zum anderen die Erarbeitung und Anwendung von Wissen über Spieleentwicklung, Gamification und Game-Engines. Das Spiel soll durch regelmäßige Tests von Prototypen kontinuierlich evaluiert werden.

\subsection{Aufgabenstellung}
	Folgende Anforderungen werden für die Durchführung der Studienarbeit und die Entwicklung der Software definiert:
	\begin{itemize}
		\item{ Die Zielgruppe für das Lernspiel besteht aus Kindern im Alter der ersten bis sechsten Klasse.\footnote{Nach deutschem Schulsystem, also Kinder von circa 6 bis 12 Jahren.} Nach Möglichkeit soll dieses Spiel und die zugehörige Lernplattform auch Flüchtlingskindern zur Verfügung gestellt werden. }
		\item{ Das Spiel soll Android als Zielplattform unterstützen. Dort befindet sich im mobilen Bereich die größte Nutzerbasis. (siehe \ref{ssec:android}) }
		\item{ Die Lernziele des Spiels sind an den gängigen Lernstandards orientiert, dabei wird darauf geachtet von keinem speziellen Lernstandard abhängig zu sein. Die Spieler (Kinder) sollen mit dem Spiel Spaß haben und das Spiel auch spielen wollen. }
		\item{ Durch Ändern der Aufgabenliste sollen beliebige Lernstandards erfüllt werden können. }
		\item{ Als Beispiel wurde \url{https://arcademics.com} genannt, eine Seite, welche Spiele der gewünschten Art beinhaltet. }
		\item{ Das Spiel soll Teil der Lernplattform von Prof. PhD. Kay Berkling werden und Lernerfolge an diese melden. }
		\item{ Das Spiel soll auch offline spielbar sein. Die eintretenden Lernerfolge sollen an die Lernplattform übermittelt werden. }
		\item{ Der Spaß steht im Vordergrund und die Lernerfolge sollen unterbewusst stattfinden. }
	\end{itemize}
	Eine vollständige Liste der Kriterien bezüglich des Spiels ist in Kapitel \ref{ssec:kriterien} zu finden.
	Für die Umsetzung der Lerninhalte ist somit ein geeignetes Format zu wählen, welches den Spielspaß nicht mindert und dennoch wichtige Grundfertigkeiten vermittelt.
	Bei der Auswahl der Gamification-Modelle gilt besondere Achtsamkeit bei der Anfälligkeit der Zielgruppe gegenüber den gewählten Methoden. Das Spiel soll die Aufmerksamkeit des Spielers erreichen und diese für einen gesetzten Zeitraum halten.

\subsection{Spielidee}\label{ssec:spielidee}
	Um zur endgültigen Spielidee zu gelangen, werden mehrere Ideen ausgearbeitet (siehe \ref{ssec:idee}) und gegen definierte Kriterien (siehe \ref{ssec:kriterien}) abgewogen.
	Als endgültige Spielidee wird ein Rennspiel ausgearbeitet, in welchem mit verschiedenen Dingen (Beispielsweise: ein Würfel, ein Auto oder eine Rakete) ein Rennen stattfinden soll. Um Lerninhalte in ein Rennspiel einzubauen werden mehrere Möglichkeiten ausgearbeitet:
	\begin{itemize}
		\item{ Aufgabe zum Start des Rennens. Hier kann der schnellste Spieler als Erstes starten. Durch eine maximale Zeit und mehrere Versuche kann der Nachteil beschränkt werden, um schwächere Spieler besser zu inkludieren. }
		\item{ Aufgaben in Form von Weggabelungen (wie beispielsweise Tore oder Schanzen). Eine Aufgabe erscheint, bevor eine Weggabelung erreicht wird, auf dem Bildschirm. Bei jeder Abzweigungsmöglichkeit steht eine Lösung zur Aufgabe. Fährt ein Spieler falsch, bekommt er einen Nachteil oder alle Spieler die korrekt fahren einen Vorteil. }
		\item{ Learn-To-Win: Spieler können durch Lösen von Aufgaben virtuelles Geld verdienen und gewisse Vorteile im Spiel durch Einsatz des Geldes genießen. Diese Idee ist von dem Pay-To-Win Monetarisierungsmodell abgeleitet. }
	\end{itemize}
	Ein wichtiges Kriterium für die Spielidee ist, dass Lerninhalte ein Teil des Spiels sind und so auch das Lernen selbst durch Spaß angeregt wird.
	Das Spiel selbst soll in mindestens zwei Spielmodi spielbar sein:
	\begin{itemize}
		\item{ Online Mehrspieler-Modus: Rennen fahren mit anderen echten Personen über das Internet. }
		\item{ Offline Zeitrennen: auf Strecken gegen die Zeit fahren. }
		\item{ Optional: Offline gegen Computer: wie das Onlinespiel nur mit Computergegnern. }
	\end{itemize}
	Auf grafischer Ebene soll das Spiel mit einer 3D Grafik entwickelt werden. Die Kamera Ansicht soll von leicht schräg oben dem Gefährt folgen.

\subsection{Abgrenzung}
	Das im Rahmen der Studienarbeit entwickelte Spiel wird im App-Store zur Verfügung stehen und in der Lernplattform von Prof. PhD. Kay Berkling integriert. Auf weitere Aspekte von Vermarktung und Vertrieb wird im Rahmen der Studienarbeit verzichtet.
	Im Bereich der Spieleentwicklung wird auf eine eigenständige Entwicklung von Charaktermodellen und Grafiken verzichtet. Diese werden über externe Quellen eingekauft und in die Anwendung integriert.

	Lernaufgaben werden nicht im Rahmen der Arbeit erstellt, Aufgaben sollen durch eine Schnittstelle geladen werden.

\subsection{Vorgehen}
	Um die Spieler langfristig zum Lernen zu motivieren, wird ein Gamification-Modell (siehe \ref{ssec:gamification-modell}) ausgearbeitet, auf die Spieler wird durch Play-Personas\footcite{persona} eingegangen (siehe \ref{ssec:persona}). Somit wird erreicht, dass jeder Spieler-Typ durch eigens für ihn entwickelte Gamification-Methoden motiviert wird. Der Fokus liegt dabei stark auf der Verbindung von Gamification und spielerischem Lernen.
	Eine passende Spielidee zur Erreichung dieser Ziele soll durch Brainstorming erarbeitet werden.
	Das Spiel soll in einer gängigen Game-Engine entwickelt werden. Um eine passende Game-Engine zu wählen, wird eine Evaluation durchgeführt (siehe \ref{ssec:engineeval}).
	Die Entwicklung des Spiels geschieht über anfängliches Prototyping und eine anschließenden Implementierung in der Game-Engine. Am Ende der Implementierung soll das entstandene Spiel in die bereits vorhandene Lernplattform von Prof. PhD. Kay Berkling eingegliedert und über diese vertrieben werden.
