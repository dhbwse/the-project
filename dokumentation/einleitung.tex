% !TEX root = dokumentation.tex
\section{Einleitung}
	Sowohl in der Spieleentwicklung, wie auch in vielen anderen Medien ergibt sich die Frage, welchen Mehrwert der Nutzer/Konsument erhält. Im Fall der Spieleentwicklung mag auf den ersten Blick der Eindruck entstehen: \enquote{Spielspaß ist Alles!}.
	Allerdings trifft diese Aussage nicht die Realität. Häufig stehen auch Aspekte wie
	\begin{itemize}
		\item Kann der Spieler etwas fürs Leben lernen?
		\item Werden bestimmte Fähigkeiten des Spielers geschult?
		\item Kann das Spiel das Verhalten des Spielers beeinflussen?
	\end{itemize}
	im Fokus der Entwickler. Generell besteht oftmals das Interesse, die angesprochenen Aspekte mit dem Spielspaß zu kombinieren. Dies gilt besonders bei Spielen, die eher jüngere Altersgruppen als Ziel haben. Durch das spielerische Lernen können die Spieler Fähigkeiten schulen oder etwas fürs Leben lernen, während sie Spaß am Spiel haben. Der Spieler merkt unter Umständen nicht einmal, dass gerade Wissen vermittelt wird.

	Das Ziel dieser Studienarbeit ist somit die Entwicklung eines Android basierten Lernspiels. Die Problemstellung liegt dabei darin, sowohl den Spielspaß, als auch den Lernerfolg zu vermitteln. Das erwartete Ergebnis ist zum einen das Release des fertigen Spiels sowie ein umfassendes Wissen über Gamification, Game Engines und Spieleentwicklung.
\subsection{Aufgabenstellung}
	Folgende Anforderungen konnten für die Durchführung der Studienarbeit und die Entwicklung der Software definiert werden:
	\begin{itemize}
		\item Die Zielgruppe für das Lernspiel besteht aus Kindern im Alter der ersten bis sechsten Klasse. Nach Möglichkeit soll dieses Spiel und die zugehörige Lernplattform auch Flüchtlingskindern zur Verfügung gestellt werden.
		\item Das zu entwickelnde Spiel soll Android als Zielplattform unterstützen.
		\item Es besteht keine Notwendigkeit der Einhaltung der Lernstandards. Die Spieler (Kinder) sollen mit dem Spiel Spaß haben und das Spiel auch spielen \enquote{wollen}.
		\item Als Beispiele zur Orientierung wurde \enquote{arcademics.com} genannt, welche Spiele der gewünschten Art beinhaltet, jedoch eine Internetverbindung benötigt. Daraus ergibt sich, dass das Spiel auch offline spielbar sein soll.
	\end{itemize}
	Auf Basis der gegebenen Anforderungen an die zu entwickelnde Software ergibt sich, dass das im Rahmen der Studienarbeit zu entwickelnde Spiel ein für Android konzipiertes Lernspiel sein soll. Dabei steht der Spaß im Vordergrund und die Lernerfolge sollen unterbewusst stattfinden. Für die Umsetzung der Lerninhalte ist somit ein geeignetes Format zu wählen, welches den Spielspaß nicht mindert und dennoch wichtige Grundfertigkeiten vermittelt.
	Bei der Auswahl der Gamification-Modelle gilt besondere Achtsamkeit bei der Anfälligkeit der Zielgruppe gegenüber den gewählten Methoden. Somit soll das Spiel sowohl die Aufmerksamkeit des Spielers erreichen, sowie diese für einen gesetzten Zeitraum halten.
\subsection{Spielidee}
	Um zur endgültigen Spielidee zu gelangen wurden mehrere Ideen ausgearbeitet (siehe \ref{ssec:idee}) und gegen unsere Kriterien (siehe \ref{ssec:kriterien}) abgewogen.
	Die Finale Spielidee wurde so ein Rennspiel in welchem statt wie normal nur Autos auch eine bunte Mischung an Dingen (Beispielsweise: ein Würfel, ein Jogger oder eine Rakete) am Rennen teilnehmen sollte. Um Lerninhalte in ein Rennspiel einzubauen wurden mehrere Möglichkeiten ausgearbeitet:
	\begin{itemize}
		\item{ Aufgabe zum Start des Rennens. Hier kann der schnellste Spieler als erstes starten. Durch eine maximale Zeit und mehrere Versuche sollte Rücksicht auf Spieler genommen werden, welche es nicht schaffen die Aufgabe zu lösen. }
		\item{ Aufgaben in Form von Weggabelungen (wie beispielsweise Tore oder Schanzen). Eine Aufgabe erscheint rechtzeitig\footnote{Was in diesem Kontext rechtzeitig bedeutet wird noch definiert.} vor der Gabelung auf dem Bildschirm. Bei jeder Abzweigungsmöglichkeit steht eine Lösung zur Aufgabe. Fährt ein Spieler falsch  bekommt er einen Nachteil oder alle Spieler die korrekt fahren einen Vorteil. }
		\item{ Learn-To-Win Spieler können durch lösen von Aufgaben virtuelles Geld verdienen und gewisse Vorteile im Spiel durch Einsatz des Geldes genießen. Diese Idee ist von dem Pay-To-Win monetarisierungs Modell abgeleitet. }
	\end{itemize}
	Ein wichtiges Kriterium für die Spielidee war dass Lerninhalte ein Teil vom Spiel sind und so auch das Lernen selbst durch Spaß angeregt wird.
	Das Spiel selbst soll in mindestens zwei Spielmodi spielbar sein:
	\begin{itemize}
		\item{ Online Multiplayer: rennen fahren mit anderen echten Personen über das Internet. }
		\item{ Offline Zeitrennen: auf Strecken gegen die Zeit fahren. }
		\item{ Optional: Offline gegen Computer: wie das Onlinespiel nur mit Computergegnern. }
	\end{itemize}
\subsection{Abgrenzung}
