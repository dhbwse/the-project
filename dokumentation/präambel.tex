% !TEX root = dokumentation.tex
\documentclass[fontsize=12,fleqn,bibliography=totoc]{scrartcl}

\usepackage{polyglossia}
\setdefaultlanguage[variant=german,spelling=new]{german}

\usepackage[hyphens]{url}
\usepackage[hyperfootnotes=false]{hyperref}
\hypersetup{
	colorlinks=true,
	allcolors=black,
	linktoc=all,
	pdfauthor=Marc Mahler \& Marvin Zerulla,
	pdfencoding=auto
}

%\usepackage{fontspec}
\usepackage{microtype}
% \setmainfont{comicsansms}
% \setsansfont{comicsansms}
% \setmonofont{comicsansms}

\usepackage{booktabs}	% better tables with \toprule \midrule and \bottomrule

\usepackage[yyyymmdd]{datetime}
\renewcommand{\dateseparator}{.}

\usepackage[
	backend=biber,
	%maxnames=7,  % hack for many names
	date=year,
	bibstyle=alphabetic
]{biblatex}
\addbibresource{quellen.bib}
%\appto{\bibsetup}{\raggedright}	% when biblography makes problems
\renewcommand{\subtitlepunct}{: }

\usepackage{scrpage2}
\automark{section}
\clearscrheadings
%\ihead{TITLE?}
\ohead{\rightmark}
\cfoot{\pagemark}

\usepackage[xindy,acronym,nopostdot]{glossaries}


\setlength{\parindent}{0pt}			% kein einzug
\setlength{\parskip}{2.5pt}			% sanfter vertical space zwischen pars
\setlength{\mathindent}{0pt}		% kein mathe einzug


\hyphenation{} % sammlung von hypenation patterns

