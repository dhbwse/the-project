% !TEX root = Projektmanagement/projectmanagement.tex
\textbf{Zweites Teammeeting:} (2016-11-17)\\
Die Tagesordnung für das zweite Teammeeting bestand aus folgenden Punkten:
\begin{itemize}
	\item{Verfassen eines aussagekräftigen Titels}
	\item{Verfassen von Requirements und Usecases}
	\item{Vorauswahl von Game-Engines}
	\item{Heraussuchen von Models und Grafiken}
\end{itemize}

Zum Verfassen des Titels wurden zunächst mehrere Möglichkeiten gesammelt:
\begin{itemize}
	\item{Konzeption und Entwicklung eines Computerspiels um Schulwissen spielerisch zu lehren}
	\item{Schulwissen lernen durch ein Android basiertes Spiel}
	\item{Vermittlung von Schulwissen durch ein Android Spiel}
	\item{Vermittlung von Schulwissen durch spielerisches lernen auf Android}
	\item{Spielerisch Schulwissen vermitteln auf Android - Konzeption und Entwicklung eines educational games}
	\item{Konzeption und Entwicklung eines educational games - Spielerisch Schulwissen vermitteln auf Android}
\end{itemize}
Der zuletzt genannte Titel hat dabei den besten Anklang bei beiden Teammitgliedern gefunden, daher wurder er als finaler Titel gewählt.

Bezüglich der Models und Grafiken wurd beschlossen, einen externen Grafiker für die Gestaltung der Spielwelt hinzuzuziehen. Weiterhin sollen Models verwendet werden, welche bereits im Besitz der DHBW sind um die entstehenden Kosten so gering wie möglich zu halten.
Daraufhin wurde eine Nachricht an einen entsprechenden Grafiker verfasst, welcher den Auftrag angenommen hat. Dieser wird bei Zeiten einige Skizzen einreichen, aus welchen das fertige Spieldesign gewählt werden soll.

Die Punkte \enquote{Verfassen von Requirements und Usecases}, sowie \enquote{Vorauswahl der Game-Engines} wurden aus Zeitgründen vertagt.
