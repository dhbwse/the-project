% !TEX root = Projektmanagement/projectmanagement.tex
\textbf{Kickoff-Meeting:} (2016-11-03)\\
Folgende Fragen waren vorbereitet, um sie Prof. PhD. Kay Berkling zu stellen:
\begin{itemize}
	\item Welche Anforderung müssen wir für Note 1 erfüllen?
	\item Welche Zielvoraussetzungen sind für das Spiel gesetzt?
	\item Für wieviel Spielzeit soll das Spiel den Spieler unterhalten?
	\item Für welche Altersklasse soll das Spiel sein?
	\item Welche Art Lernmöglichkeit soll implementiert werden?
	\item Sind die Lernstandards einzuhalten?
\end{itemize}

Auf Basis dieser Fragen konnten folgende Grundsätze für die Studienarbeit festgelegt werden:
\begin{enumerate}
	\item Die Bewertung der Studienarbeit wird anhand des Excel-Sheets zur Bewertung von Praxis-, Bachelor- und Studienarbeiten durchgeführt.
	\item Das Layout sowie die Outline der Studienarbeit und der Terminplan sind vorher mit Prof. PhD. Kay Berkling zu besprechen.
	\item Die exakten Kriterien für die Erfüllung der Anforderungen befindet sich auf der eigens dafür eingerichteten Seite \enquote{http://dhbwstudienarbeit.pbworks.com/}.
	\item Um sicherzustellen, dass die Studienarbeit im gesetzten Zeitrahmen erfolgreich abgeschlossen wird, sollen bis zum 15. Januar bereits 80\% des Textes verfasst sein. Die Evaluation über den Erfolg der Arbeit stellt die fehlenden 20\%.
	\item Die fertige Studienarbeit ist 4 Wochen vor dem eigentlichen Endtermin (also am 17.04.2017) vollständig bei Prof. PhD. Kay Berkling abzugeben.
	\item Beim Inhalt der Studienarbeit ist auf Wissenschaftlichkeit zu achten. Zum Belegen von Thesen etc. ist eine Bibliographie von min. zwei Seiten zu liefern.
	\item Es besteht keine Notwendigkeit der Einhaltung der Lernstandards. Die Spieler (Kinder) sollen mit dem Spiel Spaß haben und das Spiel auch spielen \enquote{wollen}.
	\item Die Zielgruppe für das Lernspiel besteht aus Kindern im Alter der ersten bis sechsten Klasse. Nach Möglichkeit soll dieses Spiel und die zugehörige Lernplattform auch Flüchtlingskindern zur Verfügung gestellt werden.
	\item Als Beispiele zur Orientierung wurde \enquote{arcademics.com} genannt, welche Spiele der gewünschten Art beinhaltet, jedoch eine Internetverbindung benötigt. Daraus ergibt sich, dass das Spiel auch offline spielbar sein soll.
\end{enumerate}



